\documentclass[11pt,a4paper]{book}

\usepackage{chessboard} %add chess diagrames
\usepackage{pifont} %add graphics for chessboards
\usepackage{pgfcore} %add base shapes for chessboards
%\usepackage{pgfbaseshapespackage} %add arrows for chessboards
\usepackage{skak} %allows the playing of games to be rendered by chessboard.
\usepackage{wrapfig} % add text wrapping to figures

\usepackage{titlesec} %update titles fonts
\usepackage{titling}
\usepackage{fontspec}

\usepackage{graphicx} % used for titlepage image

%Specify our main font
\setmainfont{FiraSans}[
Path			=/usr/share/fonts/firasans/ , %Change to your local directory
Extension	=.ttf ,
UprightFont	=*-Regular ,
BoldFont		=*-Bold ,
ItalicFont	=*-LightItalic
]

%Speficy fonts for headings
\newfontfamily\headingfont{Cinzel}[
Path			=/usr/share/fonts/cinzel/ , %Change to your local directory
Extension	=.otf ,
UprightFont	=*-Regular ,
BoldFont		=*-Bold
]

\titleformat{\part}[display]{\huge\headingfont}{\partname\ \thepart}{20pt}{\Huge}
%\titleformat*{\part}{\LARGE}
\titleclass{\part}{page}
\titleformat{\chapter}[display]
  {\LARGE\headingfont}{\chaptertitlename\ \thechapter}{16pt}{\LARGE}
%\titleformat*{\chapter}{\LARGE\headingfont}
\titleformat*{\section}{\large\headingfont}
\titleformat*{\subsection}{\normalfont\headingfont}
\renewcommand{\maketitlehooka}{\headingfont} %change font for the title page

\pagenumbering{roman} % everything pre TOC in roman neumerals


\begin{document}
\title{Chess Fundamentals}
\author{\includegraphics[width=0.8\textwidth]{Image/Ttl_Img.JPG}
\\
José Raúl Capablanca}
\date{1921}
\maketitle

\chapter*{Preface}

Chess Fundamentals was first published thirteen years ago. Since then there have appeared at different times a number of articles dealing with the so-called Hypermodern Theory. Those who have read the articles may well have thought that something new, of vital importance, had been discovered. The fact is that the Hypermodern Theory is merely the application, during the opening stages generally, of the same old principles through the medium of somewhat new tactics. There has been no change in the fundamentals. The change has been only a change of form, and not always for the best at that.

In chess the tactics may change but the strategic fundamental principles are always the same, so that Chess Fundamentals is as good now as it was thirteen years ago. It will be as good a hundred years from now; as long in fact as the laws and rules of the game remain what they are at present. The reader may therefore go over the contents of the book with the assurance that there is in it everything he needs, and that there is nothing to be added and nothing to be changed. Chess Fundamentals was the one standard work of its kind thirteen years ago and the author firmly believes that it is the one standard work of its kind now.

J. R. CAPABLANCA

\tableofcontents

\clearpage

\pagenumbering{arabic} %everything after TOC is now numeric page numbers

\part{Introduction and Pieces}

\chapter{Setup}

\section{Layout}

\begin{center}

\setchessboard{normalboard, 
showmover,
moverstyle = triangle}
\newgame
\chessboard[largeboard]
\end{center}

This is the beginning layout of every chess game, there are many important features to understand and ensure that you have correctly set up the game ready to play.
\clearpage

\begin{itemize}
	\item The board consists of ranks and files; ranks are horizontal columns of squares starting at 1 and working though to 8. The files are the vertical columns across the board starting with a and working though from a to h.
	\item When sitting at the board ready to play the white square should always be in the right most square.
	\item The Queen is always on her own colour, Thus if the player is playing as White then the Queen starts on the White square. The King is thus beside her and is on the opposing colour to his own.
\end{itemize}

\section{Notation}

\setlength{\parindent}{2em}
	To make it easy to write down games, replay them and share them people came up with chess notation. Over the years there have been a few different notation systems, some easy for people to read and in more recent times ones that are more easily understood for use by computers. Originally José Capablanca wrote this book in a notation system called \textbf{Descriptive chess notation} as it was in common use when he was actively playing. The \textbf{World Chess Federation} or \emph{Fédération Internationale des Échecs} referred to as \textbf{FIDE} uses \textbf{Algebraic chess notation}, this has become the most commonly used notation system for both humans and computers. For the purposes of this book we shall be using standard algebraic notation.
\par 

As shown a chessboard has eight ranks and eight files, when we have the notation we always declare the piece which is moving then the file and finally the rank of where the piece is moving to. So if we take the following layout:

\begin{center}
\newgame
\styleA
\fenboard{4Q3/8/5k2/8/2K5/8/8/3q4 w - - 0 44}
\chessboard[normalboard,
moverstyle=triangle]
\end{center}

It is the light coloured players turn to move, there are two pieces that can be moved the King which is on \textbf{c4} and the Queen that is on \textbf{e8} from this position we say the light coloured player chose to move the King to position \textbf{c3}. To write this down in notation we would get the following.

\begin{center}
\mainline{44.Kc3}
\end{center}

\begin{wraptable}{r}{0.2\textwidth} 
	\vspace{-1em}
	\begin{tabular}{ | c|c | }
		\hline
	King & K \\
	Queen & Q \\
	Bishop & B \\
	Knight & N \\
	Rook & R \\
		\hline
	\end{tabular}
\end{wraptable}

If we are writing this down on paper we use the following letters to to describe each piece instead of the symbols that we will use throughout this book.

There are some special moves and conditions that also have their own notation that you may see from time to time, each of these we will cover a little later however it does inclue Castling which is shown as \wmove{O-O} or \wmove{O-O-O}, En Passant shown as \pawn \texttimes ep. Check which is shown with a + and finally Checkmate which is shown with a \#. All of these will be fully explained in their own sections later as understanding them is critical to mastery of the game.

\begin{center}
	\begin{tabular}{ |c|c| }
	\hline
	King & \king \\
	Queen & \queen \\
	Bishop & \bishop \\
	Knight & \knight \\
	Rook & \rook \\
	Pawn & \pawn \\
	Castle Kingside & \wmove{O-O} \\
	Castle Queenside & \wmove{O-O-O} \\
	Pawn takes En Passant & \wmove{fxe6}\\
	Pawn promoted to Queen & \wmove{g8=Q} \\
	Black to move & \ldots \\
	Good move & ! \\
	Excellent move & !! \\
	Bad Move & ? \\
	Terrible move (blunder) & ?? \\
	Interesting move & !? \\
	Dubious move & ?! \\
	\hline
	\end{tabular}
\end{center}


\chapter{Movement}
\section{Capturing}
\newgame
\styleA
\fenboard{1k3r2/8/8/8/K7/8/5Q2/8 b - - 0 1}
\chessboard[smallboard,
marginleft=false,
marginrightwidth=2em,
marginbottomwidth=2em,
moverstyle=triangle,
pgfstyle=straightmove,
markmoves={f8-f2}]

\begin{wraptable}{r}{0.5\textwidth}
	\vspace{-15em}

Capturing is just the fancy term used to describe when a piece attacks another and takes its position. That is to say if we have a piece and in accordance to its movement rules, assumes the position of a piece of the opposing army then we say that piece has been captured. In the classic game of chess you cannot capture your own pieces.

\end{wraptable}

In our illustration the Rook will capture the Queen as it is blacks turn to move. It will move onto the spot where the White Queen currently occupies. This will be recorded as \mainline{1... Rxf2}.

\begin{center}
	\chessboard[smallboard,
moverstyle=triangle]
\end{center}

\section{Pawn}

Pawns are the most numerous of the pieces on the chessboard, they are also the pieces that have the most complicated rules around them. The pawns start on the second rank for White and the seventh for Black.

\newgame
\styleA
\fenboard{8/8/8/8/8/8/2PP4/8 w - - 0 1}
\chessboard[smallboard,
marginleft=false,
marginrightwidth=2em,
marginbottomwidth=2em,
moverstyle=triangle,
pgfstyle=straightmove,
markmoves={c2-c4, d2-d3}]
\begin{wraptable}{r}{0.5\textwidth}
	\vspace{-15em}

On the first move for any Pawns journey the player has the option of moving one square forward or two squares forward. This option is only available for the first move! Contrary to every other piece in the game how it captures is different from how it regularly moves. This can take some time for new players to get comfortable with.

\end{wraptable}

In this position you can see that White's Pawn can capture the Black Pawn on e5 or it can move forward to d5. There is currently no piece on c5 however if there was this would give the White Pawn the choice to capture on c5 or on e5 or simply to move forward to d5. There are two other concepts about Pawn movement that are so special we shall cover them a little later in this chapter.

\begin{center}
\newgame
\styleA
\fenboard{8/8/8/4p3/3P4/8/8/8 w - - 0 1}
\chessboard[normalboard,
marginleft=false,
marginleftwidth=2em,
moverstyle=triangle,
pgfstyle=straightmove,
markmoves={d4-e5, d4-d5},
pgfstyle=circle,
padding=-0.2em,
markfields=c5]
\end{center}

\section{Rook}

\begin{center}
\newgame
\styleA
\fenboard{8/8/8/4R3/8/8/8/8 w - - 0 1}
\chessboard[normalboard,
moverstyle=triangle,
pgfstyle=straightmove,
markmoves={e5-e8, e5-h5, e5-e1, e5-a5}]
\end{center}

The Rook is perhaps the most straight forward of all the pices in terms of its movement. Simply put it travels in a straight line from where it starts to its destination. It has no special rules about capturing, it does have a special move it can do with the King however we will cover that a little later.

\section{Bishop}

\begin{center}
\newgame
\styleA
\fenboard{8/8/8/3BB3/8/8/8/8 w - - 0 1}
\chessboard[normalboard,
moverstyle=triangle,
pgfstyle=straightmove,
markmoves={e5-h8, e5-a1, e5-b8, e5-h2, d5-a8, d5-a2, d5-g8, d5-h1}]
\end{center}

The Bishop is straight forward to understand, Bishops move in diagonals and you start with one on the Black squares and on on the White squares.

\section{Queen}
\begin{center}
\newgame
\styleA
\fenboard{8/8/8/4Q3/8/8/8/8 w - - 0 1}
\chessboard[normalboard,
moverstyle=triangle,
pgfstyle=straightmove,
markmoves={e5-h8, e5-a1, e5-b8, e5-h2, e5-e8, e5-h5, e5-e1, e5-a5}]
\end{center}
The Queen has the movement of the Bishop and the Rook, this makes the Queen the piece with the most movement options in the game and thus the most powerful.

\section{Knight}
\begin{center}
\newgame
\styleA
\fenboard{8/8/8/4N3/8/8/8/8 w - - 0 1}
\chessboard[normalboard,
moverstyle=triangle,
pgfstyle=knightmove,
markmoves={e5-c6, e5-d7, e5-f7, e5-g6, e5-g4, e5-f3, e5-d3, e5-c4},
pgfstyle=circle,
padding=-0.2em,
markfields={c6, d7, f7, g6, g4, f3, d3, c4}]
\end{center}

The Knight can take a little while to get used to, for a start it \emph{hopps} over other pieces. That means if there is anything in the way between where it is an where it wants to be, unlike any other piece it just \emph{hopps or jumps} over them and goes straight there. It also has odd movement the easiest way to think about it is that it can move up down left or right one square and then it can make one diagonal move this also looks a bit like an L shape. It is important to note that because of this very particular style of movement if the Knight starts on a White square it will land on a Black square.

\section{King}

\begin{center}
\newgame
\styleA
\fenboard{8/8/8/4K3/8/8/8/8 w - - 0 1}
\chessboard[normalboard,
moverstyle=triangle,
pgfstyle=straightmove,
markmoves={e5-e6, e5-f6, e5-f5, e5-f4, e5-e4, e5-d4, e5-d5, e5-d6}]
\end{center}
The King while being the most important piece in the game as it is what determines checkmate also has very limited movement for this great responsibility. The King can move one square in any direction. The King also has a special move of Castling which is done with the Rook. We will cover this in the Special Movement section.

\section{Special Movement}

\subsection*{Castling}
This is one of the most confused moves in the game of Chess for many reasons. First it is the only time in the game where you move two pieces at once, it is where the King moves two squares and furthermore he \emph{hopps over} the Rook in the process. Sound a little confusing? Ok lets start at the start.
\begin{center}
\newgame
\styleA
\fenboard{8/8/8/8/8/8/8/R3K2R w - - 0 1}
\chessboard[normalboard,
moverstyle=triangle,
pgfstyle=straightmove,
markmoves={e1-c1, e1-g1}]
\end{center}

Castling is where we move the King two spaces toward a chosen Rook then place the Rook on the other side of the King. Here we will show castling Queenside or castling long. \mainline{1.O-O-O}

\begin{center}
\chessboard[normalboard,
moverstyle=triangle]
\end{center}

Unfortunately it is a little more complicated than just performing this move you further have the rule \emph{One may not castle out of, through, or into check.} and finally neither the King or the Rook you wish to castle with may have moved before in the game.
\clearpage

In each of the above situations White cannot castle as it breaks at least one of these rules.
\\

Castling formally has 6 rules:
\begin{enumerate}
	\item The castling must be Kingside or Queenside, thus it must occur on the same rank as the King.
	\item Neither the king nor the chosen Rook have previously moved.
	\item There are no pieces between the King and the chosen rook.
	\item The King is not currently in check.
	\item The king does not pass through a square that is attacked by an enemy piece.
	\item The King does not end up in check.
\end{enumerate}

\begin{figure}
	\begin{minipage}{0.29\textwidth}
		\newgame
		\fenboard{3k4/8/8/5q2/8/8/8/R3K2R w - - 0 1}
		\chessboard[tinyboard,
		marginleft=false,
		marginright=false,
		showmover=false]
	\end{minipage}
	\begin{minipage}{0.29\textwidth}
		\newgame
		\fenboard{3k4/8/8/4r3/8/8/8/R3K2R w - - 0 1}
		\chessboard[tinyboard,
		marginleft=false,
		marginright=false,
		showmover=false]
	\end{minipage}
	\begin{minipage}{0.29\textwidth}
		\newgame
		\fenboard{3k4/8/8/2r5/8/8/8/R3K3 w - - 0 1}
		\chessboard[tinyboard,
		marginleft=false,
		marginright=false,
		showmover=false]
	\end{minipage}
\end{figure}

\subsection*{En-Passant}

This is the other tricky move as it can only occur in special circumstances. From White's perspective if there is a pawn on the fifth rank, and an enemy pawn moves forward two spaces on its first move and this lands the enemy pawn beside yours then you can capture En-Passant by moving your pawn to the diagonal behind the enemy pawn.

\begin{center}
	\newgame
	\fenboard{8/8/8/3Pp3/8/8/8/8 w - - 0 1}
	\chessboard[normalboard,
	moverstyle=triangle,
	pgfstyle=straightmove,
	markmoves={e7-e5, d5-e6},
	pgfstyle=circle,
	padding=-0.2em,
	markfields={e6}]
\end{center}

It is important to note that this move can only be done immediately after the enemy pawn has had its first move two squares forward. 

\section{Winning Conditions}

There are only three ways the game can end with the regular game of chess these are a victory, a loss or a draw. When we introduce time limits to the game things can get a little more complicated so for now we will focus on how we can win the game.

\subsection*{Check}

Check is what is noted when the King is under attack. You cannot leave a king in check it must be resolved before you finish your turn otherwise Checkmate is declared. When your King is in Check you only have three options;

\begin{enumerate}
	\item Move the King to a square where it is not attacked.
	\item Block the attack with another piece.
	\item Capture the piece that has placed your king in Check.
\end{enumerate}

You cannot win the game just by putting the enemy King in Check but you can force the opponent to deal with your threat.
\clearpage

\subsection*{Checkmate}

The goal for any player during the game is to checkmate the opposing King. Checkmate is the special word given when the King is currently attacked and is unable to move to a square that is not attacked or block the attack of the enemy.
When there is no possible way to remove the Check on the King by the end of the turn then Checkmate is declared. This ends the game.
 
\subsection*{Stalemate}

Stalemate is the condition where a King is not able to move because all possible moves would land the King in Check and there is no other legal move available. The King is stuck there is no were to go but it is not currently in check. This results in a draw, neither player wins. This can be be very disappointing for player that placed their opponent in Stalemate as commonly there was another approach that would have resulted in a Checkmate. On the other hand for the player in stalemate this can sometimes be a relief as they avoided a loss.

\subsection*{Draw}

Draw can be established with the previous condition stated of stalemate but also a draw can be declared by agreement. Many competitions will have their own rules around when and how you declare a draw and so it is often best to read up on the rules. In casual friendly play it is common to simply offer a shaking of the hand while stating 'Draw' to make your intentions clear. If the other person shakes your hand then the Draw is agreed to. Neither player wins or looses.

\chapter{Chess in the Modern Era}

José Raúl Capablanca released Chess Fundamentals to the world in 1921, in 2021 one hundred years later the fundamentals of the game are the same just as Capablanca predicted in the Preface. However there has been a few things which have changed how we approach the game of chess particularly at the highest levels of play. Most of what has changed our understanding of the game is the advent of sophisticated computerised Chess engines has let us look deeper into the game than previously available. While this makes little difference to casual playing of the game and still the fundamentals of the game remain largely untouched. However there are some more modern concepts in chess which can be useful to understand to accelerate learning of the game.

\subsection*{Piece Value}

A useful tool to evaluate a position is how much value does each player have. If we give each piece a value then we can compare who has the most potential to win. With sophisticated evaluations we consider a lot more factors like who has best future positions. Lets keep it simple for the moment and we will use the following table to give the pieces some relative values.
\\

\begin{tabular}{ | c|c | }
	\hline
	\king & 3.5 \\
	\queen & 9 \\
	\bishop & 3.25 \\
	\knight & 3 \\
	\rook & 5 \\
	\pawn & 1 \\
	\hline
\end{tabular}
\begin{wraptable}{r}{0.75\textwidth}
	\vspace{-8em}
	
When using chess engines to help us understand a position you can compare how far ahead or behind a position is by using these numbers, Unless you are playing a particular gambit or have a series of forcing moves it is usually best to avoid loosing pieces and thus having a lowered relative score to your opposition.

\end{wraptable}

\clearpage

\subsection*{Opening Books and End Tables}

Another tool we can use to help us is what is referred to as Opening Books and End Tables. These are where we have figured out optimal patterns for the start and end of the game. Openings can be still somewhat open to interpretation but there are some well known traps and useful ideas that many people have found though trial and error or though computer assistance. Some of these will be demonstrated later in this book. The End Tables are similar for the endgame, given a number of pieces in any configuration what are the winning moves for either player, what ends in a draw and what looses and what patterns are required to reach any of these states. You can look these up and powerful chess engines will incorporate these into their processes to enhance the chances of success. While studying endgames it can very informative to look up the position in an endgame table and see if there were some options you overlooked, or an optimisation available.

\subsection*{Computer Analysis}

As previously mentioned some of the biggest advancements to the game has been though computer analysis, this is using some sophisticated techniques to maximise future board positions or relying on the wealth of games that have been played to steer toward a more likely successful outcome. While computer engine programming is its own fascinating field of study for the purposes of study it is safe to assume in almost all positions the computer analysis is going to come up with a series of moves which are most likely to secure a victory (or a draw in the case of a lost position). This is not to say that Chess is a solved game and the analysis is infallible, but be aware that it is a helpful tool in assessing positions and can help you look deeper into a position than you may have been able to before.

\part{Fundamentals and Theory}

\chapter{First Principles: Endings, Middle-Game and Openings}

The first thing a student should do, is to familiarise them self with the power of the pieces. This can best be done by learning how to accomplish quickly some of the simple mates.

\section{Some Simple Mates}

\subsection{The ending Rook and King against King.}

\subsubsection*{Example 1}

\emph{The principle is to drive the opposing King to the last line on any side of the board.}

\begin{wraptable}{r}{0.5\textwidth}
	\vspace{-12em}
	\newgame	
	\styleA
	\fenboard{7k/8/8/8/8/8/8/R6K w - - 0 1}
	\mainline{1.Ra7 Kg8 2.Kg2} In this position the power of the Rook is demonstrated by the first move, \rook\textbf{a7} which immediately confines the Black King to the last rank, and the mate is quickly accomplished by:
\end{wraptable}

\chessboard[smallboard,
setfen = 7k/8/8/8/8/8/8/R6K w,
marginleft=false,
marginrightwidth=2em,
moverstyle=triangle]

\mainline{2... Kf8, 3. Kf3} The combined action of King and Rook is needed to arrive at a position in which mate can be forced. The general principle for a beginner to follow is to keep his King as much as possible on the same rank, or, as in this case, file, as the opposing King. When, in this case, the King has been brought to the sixth rank, it is better to place it, not on the same file, but on the one next to it towards the centre.

\mainline{3... Ke8, 4.Ke4 Kd8, 5.Kd5 Kc8, 6.Kd6} Not \king c6, because then the black king will go back to  \king d8 and it will take much longer to mate. If now the king moves back to \king d8  then \rook h8 mates at once.

\variation{6.Kd6 Kb8, 7.Rf7 Ka8, 8.Kc6 Kb8, 9.Kb6 Ka8, 10.Rf8\#}

It has taken exactly ten moves to mate from the original position. On move 5 Black could have played \king e8 and, according to principle, White would have continued:

\mainline{6... Kd8} the Black King will ultimately be forced to move in front of the White King and be mated by \rook a8

\mainline{7.Ke6 Kg8, 8.Kf6 Kh8, 9.Kg6 Kg8, 10.Ra8\#}

\begin{center}
\chessboard[largeboard,
moverstyle=triangle]
\end{center}

\subsubsection*{Example 2}

\newgame
\styleA
\fenboard{8/8/8/4k3/8/8/8/4K2R w - - 0 1}
\chessboard[smallboard,
marginleft=false,
marginrightwidth=2em,
marginbottomwidth=2em,
moverstyle=triangle]
\begin{wraptable}{r}{0.5\textwidth}
	\vspace{-15em}
	
Since the Black King is in the centre of the board, the best way to proceed is to advance your own King thus:
\mainline{1. Ke2 Kd5 2.Ke3} As the Rook has not yet come into play, it is better to advance the king straight into the centre of the board, not in front, but to one side of the other King. Should now the Black King move \king e4, the rook drive it back by:

\end{wraptable}
\variation{2.Ke3 Ke5, 3.Rh5} On the other hand, if
\mainline{2... Kc4, 3.Rh5} then the sequence is as follows,
\variation{3.Rh5 Kb4, 4.Kd3}; but if instead  
\mainline{3... Kc3, 4.Rh4} keeping the King confined to as few squares as possible.
Now the ending may continue \mainline{4... Kc2, 5.Rc4+ Kb3, 6.Kd3 Kb2, 7.Rb4+ Ka3, 8.Kc3 Ka2}. It should be noticed how often the White King has moved next to the Rook, not only to defend it, but also to reduce the mobility of the opposing King. Now White mates in three moves thus:
\mainline{9.Ra4+ Kb1, 10.Ra3 Kc1, 11. Ra1\#}. It has taken eleven moves to mate, and, under any conditions, I believe it should be done in under twenty.\footnote{Modern computer analysis indicates that the longest sequence of moves until checkmate or the Distance to Mate \textbf{DTM} is 31 moves.} while it may be monotonous, it is worth while for the beginner to practice such things, as it will teach them the proper handling of the pieces.
\begin{center}
\chessboard[normalboard,
moverstyle=triangle]
\end{center}

\subsection{Two Bishops and King against King}

\subsubsection*{Example 3}

\newgame
\styleA
\fenboard{7k/8/8/8/8/8/8/2B1KB2 w - - 0 1}
\chessboard[smallboard,
marginleft=false,
marginrightwidth=2em,
marginbottomwidth=3em,
moverstyle=triangle]
\begin{wraptable}{r}{0.5\textwidth}
	\vspace{-17em}
	Since the Black King is in the corner, White can play
	\mainline{1.Bd3 Kg7, 2.Bg5 Kf7, 3.Bf5}, and already the Black King is confined to a few squares. if the Black King, in the original position, had been in the centre of the board or away from the last row, White should have advanced the King, and then, with the aid of the Bishops, restricted the Black King's movements to as few squares as possible.
\end{wraptable}
We might now continue. \mainline{3... Kg7, 4. Kf2}. In this ending the Black King must not only be driven to the edge of the board, but he must be forced into a corner, and, before a mate can be given, the White King must be brought to the sixth rank and, at the same time, in one of the last two files; in this case either h6, g6, f7, f8 and as h6 and g6 are the nearest squares, it is to either of these squares that the king ought to go to.
\mainline{4... Kf7, 5.Kg3 Kg7, 6.Kh4 Kf7, 7.Kh5 Kg7, 8.Bg6 Kg8, 9.Kh6 Kf8}. White must now mark time and move one of the bishops, so as to force the black king to go back;

\chessboard[smallboard,
marginleft=false,
marginrightwidth=2em,
moverstyle=triangle]
\begin{wraptable}{r}{0.5\textwidth}
	\vspace{-14em}
	\mainline{10.Bh5 Kg8, 11.Be7 Kh8}. Now the White Bishop must take up a position from which it can give check next move along the White diagonal, when the Black King moves back to g8.
	\mainline{12.Bg4 Kg8, 13.Be6+ Kh8, 14.Bf6\#} It has taken fourteen moves to force the mate and, in any position, it should be done in under thirty.\footnotemark
\end{wraptable}
\footnotetext{The DTM for a Bishop and King endgame is 37 moves.}
In all endings of this kind, care must be taken not to drift into a stale mate.
In this particular ending one should remember that the King must not only be driven to the edge of the board, but also into a corner. in all such endings, however it is immaterial whether the King is forced on to the last rank, or to an outside file.

\begin{center}
\chessboard[largeboard,
moverstyle=triangle]
\end{center}
\clearpage

\subsection{Queen and King against King}
\subsubsection*{Example 4}
We now come to the Queen and King against King. As the Queen combines the power of the Rook and the Bishop, it is the easiest mate of all and should always be accomplished in under ten moves.\footnote{DTM is in 19 moves in a King and Queen endgame.} Take the following position:

\newgame
\styleA
\fenboard{8/8/8/4k3/8/8/8/4K2Q w - - 0 1}
\chessboard[smallboard,
marginleft=false,
marginrightwidth=2em,
moverstyle=triangle]
\begin{wraptable}{r}{0.5\textwidth}
	\vspace{-14em}
	A good way to begin is to make the first move with the Queen, trying to limit the Blacxk King's mobility as much as possible thus:
	\mainline{1.Qc6 Kd4, 2.Kd2}. Already the Black King has only one availible square.
	\mainline{2... Ke5, 3.Ke3 Kf5, 4.Qd6 } we consider the variation.
	\variation{4.Qd6 Kg4, 5.Qg6+} however returning to the main line.
\end{wraptable}
\mainline{4...Kg5 5.Qe6 } from here Black has two options.
\variation{5.Qe6 Kh5, 6.Kf4 Qh6\#} and the other option.
\mainline{5... Kh4, 6.Qg6 Kh3, 7.Kf3 Kh4, 8.Qg4\#}.
In this ending, as in the case of the Rook, the Black King must be forced to the edge of the board; Only the Queen being so much more powerful than the rook, the process is far easier and shorter. these are the three elementary endings and in all of these the principle is the same. In each case the cooperation of th eKing is needed. In order to froce a mate without the aid of the King, at least two rooks are required.

\begin{center}
\chessboard[normalboard,
moverstyle=triangle]
\end{center}
\clearpage

\section{Pawn Promotion}
The gain of a Pawn is the smallest material advantage that can be obtained in a game; and it often is sufficient to win, even when the Pawn is the only remaining unit, apart from the Kings. It is essential, speaking generally, that \emph{the King should be in front of his Pawn, with at least one intervening square.} If the opposing King is directly in front of the Pawn, then the game cannot be won. This can best be explained by the following examples.
\subsubsection*{Example 5}
\newgame
\styleA
\fenboard{8/8/8/8/4k3/8/3KP3/8 w - - 0 1}
\begin{center}
\chessboard[largeboard,
moverstyle=triangle]
\end{center}
The position is drawn, and the way to proceed is for Black to keep the King always directly in front of the Pawn, and when it cannot be done, as for instance in this position because of the White King, then the Black King must be kept in front of the White King. The play would proceed thus:
\mainline{1.e3 Ke5, 2.Kd3 Kd5} This is a very important move. Any other move would lose, as will be shown later. As the Black King cannot be kept close up to the Pawn, it must be brought as far forward as possible and, at the same time, in front of the White King.
\mainline{3.e4+ Ke5, 4.Ke3 Ke6, 5.Kd4 Kd6}. Again the same case. As the White King comes up, the Black King must be kept in front of it, since it cannot be brought up to the Pawn.
\mainline{6.e5+ Ke6, 7.Ke4 Ke7, 8.Kf5 Kf7, 9.e6+ Ke7, 10.Ke5 Ke8, 11.Kf6 Kf8}. If now White advances the Pawn, the Black King gets in front of it and White must either give up the Pawn or play \king e6, and a stale mate results. If instead of advancing the Pawn White withdraws his King, Black brings his King up to the Pawn and, when forced to go back, he moves to \king in front of the Pawn ready to come up again or to move in front of the White King, as before, should the latter advance.
The whole mode of procedure is very important and the student should become thoroughly conversant with its details; for it involves principles to be taken up later on, and because many a beginner has lost identical positions from lack of proper knowledge. At this stage of the book I cannot lay too much stress on its importance.
\begin{center}
\chessboard[largeboard,
moverstyle=triangle]
\end{center}
\clearpage

\subsubsection*{Example 6}
In this position White wins, as the King is in front of his Pawn and there is one intervening square.

\newgame
\styleA
\fenboard{8/8/5k2/8/5K2/8/4P3/8 w - - 0 1}
\chessboard[smallboard,
marginleft=false,
marginrightwidth=2em,
marginbottomwidth=2em,
moverstyle=triangle]
\begin{wraptable}{r}{0.5\textwidth}
	\vspace{-15em}
	The method to follow is to \emph{advance the King as far as is compatible with the safety of the pawn and never advance the Pawn until it is essential to its own safety.} Thus:
	\mainline{1.Ke4, Ke6} Black does not allow the White King to advance, therefore White is now compelled to advance his Pawn so as to force Black to move away. He is then able to advance his own King.
\end{wraptable}
\mainline{2.e3 Kf6, 3.Kd5 Ke7} if Black had played \variation{3... Kf5} then White would be forced to advance the Pawn to e4, since he could not advance his King without leaving Black the opportunity to play e4, winning the Pawn. Since he has not done so, it is better for White not to advance the pawn yet, since its own safety does not require it, but to try to bring the king still further forward. Thus:
\mainline{4.Ke5 Kd7, 5.Kf6 Ke8} Now the White Pawn is too far back and it may be brought up within protection of the King.
\mainline{6.e4 Kd7} Now it would not do to play \king f7, because Black would play \king d6 and white would have to bring back his king to protect the Pawn. Therefore he must continue.
\mainline{7.e5 Ke8} Had he moved anywhere else, White could have played \king f7 fowllowed by the advance of the pawn e6, e7, e8; all these squares being protected by the King. As Black tries to prevent that, White must now force him to move away, at the same time always keeping the King in front of the Pawn. Thus:
\mainline{8.Ke6} Moving the Pawn would make it a draw, as Black would then play \king f8, and we would have a similar position to the one explained in connection with \textbf{Example 5.}
\mainline{8... Kf8, 9.Kd7} King moves and the White Pawn advances to K 8, becomes a Queen, and it is all over. This ending is like the previous one, and for the same reasons should be thoroughly understood before proceeding any further.
\begin{center}
\chessboard[tinyboard,
moverstyle=triangle]
\end{center}

\section{Pawn Endings}
\subsection{Some Winning Positions in the End-Game}
I shall now give a couple of simple endings of two Pawns against one, or three against two, that the reader may see how they can be won. Fewer explanations will be given, as it is up to the student to work things out for himself. Furthermore, nobody can learn how to play well merely from the study of a book; it can only serve as a guide and the rest must be done by the teacher, if the student has one; if not, the student must realise by long and bitter experience the practical application of the many things explained in the book.
\subsubsection*{Example 7}
\newgame
\styleA
\fenboard{5k2/6p1/4K1P1/5P2/8/8/8/8 w - - 0 1}
\chessboard[smallboard,
marginleft=false,
marginrightwidth=2em,
marginbottomwidth=2em,
moverstyle=triangle]

\begin{wraptable}{r}{0.5\textwidth}
	\vspace{-13em}
	In this position White cannot win by playing f6 because Black plays, not gxf6, which would lose, but \king g8 and if then gxf6 \king xf6, and draws. White, however, can win the position given the diagram by playing:
\end{wraptable}
\mainline{1.Kd7 Kg8, 2.Ke7 Kh8, 3.f6 gxf6} Here if \variation{3... Kg8, 4.f7+ Kh8, 5.f8=Q\#} Thus \mainline{4.Kf7 f5, 5.g7+ Kh7, 6.g8=Q+ Kh6, 7.Qg6\#}

\begin{center}
\chessboard[normalboard,
moverstyle=triangle]
\end{center}

\clearpage

\subsubsection*{Example 8}

\newgame
\styleA
\fenboard{8/6p1/3k4/8/3K1PP1/8/8/8 w - - 0 1}
\chessboard[smallboard,
marginleft=false,
marginrightwidth=2em,
moverstyle=triangle]
\begin{wraptable}{r}{0.5\textwidth}
	\vspace{-13em}
	
In the above position White can't win by f5. Black's best answer would be g6 draws. (The student should work this out.) He cannot win by g5, because g6 draws. (This, because of the principle of the \emph{"opposition"} which governs this ending as well as all the Pawn-endings already given, 

\end{wraptable}

and which will be explained more fully later on.) White can win, however, by playing: \mainline{1.Ke4 Ke6} if Black however plays \variation{1... g6, 2.Kd4 Ke6, 3.Kc5 Kf6, 4.Kd6 Kf7 5.g5 Kg7, 6.Ke7 Kg8, 7.Kf6 Kh7, 8.Kf7} and White wins the Pawn.
\mainline{2.f5+ Kf6, 3.Kf4 g6} (If this Pawn is kept back we arrive at the ending shown in \textbf{Example 7.}) 
\mainline{4.g5+ Kf7, 5.f6 Ke6, 6.Ke4 Kf7, 7.Ke5 Kf8} White cannot force the f Pawn into the eighth rank (find out why), but by giving his Pawn up he can win the other Pawn and the game. Thus:
\mainline{8.f7 Kxf7, 9.Kd6 Kf8, 10.Ke6 Kg7, 11.Ke7 Kg8, 12.Kf6 Kh7, 13.Kf7 Kh8, 14.Kxg6 Kg8} There is still some resistance in Black's position. In fact, the only way to win is the one given here, as will easily be seen by experiment.
\mainline{15.Kh6 Kh8} if Black instead moves \king h8  in order to win White must get back to the actual position, as against \variation{15... Kh8, 16.g8+ Kh8} draws), 
\mainline{16.g6 Kg8, 17.g7 Kf7, 18.Kh7}, and White queens the Pawn and wins.
\begin{wraptable}{l}{0.5\textwidth}
	\vspace{-12em}
	This ending, apparently so simple, should show the student the enormous difficulties to be surmounted, even when there are hardly any pieces left, when playing against an adversary who knows how to use the resources at his disposal, and it should show the student, also, the necessity of paying strict attention to these elementary things which form the basis of true mastership in Chess.
\end{wraptable}
\chessboard[smallboard,
moverstyle=triangle]

\clearpage

\subsubsection*{Example 9}

\newgame
\styleA
\fenboard{8/6pp/3k4/8/3K1PPP/8/8/8 w - - 0 1}
\chessboard[smallboard,
marginleft=false,
marginrightwidth=2em,
marginbottomwidth=2em,
moverstyle=triangle]

\begin{wraptable}{r}{0.5\textwidth}
	\vspace{-15em}
	In this ending White can win by advancing any of the three Pawns on the first move, but it is convenient to follow the general rule, whenever there is no good reason against it, of \emph{advancing the Pawn that has no Pawn opposing it.} Thus we begin by:
\end{wraptable}

\mainline{1.f5 Ke7} If Black plays g6 followed by f6 and we have a similar ending to one of those shown above. If \variation{1... h6, 2.g5}
\mainline{2.Ke4 Kf7, 3.g5 Ke7} If variation{3... h6, 4.g6+}, and in either case we have a similar ending to one of those already shown.
\mainline{4.h5} and by following it up with g6 we have the same ending previously shown. Should Black play \mainline{4... g6, 5.hxg6 hxg6, 6.f6+} with the same result. Having now seen the cases when the Pawns are all on one side of the board we shall now examine a case when there are Pawns on both sides of the board.
\begin{center}
\chessboard[largeboard,
moverstyle=triangle]
\end{center}
\clearpage

\subsubsection*{Example 10}
\newgame
\styleA
\fenboard{8/6pp/3k4/8/3K1PPP/8/8/8 w - - 0 1}
\chessboard[smallboard,
marginleft=false,
marginrightwidth=2em,
marginbottomwidth=2em,
moverstyle=triangle]
\begin{wraptable}{r}{0.5\textwidth}
	\vspace{-15em}
	In these cases the general rule is to \emph{act immediately on the side where you have the superior forces.} Thus we have: \mainline{1.g4} It is generally advisable to advance the Pawn that is free from opposition. \mainline{1... a5} Black makes an advance on the other side,
\end{wraptable}
and now White considers whether or not he should stop the advance. In this case either way wins, but generally the advance should be stopped when the opposing King is far away.
\mainline{2.a4 Kf6, 3.h4 Ke6}.
If \variation{3... Kg6}, then simple counting will show that White goes to the other side with his King, wins the P at Q R 4, and then Queens his single Pawn long before Black can do the same.
\mainline{4.g5 Kf7, 5.Kf5 Kg7, 6.h5 Kf7} If \variation{6... h6, 7.g6}, and then the two Pawns defend themselves and White can go to the other side with his King, to win the other Pawn.
\mainline{7. Ke5 }Now it is time to go to the other side with the King, win the Black Pawn and Queen the single Pawn. This is typical of all such endings and should be worked out by the student in this case and in similar cases which he can put up. 
\begin{center}
\chessboard[normalboard,
moverstyle=triangle]
\end{center}
\clearpage

\subsection{Some Winning Positions in the Middle-Game}
\subsubsection*{Example 11}
By the time the student has digested all that has been previously explained, he, no doubt, is anxious to get to the actual game and play with all the pieces. However, before considering the openings, we shall devote a little time to some combinations that often arise during the game, and which will give the reader some idea of the beauty of the game, once he becomes better acquainted with it.
\\
\newgame
\styleA
\fenboard{5rk1/1b3p1p/ppq3p1/2p5/8/1P1P1R1Q/PBP3PP/7K b - - 0 1}
\chessboard[smallboard,
marginleft=false,
marginrightwidth=2em,
marginbottomwidth=2em,
moverstyle=triangle]
\begin{wraptable}{r}{0.5\textwidth}
	\vspace{-15em}
	It is Black's move, and thinking that White merely threatens to play \queen h6 and mate at g7, Black plays \mainline{1... Re8}, threatening mate by way of \rook e1. White now uncovers his real and most effective threat, via.:
\end{wraptable}
\mainline{2.Qh7+ Kxh7, 3.Rh3+ Kg8, 4.Rh8\#} This same type of combination may come as the result of a somewhat more complicated position.
\begin{center}
\chessboard[normalboard,
moverstyle=triangle]
\end{center}
\clearpage

\subsubsection*{Example 12}
\newgame
\styleA
\fenboard{5rk1/1bq1bp1p/p1n3p1/1p6/3N4/1P1PR2Q/PBP3PP/7K w - - 0 1}
\chessboard[smallboard,
marginleft=false,
marginrightwidth=2em,
marginbottomwidth=2em,
moverstyle=triangle]
\begin{wraptable}{r}{0.5\textwidth}
	\vspace{-15em}
	White is a piece behind, and unless he can win it back quickly he will lose; he therefore plays: \mainline{1.Nxc6 Bg5} The Knight cannot be taken because White threatens mate by 
	2 \queen \texttimes h7+ \king \texttimes h7, 3 \rook h3+ like the previous example.
	\mainline{2.Ne7+ Qxe7} Again if \variation{2... Bxe7, 3.Qxh7+ Kxh7, 4.Rh3+ Kg8, 5.Rh8\#}.
\end{wraptable}
\mainline{3.Rxe7 Bxe7, 4.Qd7}and White wins one of the two Bishops, remains with a \queen \ and a \bishop \ against a \rook \ and \bishop, and should therefore win easily. These two examples show the danger of advancing Whites g Pawn one square, after having Castled on that side.
\begin{center}
\chessboard[largeboard,
moverstyle=triangle]
\end{center}
\clearpage

\subsubsection*{Example 13}
\newgame
\styleC
\fenboard{2q2rk1/1b2bppp/pp6/2p5/2P1N3/PP1Q4/1B3PPP/7K w - - 0 1}
\chessboard[smallboard,
marginleft=false,
marginrightwidth=2em,
moverstyle=triangle]
\begin{wraptable}{r}{0.5\textwidth}
	\vspace{-15em}
	This is another very interesting type of combination. Black has a \rook \ for a \knight \ and should therefore win, unless White is able to obtain some compensation immediately. White, in fact, mates in a few moves thus:
	\mainline{1.Nf6 gxf6} Forced, otherwise \queen \texttimes h7 mates.
	\mainline{2.Qg3+ Kh8, 3.Bxf6\#}
\end{wraptable}
\\

\subsubsection*{Example 14}
The same type of combination occurs in a more complicated form in the following position.
\\
\newgame
\styleC
\fenboard{2q1rrk1/1bpn1ppp/pp1p4/1B6/4N3/1P1QR3/PBP2PPP/6K1 w - - 0 1}
\chessboard[smallboard,
marginleft=false,
marginrightwidth=2em,
moverstyle=triangle]
\begin{wraptable}{r}{0.5\textwidth}
	\vspace{-15em}
	\mainline{1.Bxd7 Qxd7} if \variation{1... Bxe4, 2.Qc3} Threatens mate, and therefore wins the \queen which is already attacked.
	\mainline{2.Nf6+ gxf6, 3.Rg3+ Kh8, 4.Bcf6\#}
\end{wraptable}
\clearpage

\subsubsection*{Example 15}
A very frequent type of combination is shown in the following position.
\\
\newgame
\styleC
\fenboard{2q1rrk1/1bpn1ppp/pp1p4/1B6/4N3/1P1QR3/PBP2PPP/6K1 w - - 0 1}
\chessboard[smallboard,
marginleft=false,
marginrightwidth=2em,
marginbottomwidth=2em,
moverstyle=triangle]
\begin{wraptable}{r}{0.5\textwidth}
	\vspace{-15em}
	Here White is up the exchange and a Pawn behind, but the game can be won quickly thus:
	\mainline{1.Bh7+ Kxb7} if \variation{1... Kh8, 2.Qh5 g6, 3.Qh6} and wins.
	\mainline{2.Qh5 Kg8, 3.Ng5}
\end{wraptable}
Black cannot stop the mate h7 except by sacrificing the Queen by \queen e4, which would leave White with a \queen \ for a \rook .

\subsubsection*{Example 16}
This same type of combination is seen in a more complicated form in the following position. White proceeds as follows:
\\
\newgame
\styleA
\fenboard{2nb1rk1/1pqrnppp/p2p4/2pQ1N2/6P1/1P1B1N1P/P3RP2/3R2K1 w - - 0 1}
\chessboard[smallboard,
marginleft=false,
marginrightwidth=2em,
moverstyle=triangle]
\begin{wraptable}{r}{0.5\textwidth}
	\vspace{-13em}
	\wmove{Nxe7+} this clears the line for the \bishop;
	\mainline{1.Nxe7+ Bxe7} (to stop the \knight \ from moving to g5 after the sacrifice of the \bishop); 
	\mainline{2.Rxe7 Nxe7, 3.Bxh7+ Kxh7} if \variation{3... Kh8, 4.Qh5 g6, 5.Bxg6+ Kg7, 6.Qh7+ Kf6, 7.g5+ Ke6, 8.Bxf7+ Rxf7, 9.Qe4\#} therefore:
\end{wraptable}
\mainline{4.Qh5+ Kg8, 5.Ng5 Rc8, 6.Qh7+ Kf8, 7.Qh8+ Ng8, 8.Nh7+ Ke7, 9.Re1+ Kd8, 10.Qxg8\#}
\clearpage

\section{Relative Value of the Pieces}

Before going on to the general principles of the openings, it is advisable to give the student an idea of the proper relative value of the pieces. There is no complete and accurate table for all of them, and the only thing to do is to compare the pieces separately.

For all general theoretical purposes the Bishop and the Knight have to be considered as of the same value, though it is my opinion that the Bishop will prove the more valuable piece in most cases; and it is well known that two Bishops are almost always better than two Knights.

The Bishop will be stronger against Pawns than the Knight, and in combination with Pawns will also be stronger against the Rook than the Knight will be.

A Bishop and a Rook are also stronger than a Knight and a Rook, but a Queen and a Knight may be stronger than a Queen and a Bishop.

A Bishop will often be worth more than three Pawns, but a Knight very seldom so, and may even not be worth so much.

A Rook will be worth a Knight and two Pawns, or a Bishop and two Pawns, but, as said before, the Bishop will be a better piece against the Rook.

Two Rooks are slightly stronger than a Queen. They are slightly weaker than two Knights and a Bishop, and a little more so than two Bishops and a Knight. The power of the Knight decreases as the pieces are changed off. The power of the Rook, on the contrary, increases.

The King, a purely \emph{defensive} piece throughout the middle-game, becomes an \emph{offensive} piece once all the pieces are off the board, and sometimes even when there are one or two minor pieces left. The handling of the King becomes of paramount importance once the end-game stage is reached.
\clearpage

\section{General Strategy of the Opening}
The main thing is to \emph{develop the pieces quickly}. Get them into play as fast as you can.
From the outset two moves, e4  or d4, open up lines for the Queen and a Bishop. Therefore, theoretically one of these two moves must be the best, as no other first move accomplishes so much. 

\subsubsection*{Example 17}
Suppose we begin:
\newgame
\styleC
\mainline{1.e4 e5, 2.Nc3} This is both an attacking and a developing move. Black can now either reply with the identical move or play.
\mainline{2... Nc6} This developing move at the same time defends the King's Pawn.
\mainline{3.Nc3 Nf6} These moves are of a purely developing nature.
\mainline{4.Bb5} It is \emph{generally advisable not to bring this Bishop out until one Knight is out,} preferably the King's Knight. The Bishop could also have been played to c4, but it is advisable whenever possible to combine development and attack.
\mainline{4... Bb4} Black replies in the same manner, threatening a possible exchange of Bishop for Knight with \knight \texttimes \pawn to follow.
\mainline{5.O-O O-O} an indirect way of preventing \bmove{Bxc3}, which more experience or study will show to be bad. At the same time \emph{the Rook is brought into action in the centre, a very important point.} Black follows the same line of reasoning.
\mainline{6.d3 d6} These moves have a two-fold object, viz.: to protect the King's Pawn (e4) and to open the diagonal for the development of the Queen's Bishop.
\mainline{7.Bg5}

\begin{center}
\chessboard[normalboard,
moverstyle=triangle]
\end{center}
A very powerful move, which brings us to the middle-game stage, as there is already in view a combination to win quickly by \bmove{Kb5}. This threat makes it impossible for Black to continue the same course. (There is a long analysis showing that Black should lose if he also plays \bmove{Bg4}.) He is now forced to play \bmove{Bxc3}, as experience has shown, thus bringing up to notice three things.

First, the complete development of the opening has taken only seven moves. (This varies up to ten or twelve moves in some very exceptional cases. As a rule, eight should be enough.) Second, Black has been compelled to exchange a Bishop for a Knight, but as a compensation he has isolated White's \queen \ \rook \ \pawn and doubled a Pawn. (This, at such an early stage of the game, is rather an advantage for White, as the Pawn is doubled towards the centre of the board.) Third, White by the exchange brings up a Pawn to control the square d4, puts Black on the defensive, as experience will show, and thus keeps the initiative, an unquestionable advantage.\footnote{The value of the initiative is explained in section 'The Initiative'}

The strategical principles expounded above are the same for all the openings, only their tactical application varies according to the circumstances.

Before proceeding further I wish to lay stress on the following point which the student should bear in mind.

Before development has been completed no piece should be moved more than once, unless it is essential in order to obtain either material advantage or to secure freedom of action.

The beginner would do well to remember this, as well as what has already been stated: viz., bring out the Knights before bringing out the Bishops.

\section{Control Of the Centre}
The four squares, d4, d5, e4 and e5, are the centre squares, and control of these squares is called control of the centre. The control of the centre is of great importance. No violent attack can succeed without controlling at least two of these squares, and possibly three. Many a manoeuvre in the opening has for its sole object the control of the centre, which invariably ensures the initiative. It is well always to bear this in mind, since it will often be the reason of a series of moves which could not otherwise be properly understood. As this book progresses I shall dwell more fully on these different points. At present I shall devote some time to openings taken at random and explain the moves according to general principles. The student will in that way train his mind in the proper direction, and will thus have less trouble in finding a way out when confronted with a new and difficult situation.

\subsubsection*{Example 18}
\newgame
\styleC
\mainline{1.e4 e5, 2.Nf3 d6} A timid move. Black assumes a defensive attitude at once. On principle the move is wrong. In the openings, whenever possible, \emph{pieces should be moved in preference to Pawns.}
\mainline{3.d4}White takes the offensive immediately and strives to control the centre so as to have ample room to deploy his forces.
\mainline{3... Nd7} Black does not wish to relinquish the centre and also prefers the next move to \bmove{Kc6}, which would be the more natural square for the \knight . But on principle the move is wrong, because it blocks the action of the Queen's Bishop, and instead of facilitating the action of Black's pieces, tends, on the contrary, to cramp them.
\mainline{4.Bc4 h6} Black is forced to pay the penalty of his previous move. Such a move on Black's part condemns by itself any form of opening that makes it necessary. White threatened \wmove{Kg5} and Black could not stop it with \variation{4... Be7, 5.dxe5 Nxe5, 6.Nxe5 dxe5, 7.Qh5}, and White wins a Pawn and has besides a perfectly safe position. You can also consider the other variation.
\variation{4... Be7, 5.dxe6 dxe5, 6.Qd5} Which is also a dire situation for Black.
\mainline{5.Nc3 Ngf6, 6.Be3 Be7, 7.Qe2} It should be noticed that White does not Castle yet. The reason is that White wants to deploy forces first, and through the last move force Black to play \bmove{c5} to make room for the Queen as White threatens \wmove{rd1}, to be followed by \bmove{exd4}. Black's other alternatives would finally force him to play \bmove{cxe4}, thus abandoning the centre to White.
\mainline{7... c6, 8.Rd1 Qc7, 9.O-O} With this last move White completes development, while Black is evidently somewhat hampered. A simple examination will suffice to show that White's position is unassailable. There are no weak spots in his armour, and his pieces are ready for any manoeuvre that he may wish to carry out in order to begin the attack on the enemy's position. The student should carefully study this example. It will show them that it is sometimes convenient to delay Castling. I have given the moves as they come to my mind without following any standard book on openings. Whether the moves given by me agree or not with the standard works, I do not know, but at the present stage of this book it is not convenient to enter into discussions of mere technicalities which the student will be able to understand when he has become more proficient.

\begin{center}
\chessboard[normalboard,
moverstyle=triangle]
\end{center}
\clearpage

\subsubsection*{Example 19}
\newgame
\styleC
\mainline{1.e4 e5, 2.Nf3 d6, 3.d4 Bg4} A bad move, which violates one of the principles set down, according to which at least one Knight should be developed before the Bishops are brought out, and also because it exchanges a Bishop for a Knight, which in the opening is generally bad, unless there is some compensation.
\mainline{4.dxe5 Bxf3} \variation{4... dxe5} looses a Pawn.
\mainline{5.Qxf3 dxe5, 6.Bc4 Qf6} if \variation{6... Nf6, 7.Qb3} wins a pawn.
\mainline{7.Qb3 b6, 8.Nc3 c6} To prevent \wmove{Kd5}
\begin{center}
\chessboard[normalboard,
moverstyle=triangle]
\end{center}
Black, however, has no pieces out except his Queen, and White, with a Bishop and a Knight already developed, has a chance of obtaining an advantage quickly by playing \wmove{Nd5} anyway. The student is left to work out the many variations arising from this position.

These examples will show the practical application of the principles previously enunciated. The student is warned against playing Pawns in preference to pieces at the beginning of the game, especially h4 and a4, which are moves very commonly indulged in by beginners.

\section{Traps}
I shall now give a few positions or traps to be avoided in the openings, and in which (practice has shown) beginners are often caught.
\subsubsection*{Example 20}
\newgame
\styleC
\fenboard{r2qkbnr/ppp2pp1/2np3p/4p3/2BPP1b1/2N2N2/PPP2PPP/R1BQK2R w KQkq - 0 1}
\chessboard[smallboard,
marginleft=false,
marginrightwidth=2em,
moverstyle=triangle]
\begin{wraptable}{r}{0.5\textwidth}
	\vspace{-14em}
	\mainline{1.dxe5 Nxe5} Black should have recaptured with the Pawn.
	\mainline{2.Nxe5 Bxd1, 3.Bxf7+ Ke7, 4.Nd5\#}
\end{wraptable}
\begin{center}
\chessboard[normalboard,
moverstyle=triangle]
\end{center}

\subsubsection*{Example 21}
\newgame
\styleC
\fenboard{rn1qkbnr/pppp1pp1/7p/8/2B3b1/2N2N2/PPPP1PPP/R1BQK2R b KQkq - 0 1}
\chessboard[smallboard,
marginleft=false,
marginrightwidth=2em,
moverstyle=triangle]
\begin{wraptable}{r}{0.5\textwidth}
	\vspace{-14em}
	Black, having the move, should play \bmove{e6}. But suppose Black plays \bmove{Nc6} instead, then comes—
	\mainline{1... Nc6 2.Bxf7+} \bmove{Ne5} would also give White the advantage, the threat being of course if 
\end{wraptable}
\variation{2.Ne5 Bxd1, 3.Bxf7\#}. Nor does \bmove{Bh5} help matters, because of \wmove{Qxb5}, \bmove{Be6} leaves Black with the inferior position. But White's move in the text secures an immediate material advantage, and the beginner at any rate should never miss such an opportunity for the sake of a speculative advantage in position.
\mainline{2... Kxf7, 3.Ne5+ Kf6, 4.Nxg4+} and White has won a Pawn besides having the better position.
There are a good many other traps—in fact, there is a book written on traps on the chess board; but the type given above is the most common of all.
\begin{center}
\chessboard[normalboard,
moverstyle=triangle]
\end{center}

\chapter{Further Principles in End-Game Play}

We shall now go back to the endings in search of a few more principles, then again to the middle-game, and finally to the openings once more, so that the advance may not only be gradual but homogeneous. In this way the foundation on which we expect to build the structure will be firm and solid.

\section{Cardinal Principle}

\newgame
\styleA
\fenboard{8/p5p1/7p/6k1/8/7K/PP5P/8 w - - 0 1}
\chessboard[smallboard,
marginleft=false,
marginrightwidth=2em,
moverstyle=triangle]
\begin{wraptable}{r}{0.5\textwidth}
	\vspace{-14em}
In the position shown above, White can draw by playing \wmove{b4} according to the general rule that governs such cases, i.e. to advance the Pawn that is free from opposition. But suppose that White, either because he does not know this principle or because he does not, in this case, sufficiently appreciate the value of its application;
\end{wraptable}

suppose, we say, that he plays \wmove{a4}. Then Black can win by playing \bmove{a5}, applying one of the cardinal principles of the high strategy of chess—

\emph{A unit that holds two.}

In this case one Pawn would hold two of the opponent's Pawns. The student cannot lay too much stress on this principle. It can be applied in many ways, and it constitutes one of the principal weapons in the hands of a master.

\subsubsection*{Example 22}
The example given should be sufficient proof. We give a few moves of the main variation:—

\styleC

\mainline{1.a4 a5, 2.Kg2 Kf4} Best; see why.
\mainline{3.b4 axb4} Best
\mainline{4.a5 b3, 5.a6 b2, 6.a7 b1=Q, 7.a8=Q Qe4+, 8. Qxe4 Kxe4}

This brings the game to a position (shown above as the final position of Example 22) which is won by Black, and which constitutes one of the classical endings of King and Pawns. I shall try to explain the guiding idea of it to those not familiar with it.

\begin{center}
\chessboard[largeboard,
moverstyle=triangle]
\end{center}

\section{A Classical Ending}

\subsubsection*{Example 23}
In this position White's best line of defence consists in keeping his Pawn where it stands at h2. As soon as the Pawn is advanced it becomes easier for Black to win. On the other hand, Black's plan to win (supposing that White does not advance his Pawn) may be divided into three parts. The first part will be to get his King to h6 at the same time keeping intact the position of his Pawns. (This is all important, since, in order to win the game, it is essential at the end that Black may be able to advance his rearmost Pawn one or two squares according to the position of the White King.)

\mainline{9.Kg3 Ke3, 10.Kg2}

\variation{10.Kg4 Kf2, 11.h4 g6} will win.

\mainline{10... Kf4, 11.Kf2 Kg5, 12.Kg2 Kh4, 13.Kg1 Kh3} The first part has been completed.

The second part will be short and will consist in advancing the g and h \pawn up the g file.

\mainline{14.Kh1 h5 15.Kg1 h4} This ends the second part.

\begin{center}
\chessboard[normalboard,
moverstyle=triangle]
\end{center}

\chessboard[smallboard,
marginleft=false,
marginrightwidth=2em,
moverstyle=triangle]
\begin{wraptable}{r}{0.5\textwidth}
	\vspace{-13em}
The third part will consist in timing the advance of the g Pawn so as to play \bmove{g3} when the White King is at \wmove{h1}. It now becomes evident how necessary it is to be able to move the g Pawn either one or two squares according to the position of the White King, as indicated previously.
\end{wraptable}

In this case, as it is White's move, the Pawn will be advanced two squares since the White King will be in the corner, but if it were now Black's move the g Pawn should only be advanced one square since the White King is at \wmove{g1}.
 
\mainline{16.Kh1 g5, 17.Kg1 g4, 18.Kh1 g3, 19.hxg3} If \wmove{Kg1, g2}.

\mainline{19... hxg3, 20.Kg1 g2, 21.Kf2 Kh2} and wins.

\begin{center}
\chessboard[normalboard,
moverstyle=triangle]
\end{center}

It is in this analytical way that the student should try to learn. He will thus train his mind to follow a logical sequence in reasoning out any position. This example is excellent training, since it is easy to divide it into three stages and to explain the main point of each part.

The next subject we shall study is the simple opposition, but before we devote our time to it I wish to call attention to two things.

\section{Obtaining A Passed Pawn}

When three or more Pawns are opposed to each other in some such position as the one in Example 24, there is always a chance for one side or the other of obtaining a passed Pawn.

\subsubsection*{Example 24}
In the following position the way of obtaining a passed Pawn is to advance the centre Pawn.

\newgame
\fenboard{8/ppp4k/8/PPP5/8/8/8/7K w - - 0 1}
\chessboard[smallboard,
marginleft=false,
marginrightwidth=2em,
moverstyle=triangle]
\begin{wraptable}{r}{0.5\textwidth}
	\vspace{-13em}
\mainline{1.b6 axb6} if \bmove{cxb6, a6}
\mainline{2.c6 bxc6, 3.a6}

\end{wraptable}

and as in this case the White Pawn is nearer to Queen than any of the Black Pawns, White will win. Now if it had been Black's move Black could play.

\newgame
\fenboard{8/ppp4k/8/PPP5/8/8/8/7K b - - 0 1}
\mainline{1... b6, 2.cxb6 cxb6} It would not be advisable to try to obtain a passed Pawn because the White Pawns would be nearer to Queen than the single Black Pawn.

\mainline{3.axb6 axb6} and the game properly played would be a draw. The student should work this out for himself.

\section{How To Find Out Which Pawn Will Be First To Queen}
When two Pawns are free, or will be free, to advance to Queen, you can find out, by counting, which Pawn will be the first to succeed.

\subsubsection*{Example 25}
In this position whoever moves first wins.

\newgame
\fenboard{8/p5kp/8/1P6/8/1K6/P7/8 w - - 0 1}
\chessboard[smallboard,
marginleft=false,
marginrightwidth=2em,
moverstyle=triangle]
\begin{wraptable}{r}{0.5\textwidth}
	\vspace{-13em}
The first thing is to find out, by counting, whether the opposing King can be in time to stop the passed Pawn from Queening. When, as in this case, it cannot be done, the point is to count which Pawn comes in first. In this case the time is the same, 
\end{wraptable}

but the Pawn that reaches the eighth square first and becomes a Queen is in a position to capture the adversary's Queen when he makes one. Thus:

\mainline{1.a4 h5, 2.a5 h4, 3.b6 axb6}

Now comes a little calculation. White can capture the Pawn, but if so, will not, when Queening, command the square where Black will also Queen their Pawn. Therefore, instead of taking, plays:

\mainline{4.a6 b5, 5.a7 b4, 6.a8=Q} and wins.

The student would do well to acquaint themselves with various simple endings of this sort, so as to acquire the habit of counting, and thus be able to know with ease when one can or cannot get there first. Once again I must call attention to the fact that a book cannot by itself teach how to play. It can only serve as a guide, and the rest must be learned by experience, and if a teacher can be had at the same time, so much the faster will the student be able to learn.

\section{The Opposition}

When Kings have to be moved, and one player can, by force, bring their King into a position similar to the one shown in the following diagram, so that his adversary is forced to move and make way, the player obtaining that advantage is said to have \emph{the opposition}.

\subsubsection*{Example 26}
Suppose in the above White plays

\newgame
\fenboard{8/8/4k3/1p5p/1P2K2P/8/8/8 w - - 0 1}
\chessboard[smallboard,
marginleft=false,
marginrightwidth=2em,
moverstyle=triangle]
\begin{wraptable}{r}{0.5\textwidth}
	\vspace{-15em}
\mainline{1.Kd4}

Now Black has the option of either opposing the passage of the White King by playing \bmove{Kd6} or, if preferred, reply by moving the King thus \bmove{Kd6}.
\end{wraptable}

 Notice that the Kings are directly opposed to each other, and the number of intervening squares between them is odd—one in this case.

The opposition can take the form shown above, which can be called actual or close frontal opposition; or this form:

\begin{center}
\newgame
\fenboard{8/8/2k5/8/4K3/8/8/8 w - - 0 1}
\chessboard[tinyboard,
moverstyle=triangle]

which can be called actual or close diagonal opposition, or, again, this form:

\newgame
\fenboard{8/8/8/8/2k1K3/8/8/8 w - - 0 1}
\chessboard[tinyboard,
moverstyle=triangle]
\end{center}

which can be called actual or close lateral opposition.

In practice they are all one and the same. The Kings are always on squares of the same colour, there is only one intervening square between the Kings, and the player who has moved last \emph{"has the opposition."}

Now, if the student will take the trouble of moving each King backwards as in a game in the same frontal, diagonal or lateral line respectively shown in the diagrams, we shall have what may be called distant frontal, diagonal and lateral opposition respectively.

The matter of the opposition is highly important, and takes at times somewhat complicated forms, all of which can be solved mathematically; but, for the present, the student should only consider the most simple forms. (An examination of some of the examples of King and Pawns endings already given will show several cases of close opposition.)

In all simple forms of opposition,

\emph{when the Kings are on the same line and the number of intervening squares between them is even, the player who has the move has the opposition.}

\subsubsection*{Example 27}

\newgame
\fenboard{4k3/8/8/1p5p/1P5P/8/8/4K3 w - - 0 1}
\chessboard[smallboard,
marginleft=false,
marginrightwidth=2em,
moverstyle=triangle]
\begin{wraptable}{r}{0.5\textwidth}
	\vspace{-13em}
This position shows to advantage the enormous value of the opposition. The position is very simple. Very little is left on the board, and the position, to a beginner, probably looks absolutely even. It is not the case, however. Whoever has the move wins. Notice that the Kings are directly in front of one another, and that the number of intervening squares is even.
\end{wraptable}

Now as to the procedure to win such a position. The proper way to begin is to move straight up. Thus:

\mainline{1.Ke2 Ke7, 2.Ke3 Ke6, 3.Ke4 Kf6}

Now White can exercise the option of either playing \wmove{Kd5} and thus passing with his King, or of playing \wmove{Kf4} and prevent the Black King from passing, thereby keeping the opposition. Mere counting will show that the former course will only lead to a draw, therefore White takes the latter course and plays:

\mainline{4.Kf4 Kg6}

If \variation{4... Ke6, 5.Kg5} will win.

\mainline{5.Ke5 Kg7}

Now by counting it will be seen that White wins by capturing Black's Knight Pawn. The process has been comparatively simple in the variation given above, but Black has other lines of defence more difficult to overcome. Let us begin anew.

\newgame
\fenboard{4k3/8/8/1p5p/1P5P/8/8/4K3 w - - 0 1}

\mainline{1.Ke2 Kd8}

Now if \wmove{2.Kd3 Kd7} or if \wmove{2.Ke3 Ke7}, and Black obtains the opposition in both cases. (When the Kings are directly in front of one another, and the number of intervening squares between the Kings is odd, the player who has moved last has the opposition.)

Now in order to win, the White King must advance. There is only one other square where he can go, \wmove{Kf3}, and that is the right place. Therefore it is seen that in such cases when the opponent makes a so-called waiting move, you must advance, leaving a rank or file free between the Kings. Therefore we have—

\mainline{2.Kf3 Ke7}

Now, it would be bad to advance, because then Black, by bringing up the King in front of your King, would obtain the opposition. It is White's turn to play a similar move to Black's first move, viz.:

\mainline{3.Ke3}

Which brings the position back to the first variation shown. The student would do well to familiarise themselves with the handling of the King in all examples of opposition. It often means the winning or losing of a game.

\subsubsection*{Example 28}

The following position is an excellent proof of the value of the opposition as a means of defence.

\begin{center}
\newgame
\fenboard{8/8/8/4p1p1/8/5P2/6K1/3k4 w - - 0 1}
\chessboard[smallboard,
marginleft=false,
marginrightwidth=2em,
moverstyle=triangle]
\end{center}

White is a Pawn behind and apparently lost, yet he can manage to draw as follows:

\mainline{1.Kh1!}

The position of the Pawns does not permit White to draw by means of the actual or close opposition, hence he takes the distant opposition: In effect if \wmove{1.Kf1} (actual or close opposition), \variation{1.Kh1! Kd2, 2.Kf2 Kf3} and White cannot continue to keep the lateral opposition essential to player safety, because of their own Pawn at f6. On the other hand, after the next move, if

\mainline{1... Kd2, 2.Kh2 Kd3, 3.Kh3! Ke2, 4.Kg2 Ke3, 5.Kg3 Kd4, 6.Kg4}

\chessboard[smallboard,
marginleft=false,
marginrightwidth=2em,
moverstyle=triangle]
\begin{wraptable}{r}{0.5\textwidth}
	\vspace{-13em}

Attacking the Pawn and forcing Black to play \bmove{Ke3} when it is possible to go back to \wmove{Kg3} as already shown, and always keep the opposition.

Going back to the original position, if

\end{wraptable}

\newgame
\fenboard{8/8/8/4p1p1/8/5P2/6K1/3k4 w - - 0 1}

\mainline{1.Kh1! g4}

White does not play \wmove{fxg4}, because \bmove{e4} will win, but plays:

\mainline{2.Kg2 Kd2}

if \variation{2... gxf3+, 3.Kxf3 e4} will force a draw

\mainline{3.fxg4 e4}


and mere counting will show that both sides Queen, drawing the game.

If the student will now take the trouble to go back to the examples of King and Pawns which I have given in this book, they will realise that in all of them the matter of the opposition is of paramount importance; as, in fact, it is in nearly all endings of King and Pawns, except in such cases where the Pawn-position in itself ensures the win.

\begin{center}
\chessboard[largeboard,
marginleft=false,
marginrightwidth=2em,
moverstyle=triangle]
\end{center}

\clearpage

\section{The Relative Value of Knight and Bishop}

Before turning our attention to this matter it is well to state now that \emph{two Knights alone cannot mate}, but, under certain conditions of course, they can do so if the opponent has one or more Pawns.

\subsubsection*{Example 29}

\newgame
\fenboard{k7/8/1K1NN3/8/8/8/8/8 w - - 0 1}
\chessboard[smallboard,
marginleft=false,
marginrightwidth=2em,
moverstyle=triangle]
\begin{wraptable}{r}{0.5\textwidth}
	\vspace{-13em}

In this position White cannot win, although the Black King is cornered, but in the following position, in which Black has a Pawn,White wins with or without the move. Thus:

\end{wraptable}


\newgame
\fenboard{k7/8/1K1N4/7p/7N/8/8/8 w - - 0 1}
\chessboard[smallboard,
marginleft=false,
marginrightwidth=2em,
moverstyle=triangle]
\begin{wraptable}{r}{0.5\textwidth}
	\vspace{-13em}

In the position where Black has a Pawn, White wins with or without the move. Thus:

\mainline{1.Nf3 h4}

White cannot take the Pawn because the game will be drawn, as explained before.

\end{wraptable}

\mainline{2.Ne5 h3, 3.Nc6 h2, 4.Nb5 h1=Q, 5.Nc7#}

The reason for this peculiarity in chess is evident.

\emph{White with the two Knights can only stalemate the King, unless Black has a Pawn which can be moved.}

\begin{center}
\chessboard[largeboard,
moverstyle=triangle]
\end{center}

\clearpage


\subsubsection*{Example 30}

Although he is a Bishop and a Pawn ahead the following position cannot be won by White.


\newgame
\fenboard{7k/8/8/6K1/8/3B4/7P/8 w - - 0 1}
\chessboard[smallboard,
marginleft=false,
marginrightwidth=2em,
moverstyle=triangle]
\begin{wraptable}{r}{0.5\textwidth}
	\vspace{-13em}

It is the greatest weakness of the Bishop, that when the H Pawn Queens on a square of opposite colour and the opposing King is in front of the Pawn, the Bishop is absolutely worthless. All that Black has to do is to keep moving his King close to the corner square. 

\end{wraptable}

\subsubsection*{Example 31}

\newgame
\fenboard{8/8/5Np1/8/8/7p/5K2/7k b - - 0 1}
\chessboard[smallboard,
marginleft=false,
marginrightwidth=2em,
moverstyle=triangle]
\begin{wraptable}{r}{0.5\textwidth}
	\vspace{-13em}

In the above position White with or without the move can win. Take the most difficult variation.

\end{wraptable}

\mainline{1... Kh2, 2.Ng4+ Kh1, 3.Kf1 g5, 4.Kf2 h2, 5.Ne3 g4, 6.Nf1 g3+, 7.Nxg3#}

\clearpage

\chessboard[smallboard,
marginleft=false,
marginrightwidth=2em,
moverstyle=triangle]
\begin{wraptable}{r}{0.5\textwidth}
	\vspace{-13em}

Now that we have seen these exceptional cases, we can analyse the different merits and the relative value of the Knight and the Bishop.

It is generally thought by amateurs that the Knight is the more valuable piece of the two, the chief reason being that, unlike the Bishop, 

\end{wraptable}

the Knight can command both Black and White squares. However, the fact is generally overlooked that the Knight, at any one time, has the choice of one colour only. It takes much longer to bring a Knight from one wing to the other. Also, as shown in the following Example, a Bishop can stalemate a Knight; a compliment which the Knight is unable to return.

\subsubsection*{Example 32}

\newgame
\fenboard{8/8/8/N2b4/8/8/8/8 w - - 0 1}
\chessboard[smallboard,
marginleft=false,
marginrightwidth=2em,
moverstyle=triangle]
\begin{wraptable}{r}{0.5\textwidth}
	\vspace{-13em}

The weaker the player the more terrible the Knight is to him, but as a player increases in strength the value of the Bishop becomes more evident to him, and of course there is, or should be, a corresponding decrease in his estimation of the value of the Knight as compared to the Bishop.

\end{wraptable}

In this respect, as in many others, the masters of to-day are far ahead of the masters of former generations. While not so long ago some of the very best amongst them, like Pillsbury and Tchigorin, preferred Knights to Bishops, there is hardly a master of to-day who would not completely agree with the statements made above.

\clearpage

\subsubsection*{Example 33}

This is about the only case when the Knight is more valuable than the Bishop.

\newgame
\fenboard{4k3/3n4/2p5/1pPp4/pP1P4/P7/8/2B1K3 w - - 0 1}
\chessboard[smallboard,
marginleft=false,
marginrightwidth=2em,
moverstyle=triangle]
\begin{wraptable}{r}{0.5\textwidth}
	\vspace{-13em}

It is what is called a \emph{"block position,"} and all the Pawns are on one side of the board. (If there were Pawns on both sides of the board there would be no advantage in having a Knight.) In such a position Black has excellent chances of winning.

\end{wraptable}

Of course, there is an extra source of weakness for White in having their Pawns on the same colour-squares as his Bishop. This is a mistake often made by players. The proper way, generally, in an ending, is to have your Pawns on squares of opposite colour to that of your own Bishop. When you have your Pawns on squares of the same colour the action of your own Bishop is limited by them, and consequently the value of the Bishop is diminished, since the value of a piece can often be measured by the number of squares it commands. While on this subject, I shall also call attention to the fact that it is generally preferable to keep your Pawns on squares of the same colour as that of the opposing Bishop, particularly if they are passed Pawns supported by the King. The principles might be stated thus:

\emph{When the opponent has a Bishop, keep your Pawns on squares of the same colour as your opponent's Bishop.}

\emph{Whenever you have a Bishop, whether the opponent has also one or not, keep your Pawns on squares of the opposite colour to that of your own Bishop.}

Naturally, these principles have sometimes to be modified to suit the exigencies of the position.

\clearpage

\subsubsection*{Example 34}

\newgame
\fenboard{4k3/3n1ppp/8/8/8/8/5PPP/2B1K3 w - - 0 1}
\chessboard[smallboard,
marginleft=false,
marginrightwidth=2em,
moverstyle=triangle]
\begin{wraptable}{r}{0.5\textwidth}
	\vspace{-13em}

In this position the Pawns are on one side of the board, and there is no advantage in having either a Knight or a Bishop. The game should surely end in a draw.

\end{wraptable}

\subsubsection*{Example 35}

\newgame
\fenboard{4k3/pppn1ppp/8/8/8/8/PPP2PPP/2B1K3 w - - 0 1}
\chessboard[smallboard,
marginleft=false,
marginrightwidth=2em,
moverstyle=triangle]
\begin{wraptable}{r}{0.5\textwidth}
	\vspace{-13em}

Now let us add three Pawns on each side to the above position, so that there are Pawns on both sides of the board.

\end{wraptable}

\newgame
\fenboard{4k3/pppn1ppp/8/8/8/8/PPP2PPP/2B1K3 w - - 0 1}
\chessboard[smallboard,
marginleft=false,
marginrightwidth=2em,
moverstyle=triangle]
\begin{wraptable}{r}{0.5\textwidth}
	\vspace{-13em}

It is now preferable to have the Bishop, though the position, if properly played out, should end in a draw. The advantage of having the Bishop lies as much in its ability to command, at long range, both sides of the board from a central position as in its ability to move quickly from one side of the board to the other.

\end{wraptable}

\clearpage

\subsubsection*{Example 36}

\newgame
\fenboard{4k3/pppn1pp1/8/8/8/8/1PP2PPP/2B1K3 w - - 0 1}
\chessboard[smallboard,
marginleft=false,
marginrightwidth=2em,
moverstyle=triangle]
\begin{wraptable}{r}{0.5\textwidth}
	\vspace{-13em}

In this position it is unquestionably an advantage to have the Bishop, because, although each player has the same number of Pawns, they are not balanced on each side of the board. Thus, on the King's side, White has three to two, while on the Queen's side it is Black that has three to two. Still, with proper play, the game should end in a draw, though White has somewhat better chances.

\end{wraptable}

\subsubsection*{Example 37}

\newgame
\fenboard{4k3/pppn1p2/8/8/8/8/2P2PPP/2B1K3 w - - 0 1}
\chessboard[smallboard,
marginleft=false,
marginrightwidth=2em,
moverstyle=triangle]
\begin{wraptable}{r}{0.5\textwidth}
	\vspace{-13em}

Here is a position in which to have the Bishop is a decided advantage, since not only are there Pawns on both sides of the board, but there is a passed Pawn (g  \& h Pawn for White, a \& b Pawn for Black). Black should have extreme difficulty in drawing this position, if it can do it at all.\footnotemark
\end{wraptable}

\footnotetext{Computer analysis gives a +0.1 advantage to White in this position, the situation does not improve much with it being Black's turn.}

\subsubsection*{Example 38}

\newgame
\fenboard{4k3/p1pn1p2/8/8/8/8/2P2P1P/2B1K3 w - - 0 1}
\chessboard[smallboard,
marginleft=false,
marginrightwidth=2em,
moverstyle=triangle]
\begin{wraptable}{r}{0.5\textwidth}
	\vspace{-13em}

Again Black would have great difficulty in drawing this position.

\end{wraptable}

The student should carefully consider these positions. I hope that the many examples will help them to understand, in their true value, the relative merits of the Knight and Bishop. As to the general method of procedure, a teacher, or practical experience, will be best. I might say generally, however, that the proper course in these endings, as in all similar endings, is: Advance of the King to the centre of the board or towards the passed Pawns, or Pawns that are susceptible of being attacked, and rapid advance of the passed Pawn or Pawns as far as is consistent with their safety.

To give a fixed line of play would be folly. Each ending is different, and requires different handling, according to what the adversary proposes to do. Calculation by visualising the future positions is what will count. 

\clearpage

\section{How to Mate with a Knight and a Bishop}

Now, before going back again to the middle-game and the openings, let us see how to mate with Knight and Bishop, and, then, how to win with a Queen against a Rook.

With a Knight and a Bishop \emph{the mate can only be given in the corners of the same colour as the Bishop.}

\subsubsection*{Example 39}

\newgame
\fenboard{4k3/8/8/8/8/8/8/2B1K1N1 w - - 0 1}
\chessboard[smallboard,
marginleft=false,
marginrightwidth=2em,
moverstyle=triangle]
\begin{wraptable}{r}{0.5\textwidth}
	\vspace{-13em}
	
In this example we must mate either at a8 or h8. The ending can be divided into two parts. Part one consists in driving the Black King to the last line. We might begin, as is generally done in all such cases, by advancing the King to the centre of the board:

\end{wraptable}

\mainline{1.Ke2 Ke7} Black, in order to make it more difficult, goes towards the white-squared corner:

\mainline{2.Kd3 Kc6, 3.Bf4 Kd5, 4.Ne2 Kc5, 5.Nc3 Kb4, 6.Kd4 Ka5, 7.Kc5 Ka6, 8.Kc6 Ka7, 9.Nd5 Ka8}The first part is now over; the Black King is in the white-squared corner.

\chessboard[smallboard,
marginleft=false,
marginrightwidth=2em,
moverstyle=triangle]
\begin{wraptable}{r}{0.5\textwidth}
	\vspace{-13em}

The second and last part will consist in driving the Black King now from a8 to a1 or h8 in order to mate him. a1 will be the quickest in this position.

\mainline{10.Nb6+ Ka7 11.Bc7 Ka6, 12.Bb8 Ka5, 13.Nd5 Ka4, 14.Kc5!}

\end{wraptable}

Black tries to make for a8 with his King. White has two ways to prevent that, one by \variation{14.Be5 Kb3, 15.Ne3} and the other which I give as the text, and which I consider better for the student to learn, because it is more methodical and more in accord with the spirit of all these endings, by using the King as much as possible.

\mainline{14... Kb3, 15.Nb4 Kc3, 16.Bf4 Kb3 17.Be5 Ka4, 18.Kc4 Ka5, 19.Bc7+ Ka4, 20.Nd3 Ka3, 21.Bb6 Ka4, 22.Nb2+ Ka3, 23.Kc3 Ka2, 24.Kc2 Ka3, 25.Bc5+ Ka2, 26.Nd3 Ka1, 27.Bb4 Ka2, 28.Nc1+ Ka1, 29.Bc3#}

It will be seen that the ending is rather laborious. There are two outstanding features: the close following by the King, and the controlling of the squares of opposite colour to the Bishop by the combined action of the Knight and King. The student would do well to exercise themselves methodically in this ending, as it gives a very good idea of the actual power of the pieces, and it requires foresight in order to accomplish the mate within the fifty moves which are granted by the rules.

\clearpage

\section{Queen Against Rook}

This is one of the most difficult endings without Pawns. The resources of the defence are many, and when used skilfully only a very good player will prevail within the limit of fifty moves allowed by the rules. (The rule is that at any moment you may demand that your opponent mate you within fifty moves. However, every time a piece is exchanged or a Pawn advanced the counting must begin afresh.)

\subsubsection*{Example 40}

\newgame
\fenboard{1k6/1r6/2K5/Q7/8/8/8/8 w - - 0 1}
\chessboard[smallboard,
marginleft=false,
marginrightwidth=2em,
moverstyle=triangle]
\begin{wraptable}{r}{0.5\textwidth}
	\vspace{-13em}
	
	his is one of the standard positions which Black can often bring about. Now, it is White's move. If it were Black's move it would be simple, as they would have to move their Rook away from the King (find out why), and then the Rook would be comparatively easy to win. 

\end{wraptable}

We deduce from the above that the main object is to force the Black Rook away from the defending King, and that, in order to compel Black to do so, we must bring about the position in the diagram with Black to move. Once we know what is required, the way to proceed becomes easier to find. Thus:

\mainline{1.Qe5+} Not \variation{1.Qa6 Rc7+, 2.Kb6 Rc3+, 3.Kxc3} Stalemate.(The beginner will invariably fall into this trap.)

\mainline{1... Ka7, 2.Qa1 Kb8, 3.Qa5} In a few moves we have accomplished our object. The first part is concluded. Now we come to the second part. The Rook can only go to a White square, otherwise the first check with the Queen will win it. Therefore:

\mainline{3... Rb3, 4.Qe5+ Ka8!, 5.Qh8+ Ka7, 6.Qg7+ Ka8, 7.Qg8+ Rb8, 8.Qa2#} (The student should find out by themself how to win when \bmove{3...Rb1 4.Qe5+ Ka7}) 

\subsubsection*{Example 41}

\newgame
\fenboard{8/k7/2K5/4Q3/8/8/8/1r6 w - - 0 1}
\chessboard[smallboard,
marginleft=false,
marginrightwidth=2em,
moverstyle=triangle]
\begin{wraptable}{r}{0.5\textwidth}
	\vspace{-13em}
	
The procedure here is very similar. The things to bear in mind are that the Rook must be prevented from interposing at b1 because of an immediate mate, and in the same way the King must be prevented from going either to a3 or c1.

\end{wraptable}

\subsubsection*{Example 42}
We shall now examine a more difficult position.

\newgame
\fenboard{8/5rk1/8/5Q1K/8/8/8/8 w - - 0 1}
\chessboard[smallboard,
marginleft=false,
marginrightwidth=2em,
moverstyle=triangle]
\begin{wraptable}{r}{0.5\textwidth}
	\vspace{-13em}

Many players would be deceived by this position. The most likely looking move is not the best. Thus suppose we begin:

\mainline{1.Qe5+ Kf8, 2.Kg6 Rd7}

\end{wraptable}

The only defence, but, unfortunately, a very effective one, which makes it very difficult for White, since he cannot play \wmove{3.Qe6} because of \wmove{3.Qe6 Kg7+, 4.Kf6 Rf7+} draws. 
Nor can he win quickly by \wmove{Qc5+} because \wmove{3.Qc5+ Ke8, 4.Kf6 Rd6+} driving back the White King.

Now that we have seen the difficulties of the situation let us go back. The best move is:

\newgame
\fenboard{8/5rk1/8/5Q1K/8/8/8/8 w - - 0 1}

\mainline{1.Qg5+ Kh8} If \variation{1... Kh7, 2.Qg6+ Kh8, 3.Kh6!}

\mainline{2.Qe5+ Kh7!, 3.kg5 Ra7!} If \variation{3... Kg7+, 4.Kf6} leads to a position similar to those in Examples 40 and 41.

\mainline{4.Qe4+ Kg8, 5.Qc4+ Kh7, 6.Kf6 Rg7, 7.Qh4+ Kg8, 8.Qh5} and we have the position of Example 40 with Black to move.

Let us go back again

\newgame
\fenboard{8/5rk1/8/5Q1K/8/8/8/8 w - - 0 1}

\mainline{1.Qg5+ Kg8, 2.Qd8+ Kg7, 3.Kg5 Rf3} The best place for the Rook away from the King \variation{3... Kh7, 4.Qd4 Rg7+, 5.Kf6} would lead to positions similar to those already seen.

\mainline{4.Qd4+ Kf8, 5.Kg6} \variation{5.Qd6+ Kg7, 6.Qe5+ Kf8, 7.Kg6} would also win the Rook. The next move, however, is given to show the finesse of such endings. White now threatens mate at d8.

\mainline{5... Rg3+, 6.Kf6 Rf3+, 7.Ke6 Rh3} White threatened mate at h8.

\mainline{8.Qf4+} and the rook is lost.

Note, in these examples, that the checks at long range along the diagonals have often been the key to all the winning manoeuvres. Also that the Queen and King are often kept on different lines. The student should carefully go over these positions and consider all the possibilities not given in the text.

The student should once more go through everything already written before proceeding further with the book.

\chapter{Planning a Win in Middle-Game Play}

I shall now give a few winning positions taken from my own games. I have selected those that I believe can be considered as types, i.e. positions that may easily occur again in a somewhat similar form. A knowledge of such positions is of great help; in fact, one cannot know too many. It often may help the player to find, with little effort, the right move, which they might not be able to find at all without such knowledge.

\section{Attacking Without the aid of Knights}

\subsubsection{Example 43}

\newgame
\fenboard{2kr4/ppp1qp2/4b3/8/3P2r1/N1P5/PP4P1/R1B2RK1 b - - 0 1}
\chessboard[smallboard,
marginleft=false,
marginrightwidth=2em,
moverstyle=triangle]
\begin{wraptable}{r}{0.5\textwidth}
	\vspace{-13em}

It is Black's move, Black is a \knight and a \pawn behind he must win quickly, if at all. plays:

\mainline{1... Rdg8!, 2.Rf2}

If \variation{2.Qxe7 Rxg2, 3.Kh1 Bd5} and mate follows in a few moves.

\end{wraptable}

\mainline{2... Rxg2+, 3.Kf1 Bc4+, 4.Nxc4 Rg1#}

\subsubsection*{Example 44}

\newgame
\fenboard{kr2r3/pRp3pp/Q1P5/5R2/Pp1q4/3pp2P/6PK/8 w - - 0 1}
\chessboard[smallboard,
marginleft=false,
marginrightwidth=2em,
moverstyle=triangle]
\begin{wraptable}{r}{0.5\textwidth}
	\vspace{-13em}
	
Black's last move was \bmove{e3}, played with the object of stopping what was thought to be White's threat, viz.:\wmove{Ra5}, to which he would have answered \bmove{Qf4+} and drawn the game by perpetual check. White, however, has a more forceful move, and mates in three moves as follows: 

\end{wraptable}

\mainline{1.Rxa7+ Qxa7, 2.Ra5 Rb7, 3.Qxb7#} Where no matter how Black responds to \wmove{2.Ra5} White can always respond with checkmate.


\subsubsection*{Example 45}

\newgame
\fenboard{3q1kn1/r5p1/1ppr1pQn/p2ppP2/P3P2R/1BPPB3/1P3PK1/7R w - - 0 1}
\chessboard[smallboard,
marginleft=false,
marginrightwidth=2em,
moverstyle=triangle]
\begin{wraptable}{r}{0.5\textwidth}
	\vspace{-13em}
	
White has a beautiful position, but still had better gain some material, before Black consolidates their defensive position. plays:

\mainline{1.Rxh6! gxh6, 2.Bxh6+ Ke7} If \variation{2... Nxh6 3.Rxh6} and Black would be helpless.\footnotemark

\end{wraptable}

\footnotetext{Engine Analysis suggests that \bmove{Nxh6} is best with the variation \variation{2... Nxh6 3.Rxh6 Rf7, 4.Rh8+ Ke7, 5.Rxd8 Rxd8}}

\mainline{3.Qh7+ Ke8, 4.Qxg8+ Kd7, 5.Qh7+ Qe7, 6.Bf8 Qxh7, 7.Rxh7+ Ke8, 8.Rxa7}

In these few examples the attacking has been done by Rooks and Bishops in combination with the Queen. There have been no Knights to take part in the attack. We shall now give some examples in which the Knights play a prominent part as an attacking force.


\section{Attacking with Knights as a Prominent Force}

\subsubsection*{Example 46}

\newgame
\fenboard{4r2r/p1pb1ppk/2ppn2p/1q3N1N/4PPP1/1P2Q3/4R2P/2R3K1 w - - 0 1}
\chessboard[smallboard,
marginleft=false,
marginrightwidth=2em,
moverstyle=triangle]
\begin{wraptable}{r}{0.5\textwidth}
	\vspace{-13em}

White is two Pawns behind. He must therefore press on his attack. The game continues:

\mainline{1.Nfxg7 Nc5}

\end{wraptable}

Evidently an error which made the winning easier for White, as he simply took the Rook with the Knight and kept up the attack. Black should have played: 

\variation{1... Nxg7, 2.Nf6+ Kg6, 3.Nxd7 f6, 4.e5 Kf4, 5.Nxf6 Re7, 6.Ne4} and Black Should lose.\footnote{Full score and notes are given in My Chess Career, by J. R. Capablanca (Game No. 11). Jose Raul Capablanca vs Ossip Bernstein, San Sebastian (1911), San Sebastian ESP, rd 1, Feb-20}

\subsubsection*{Example 47}

\newgame
\fenboard{r1bq1rk1/pp2nppp/4p3/2n5/8/2NBPN2/PP3PPP/R2Q1RK1 w - - 0 1}
\chessboard[smallboard,
marginleft=false,
marginrightwidth=2em,
moverstyle=triangle]
\begin{wraptable}{r}{0.5\textwidth}
	\vspace{-13em}
	
The student should carefully examine the position, as the sacrifice of the Bishop in similar situations is typical, and the chance for it is of frequent occurrence in actual play. The game continues:

\end{wraptable}

\mainline{1.Bxh7+ Kxh7, 2.Ng5+ Kg6} best.

\variation{2... Kh6, 3.Nxf7+} wins the Queen.

\variation{2... Kg8, 3.Qh5} with an irresistible attack.

\mainline{3.Qg4 f5, 4.Qg3 Kh6} White finally won.\footnote{This is elaborated more in Example 50}

\section{Winning by Indirect Attack}

We have so far given positions where the attacks were of a violent nature and directed against the King's position. Very often, however, in the middle-game attacks are made against a position or against pieces, or even Pawns.

\emph{The winning of a Pawn among good players of even strength often means the winning of the game.}

Hence the study of such positions is of great importance. We give below two positions in which the attack aims at the gain of a mere Pawn as a means of ultimately winning the game.


\subsubsection*{Example 48}

\newgame
\fenboard{2r1r1k1/2p2pbp/2ppq1p1/8/2n1P3/2N2P2/PPPBQ1PP/1R3RK1 b - - 0 1}
\chessboard[smallboard,
marginleft=false,
marginrightwidth=2em,
moverstyle=triangle]
\begin{wraptable}{r}{0.5\textwidth}
	\vspace{-13em}
	
Back is a Pawn behind, and there is no violent direct attack against White's King. Black's pieces, however, are very well placed and free to act, and by co-ordinating the action of all his pieces he is soon able not only to regain the Pawn but to obtain the better game.

\end{wraptable}

The student should carefully consider this position and the subsequent moves. It is a very good example of proper co-ordination in the management of forces. The game continues:

\mainline{1... Ra8, 2.h4} white's best move was \wmove{b3}, when would follow \variation{2.b3 Nxd2, 3.Qxd2 Ra3} and black would ultimately win the Pawn on a2, always keeping a slight advantage in the position. The next move make matters easier.

\mainline{2... Nxd2, 3.Qxd2 Qc4, 4.Rfd1, Reb8} Black could have regained the pawn by playing \bmove{Bxc3}, but there is more to be had, and therefore increases the pressure against White's Queen side. Now threatening, amove other things, \bmove{b2}

\mainline{5.Qe3 Rb4} Threatening to win the exchange by \bmove{Bd4}.

\mainline{6.Qg5 Bd4+, 7.Kh1 Rab8} This threatens to with the Knight , and thus forces White to give up the exchange.

\mainline{8.Rxd4 Qxd4, 9.Rd1 Qc4} Now Black will recover the Pawn.

\subsubsection*{Example 49}

\newgame
\fenboard{r4rk1/p3q1pp/1pb5/2pn1p2/8/3BPN2/PP2QPPP/2RR2K1 w - - 0 1}
\chessboard[smallboard,
marginleft=false,
marginrightwidth=2em,
moverstyle=triangle]
\begin{wraptable}{r}{0.5\textwidth}
	\vspace{-13em}
	
An examination of this position will show that Black's main weakness lies in the exposed position of the King, and in the fact that his Rook on e8 has not yet come into the game.

\end{wraptable}

Indeed, if it were Black's move, we might conclude that Black has the better game, on account of having three Pawns to two on the Queen's side, and his Bishop commanding the long diagonal.

It is, however, White's move, and has two courses to choose from. The obvious move, \wmove{Bc4}, might be good enough, since after \variation{1.Bc4 Rd8, 2.b4} would make it difficult for Black. But there is another move which completely upsets Black's position and wins a Pawn, besides obtaining the better position. That move is \wmove{Nd4} The game continues as follows:

\mainline{1.Nd4! cxd4, 2.Rxc6 Nb4} There is nothing better, as White threatened \wmove{Bc4}

\mainline{3.Bc4+ Kh8, 4.Re6 d3, 5.Rxd3} And White, with the better position, is a Pawn ahead.

These positions have been given with the idea of acquainting the student with different types of combinations. I hope they will also help to develop the imagination, a very necessary quality in a good player. The student should note, in all these middle-game positions, that—

\emph{once the opportunity is offered, all the pieces are thrown into action "en masse" when necessary; and that all the pieces smoothly co-ordinate their action with machine-like precision.}

That, at least, is what the ideal middle-game play should be, if it is not so altogether in these examples.

\chapter{General Theory}

Before we revert to the technique of the openings it will be advisable to dwell a little on general theory, so that the openings in their relation to the rest of the game may be better understood.

\section{The Initiative}

As the pieces are set on the board both sides have the same position and the same amount of material. White, however, has the move, and the move in this case means \emph{the initiative}, and the initiative, other things being equal, is an advantage. Now this advantage must be kept as long as possible, and should only be given up if some other advantage, material or positional, is obtained in its place. White, according to the principles already laid down, develops their pieces as fast as possible, but in so doing also tries to hinder their opponent's development, by applying pressure wherever possible. They try first of all to control the centre, and failing this to obtain some positional advantage that will make it possible for them to keep on harassing the enemy. They only relinquish the initiative when they get for it some material advantage under such favourable conditions as to make them feel assured that they will, in turn, be able to withstand their adversary's thrust; and finally, through their superiority of material, once more resume the initiative, which alone can give them the victory. This last assertion is self-evident, since, in order to win the game, the opposing King must be driven to a position where they are attacked without having any way of escape. Once the pieces have been properly developed the resulting positions may vary in character. It may be that a direct attack against the King is in order; or that it is a case of improving a position already advantageous; or, finally, that some material can be gained at the cost of relinquishing the initiative for a more or less prolonged period.

\section{Direct Attacks \emph{En Masse}}

In the first case the attack must be carried on with sufficient force to guarantee its success. Under no consideration must a direct attack against the King be carried on \emph{à outrance} unless there is absolute certainty in one's own mind that it will succeed, since failure in such cases means disaster.

\subsubsection*{Example 50}

A good example of a successful direct attack against the King is shown in the following diagram:

\newgame
\fenboard{r1bq1rk1/pp2nppp/4p3/2n5/8/2NBPN2/PP3PPP/R2Q1RK1 w - - 0 12}
\chessboard[smallboard,
marginleft=false,
marginrightwidth=2em,
moverstyle=triangle]
\begin{wraptable}{r}{0.5\textwidth}
	\vspace{-13em}

In this position White could simply play \wmove{bc2} and still have the better position, but instead he prefers an immediate attack on the King's side, with the certainty in his mind that the attack will lead to a win. The game continues thus:\footnotemark

\end{wraptable}

\footnotetext{We give, from now on, games and notes, so that the student may familiarise themself with the many and varied considerations that constantly are borne in mind by the Chess Master. We must take it for granted that the student has already reached a stage where, while not being able fully to understand every move, yet he can derive benefit from any discussion with regard to them.}

\mainline{12.Bxh7+ Kxh7, 13.Ng5+ Kg6, 14.Qg4 f5!} \bmove{e5} would have been immediately fatal. Thus: \variation{14... e5, 15.Ne6+ Kf6, 16.f4! e4, 17.Qg5+ Kxe6, 18.Qe5+ Kd7, 19.Rfd1+ Nd3, 20.Nxe4 Kc6, 21.Rxd3 Qxd3, 22.Rc1+ Kb6, 23.Qc7+ Ka6, 24.Nc5+ Kb5, 25.a4+ Kb4, 26.Nxd3+ Kxa4, 27.Ra1+ Kb3, 28.Ra3#}

\mainline{15.Qg3 Kh6, 16.Qh4+ Kg6, 17.Qh7+ Kf6} if \variation{17... Kxg5, 18. Qxg7+} and mate in a few more moves.

\mainline{18.e4 Ng6, 19.exf5 exf5, 20.Rad1 Nd3, 21.Qh3 Ndf4, 22.Qg3 Qc7, 23.Rfe1 Ne2+} This blunder loses at once, but the game could not be save in any case; e.g. \variation{23... Be6, 24.Rxe6+ Nxe6, 25.Nd5#}.

\mainline{24.Rxe2 Qxg3, 25.Nh7 Kf7, 26.hxg3 Rh8, 27.Ng5+ Kf6, 28.f4} Resigns.

\subsubsection*{Example 51}
Another example of this kind:

\newgame
\fenboard{r1bq1k1r/b5pp/1nRN4/4p3/1P2P1n1/5NB1/P4PPP/3Q1RK1 w - - 0 21}
\chessboard[smallboard,
marginleft=false,
marginrightwidth=2em,
moverstyle=triangle]
\begin{wraptable}{r}{0.5\textwidth}
	\vspace{-13em}

In the above position the simple move \wmove{Ne5} would win, but White looks for complications and their beauties. Such a course is highly risky until a wide experience of actual masterplay has developed a sufficient insight into all the possibilities of a position. 

\end{wraptable}

This game, which won the brilliancy prize at St. Petersburg in 1914, continued as follows:

\mainline{21.Bh4 Qd7, 22.Nxc8 Qxc6, 23.Qd8+ Qe8} If \variation{23... Kf7, 24.Nd6+} King moves; 25 mate.

\mainline{24.Be7+ Kf7, 25.Nd6+ Kg6, 26.Nh4+ Kh5} if \variation{26... Kh6, 27.Ndf5+ Kh5, 28.Ng3+ Kh6, 29.Bg5#}

\mainline{27.Nxe8 Rxd8, 28.Nxg7 Kh6, 29.Ngf5+ Kh5, 30.h3!}

The climax of the combination started with \wmove{21.Bh4}. White is still threatening mate, and the best way to avoid it is for Black to give back all the material he has gained and to remain three Pawns behind.

The student should note that in the examples given the attack is carried out with every available piece, and that often, as in some of the variations pointed out, it is the coming into action of the last available piece that finally overthrows the enemy. It demonstrates the principle already stated:

\emph{Direct and violent attacks against the King must be carried en masse, with full force, to ensure their success. The opposition must be overcome at all cost; the attack cannot be broken off, since in all such cases that means defeat.}

\begin{center}
\chessboard[largeboard,
moverstyle=triangle]
\end{center}

\clearpage

\section{The Force of the Threatened Attack}

Failing an opportunity, in the second case, for direct attack, one must attempt to increase whatever weakness there may be in the opponent's position; or, if there is none, one or more must be created. It is always an advantage to threaten something, but such threats must be carried into effect only if something is to be gained immediately. For, holding the threat in hand, forces the opponent to provide against its execution and to keep material in readiness to meet it. Thus he may more easily overlook, or be unable to parry, a thrust at another point. But once the threat is carried into effect, it exists no longer, and your opponent can devote his attention to his own schemes. One of the best and most successful manœuvres in this type of game is to make a demonstration on one side, so as to draw the forces of your opponent to that side, then through the greater mobility of your pieces to shift your forces quickly to the other side and break through, before your opponent has had the time to bring over the necessary forces for the defence.

A good example of positional play is shown in the following game:

\subsubsection*{Example 52}
Played at the Havana International Masters Tournament, 1913. (French Defence.) White: J. R. Capablanca. Black: R. Blanco.

\newgame

\mainline{1.e4 e6, 2.d4 d5, 3.Nc3 dxe4, 4.Nxe4 Nd7, 5.Nf3 Ngf6, 6.Nxf6+ Nxf6, 7.Ne5}

\chessboard[smallboard,
marginleft=false,
marginrightwidth=2em,
moverstyle=triangle]
\begin{wraptable}{r}{0.5\textwidth}
	\vspace{-13em}

This move was first shown to me by the talented Venezuelan amateur, M. Ayala. The object is to prevent the development of Black's Queen's Bishop viâ \bmove{b7}, after \bmove{b6}, which is Black's usual development in this variation. 

\end{wraptable}

Generally it is bad to move the same piece twice in an opening before the other pieces are out, and the violation of that principle is the only objection that can be made to this move, which otherwise has everything to recommend it.

\chessboard[smallboard,
marginleft=false,
marginrightwidth=2em,
moverstyle=triangle]
\begin{wraptable}{r}{0.5\textwidth}
	\vspace{-13em}

\mainline{7... Bd6, 8.Qf3} \variation{8.Bg5} might be better. The next move gives Black an opportunity of which he does not avail himself

\end{wraptable}

\mainline{8... c6} \bmove{c5} was the right move. It would have led to complications, in which Black might have held his own; 

at least, White's play would be very difficult. The next move accomplishes nothing, and puts Black in an altogether defensive position. The veiled threat \bmove{Bxe5}; followed by \bmove{Qa5+}; is easily met.

\mainline{9.c3 O-O, 10.Bg5 Be7} The fact that Black has now to move the Bishop back clearly demonstrates that Black's plan of development is faulty. Too much time has been lost, and White brings the pieces into their most attacking position without hindrance of any sort.

\mainline{11.Bd3 Ne8} The alternative was \bmove{Nd5}. Otherwise White would play \wmove{Qh3}, and Black would be forced to play \bmove{g6} (not \bmove{h6}, because of the sacrifice \wmove{bxh6}), seriously weakening his King's side.

\mainline{12.Qh3 f5} White has no longer an attack, but has compelled Black to create a marked weakness. Now White's whole plan will be to exploit this weakness (the weak e Pawn), and the student can now see how the principles expounded previously are applied in this game. Every move is directed to make the weak King's Pawn untenable, or to profit by the inactivity of the Black pieces defending the Pawn, in order to improve the position of White at other points.

\mainline{13.Bxe7 Qxe7, 14.O-O Rf6, 15.Rfe1 Nd6, 16.Re2 Bd7} At last the Bishop comes out, not as an active attacking piece, but merely to make way for the Rook.

\mainline{17.Rae1 Re8, 18.c4 Nf7} A very clever move, tending to prevent \wmove{b5}, and tempting White to play \wmove{Nxd7}, followed by \wmove{bxf5}, which would be bad, as the following variation shows: 
\variation{18... Nf7, 19.Nxd7 Qxd7, 20.Bxf5 Ng5, 21.Qg4 Rxf5, 22.h4 h5, 23.Qxf5 exf5, 24.Rxe8+ Kh7, 25.hxg5 Qxd4} But it always happens in such cases that, if one line of attack is anticipated, there is another; and this is no exception to the rule, as will be seen.

\chessboard[smallboard,
marginleft=false,
marginrightwidth=2em,
moverstyle=triangle]
\begin{wraptable}{r}{0.5\textwidth}
	\vspace{-13em}

\mainline{19.d5! Nxe5} Apparently the best way to meet the manifold threats of White. \bmove{cxd5} would make matters worse, as the White Bishop would finally bear on the weak King's Pawn viâ \wmove{c4}.

\end{wraptable}

\mainline{20.Rxe5 g6, 21.Qh4 Kg7, 22.Qd4 c5} Forced, as White threatened \wmove{dxe6}, and also \wmove{Qxa7}.

\mainline{23.Qc3 b6} \bmove{Qd3} was better. But Black wants to tempt White to play \wmove{dxe6}, thinking that he will soon after regain the Pawn with a safe position. Such, however, is not the case, as White quickly demonstrates. I must add that in any case Black's position is, in my opinion, untenable, since all his pieces are tied up for the defence of a Pawn, while White's pieces are free to act.

\mainline{24.dxe6 Bc8}

\chessboard[smallboard,
marginleft=false,
marginrightwidth=2em,
moverstyle=triangle]
\begin{wraptable}{r}{0.5\textwidth}
	\vspace{-13em}

\mainline{25.Be2!}The deciding and timely manœuvre. All the Black pieces are useless after this Bishop reaches \wmove{Bd5}. 

\mainline{25... Bxe6, 26.Bf3 Kf7}

\end{wraptable}

\mainline{27.Bd5 Qd6} Now it is evident that all the Black pieces are tied up, and it only remains for White to find the quickest way to force the issue. White will now try to place his Queen at h6, and then advance the h Pawn to h5 in order to break up the Black Pawns defending the King.

\mainline{28.Qe3 Re7} if \variation{28... f4, 29.Qh3 h5, 30.Qh4 Re7, 31.Qg5 Kg7, 32.h4 Qd7, 33.g3 fxg3, 34.f4} and Black will soon be helpless, as he has to mark time with his pieces while White prepares to advance \wmove{h5}, and finally at the proper time to play \wmove{Rxe6}, winning.

\mainline{29.Qh6 Kg8, 30.h4 a6, 31.h5 f4, 32.hxg6 hxg6, 33. Rxe6} Resigns.

Commenting on White's play in this game, Dr. E. Lasker said at the time that if White's play were properly analysed it might be found that there was no way to improve upon it.

These apparently simple games are often of the most difficult nature. Perfection in such cases is much more difficult to obtain than in those positions calling for a brilliant direct attack against the King, involving sacrifices of pieces.

\clearpage

\section{Relinquishing the Initiative}

In the third case, there is nothing to do, once the material advantage is obtained, but to submit to the opponent's attack for a while, and once it has been repulsed to act quickly with all your forces and win on material. A good example of this type of game is given below.

\subsubsection*{Example 53}

From the Havana International Masters Tournament, 1913. (Ruy Lopez.) White: J. R. Capablanca. Black: D. Janowski.

\newgame
\mainline{1.e4 e5, 2.Nf3 Nc6, 3.Bb5 Nf6, 4.O-O d6, 5.Bxc6+ bxc6, 6.d4 Be7, 7.Nc3} \wmove{dxe5} might be better, but at the time I was not familiar with that variation, and therefore I played what I knew to be good

\mainline{7... Nd7, 8.dxe5 dxe5, 9.Qe2 O-O, 10.Rd1 Bd6, 11.Bg5 Qe8, 12.Nh4 g6} Black offers the exchange in order to gain time and to obtain an attack. Without considering at all whether or not such a course was justified on the part of Black, it is evident that as far as White is concerned there is only one thing to do, viz., to win the exchange and then prepare to weather the storm. Then, once it is passed, to act quickly with all forces to derive the benefit of numerical superiority.

\mainline{13.Bh6 Nc5, 14.Rd2 Rb8, 15.Nd1 Rb4} To force White to play \wmove{c4}, and thus create a hole at d4 for his Knight.\footnote{A "hole" in chess parlance has come to mean a defect in Pawn formation which allows the opponent to establish his forces in wedge formation or otherwise without the possibility of dislodging him by Pawn moves. Thus, in the following diagram, Black has two holes at f6 and h6, where White forces, e.g. a Knight or Bishop, could establish themselves, supported by pieces or Pawns.} Such grand tactics show the hand of a master.

\mainline{16.c4 Ne6, 17.Bxf8 Qxf8, 18.Ne3} \wmove{Nf3} Was better.

\mainline{18... Nd4, 19.Qd1 c5} In order to prevent \wmove{Rxd4} giving back the exchange, but winning a Pawn and relieving the position.

\mainline{20.b3 Rb8} In order to play \bmove{Bb7} without blocking his Rook.

Black's manœuvring for positional advantage is admirable throughout this game, and if he loses it is due entirely to the fact that the sacrifice of the exchange, without even a Pawn for it, could not succeed against sound defensive play.

\chessboard[smallboard,
marginleft=false,
marginrightwidth=2em,
moverstyle=triangle]
\begin{wraptable}{r}{0.5\textwidth}
	\vspace{-13em}

\mainline{21.Nf3 f5, 22.exf5 gxf5}

\end{wraptable}

\chessboard[smallboard,
marginleft=false,
marginrightwidth=2em,
moverstyle=triangle]
\begin{wraptable}{r}{0.5\textwidth}
	\vspace{-13em}

The position begins to look really dangerous for White. In reality Black's attack is reaching its maximum force. Very soon it will reach the apex, and then White, who is well prepared, will begin his counter action, and through his superiority in material obtain an undoubted advantage.

\end{wraptable}

\mainline{23.Nf1 f4 24.Nxd4 cxd4 25.Qh5 Bb7 26.Re1 c5} He could not play \bmove{Re8} because of \wmove{Rxd4}. Besides, he wants to be ready to play \bmove{e4}. At present White cannot with safety play \wmove{Rxe5}, but he will soon prepare the way for it. Then, by giving up a Rook for a Bishop and a Pawn, he will completely upset Black's attack and come out a Pawn ahead. It is on this basis that White's whole defensive manœuvre is founded.

\mainline{27.f3 Re8, 28.Rde2 Re6} Now the Black Rook enters into the game, but White is prepared. It is now time to give back the exchange.

\chessboard[smallboard,
marginleft=false,
marginrightwidth=2em,
moverstyle=triangle]
\begin{wraptable}{r}{0.5\textwidth}
	\vspace{-13em}

\mainline{29.Rxe5 Bxe5, 30.Rxe5 Rh6, 31.Qe8 Qxe8, 32.Rxe8+ Kf7, 33.Re5 Rc6, 34.Nd2}

\end{wraptable}

\wmove{Rf5} might have been better. The next move did not prove as strong as anticipated.

\mainline{34... Kf6 35.Rd5 Re6, 36.Ne4+ Ke7} \bmove{Rxe4} would lose easily

\mainline{37.Rxc5 d3!} Very fine. White cannot play \wmove{Rc7+} because of \bmove{Kd8, 39.Rxb7 Rxe4} is winning.

\mainline{38.Kf2 Bxe4 39.fxe4 Rxe4 40.Rd5 Re3} The ending is very difficult to win. At this point White had to make the last move before the game was adjourned.

\chessboard[smallboard,
marginleft=false,
marginrightwidth=2em,
moverstyle=triangle]
\begin{wraptable}{r}{0.5\textwidth}
	\vspace{-13em}

\mainline{41.b4! Re4 42.Rxd3 Rxc4 43.Rh3 Rxb4 44.Rxh7+ Kf6 45.Rxa7 Kf5 46.Kf3 Rb2 47.Ra5+ Kf6 }

\end{wraptable}

\mainline{48.Ra4 Kg5 49.Rxf4 Rxa2 50.h4+ Kh5 51.Rf5+ Kh6 52.g4} Resigns

I have passed over the game lightly because of its difficult nature, and because we are at present concerned more with the opening and the middle-game than we are with the endings, which will be treated separately.

\begin{center}
\chessboard[normalboard,
moverstyle=triangle]
\end{center}

\clearpage

\section{Cutting off Pieces from the Scene of Action}

Very often in a game a master only plays to cut off, so to speak, one of the pieces from the scene of actual conflict. Often a Bishop or a Knight is completely put out of action. In such cases we might say that from that moment the game is won, because for all practical purposes there will be one more piece on one side than on the other. A very good illustration is furnished by the following game. 

\subsubsection*{Example 54}

Played at the Hastings Victory Tournament, 1919. (Four Knights.) White: W. Winter. Black: J. R. Capablanca.

\newgame
\mainline{1.e4 e5, 2.Nf3 Nc6, 3.Nc3 Nf6, 4.Bb5 Bb4, 5.O-O O-O, 6.Bxc6} Niemzowitch's variation, which I have played successfully in many a game. It gives White a very solid game. Niemzowitch's idea is that White will in due time be able to play \wmove{f4}, opening a line for his Rooks, which, in combination with the posting of a Knight at f5, should be sufficient to win. He thinks that should Black attempt to stop the Knight from going to f5, he will have to weaken his game in some other way. Whether this is true or not remains to be proved, but in my opinion the move is perfectly good. On the other hand, there is no question that Black can easily develop his pieces. But it must be considered that in this variation White does not attempt to hinder Black's development, he simply attempts to build up a position which he considers impregnable and from which he can start an attack in due course.

\mainline{6... dxc6} The alternative, \bmove{bxc6}; gives White the best of the game, without doubt.\footnote{See game Capablanca-Kupchick, from Havana International Masters Tournament Book, 1913, by J. R. Capablanca; or a game in the Carlsbad Tournament of 1911, Vidmar playing Black against Alechin.}

\mainline{7.d3 Bd6, 8.Bg5} This move is not at all in accordance with the nature of this variation. The general strategical plan for White is to play \wmove{h3}, to be followed in time by the advance of the g Pawn to g4, and the bringing of the Knight to f5 via e2 and g3 or d1 and e3. Then, if possible, the Knight is linked with the other Knight by placing it at either h4, g3, or  e3 as the occasion demands. The White King sometimes remains at g1, and other times it is placed at g2, but mostly at h1. Finally, in most cases comes the advancement of the f Pawn to f4, and then the real attack begins. Sometimes it is a direct assault against the King,\footnote{See Niemzowitch's game in the All Russian Masters Tournament, 1914, at St. Petersburg, against Levitzki, I believe.} and at other times it comes simply to finessing for positional advantage in the end-game, after most of the pieces have been exchanged.\footnote{See Capablanca-Janowski game, New York Masters Tournament, 1913.}

\mainline{8... h6 9.Bh4 c5}

\chessboard[smallboard,
marginleft=false,
marginrightwidth=2em,
moverstyle=triangle]
\begin{wraptable}{r}{0.5\textwidth}
	\vspace{-13em}

To prevent \wmove{d4} and to draw White into playing \wmove{Kd5}, which would prove fatal. Black's plan is to play \bmove{g5}, as soon as the circumstances permit, in order to free his Queen and Knight from the pin by the Bishop.

\end{wraptable}

\mainline{10.Nd5} White falls into the trap. Only lack of experience can account for this move. White should have considered that a player of my experience and strength could never allow such a move if it were good.

\mainline{10... g5}

\chessboard[smallboard,
marginleft=false,
marginrightwidth=2em,
moverstyle=triangle]
\begin{wraptable}{r}{0.5\textwidth}
	\vspace{-13em}

After this move White's game is lost. White cannot play \wmove{Kxf6}, because \wmove{Kxf6} will win a piece. Therefore he must play \wmove{Bxf6}, either before or after Knight takes Knight, with disastrous results in either case, as will be seen.

\end{wraptable}

\mainline{11.Nxf6+ Qxf6, 12.Bg3 Bg4, 13.h3 Bxf3, 14.Qxf3 Qxf3, 15.gxf3 f6}

\chessboard[smallboard,
marginleft=false,
marginrightwidth=2em,
moverstyle=triangle]
\begin{wraptable}{r}{0.5\textwidth}
	\vspace{-13em}

A simple examination will show that White is minus a Bishop for all practical purposes. He can only free it by sacrificing one Pawn, and possibly not even then. At least it would lose time besides the Pawn. 

\end{wraptable}

Black now devotes all his energy to the Queen's side, and, having practically a Bishop more, the result cannot be in doubt. The rest of the game is given, so that the student may see how simple it is to win such a game. 

\mainline{16.Kg2 a5, 17.a4 Kf7, 18.Rh1 Ke6, 19.h4 Rfb8} There is no necessity to pay any attention to the King's side, because White gains nothing by exchanging Pawns and opening the King's Rook file.

\mainline{20.hxg5 hxg5, 21.b3 c6, 22.Ra2 b5, 23.Rha1 c4} If White takes the proffered Pawn, Black regains it immediately by \bmove{Rb4}, after \bmove{bxc4}.

\mainline{24.axb5 cxb3, 25.cxb3 Rxb5, 26.Ra4 Rxb3, 27.d4 Rb5, 28.Rc4 Rb4, 29.Rxc6 Rxd4} Resigns

\begin{center}
\chessboard[largeboard,
moverstyle=triangle]
\end{center}

\clearpage


\section{A Player's Motives Criticised in a Specimen Game}

Now that a few of my games with my own notes have been given, I offer for close perusal and study a very fine game played by Sir George Thomas, one of England's foremost players, against Mr. F. F. L. Alexander, in the championship of the City of London Chess Club in the winter of 1919-1920. It has the interesting feature for the student that Sir George Thomas kindly wrote the notes to the game for me at my request, and with the understanding that I would make the comments on them that I considered appropriate. Sir George Thomas' notes are in brackets and thus will be distinguished from my own comments.

\subsubsection*{Example 55}

Queen's Gambit Declined. (The notes within brackets by Sir George Thomas.) White: Mr. F. F. L. Alexander. Black: Sir George Thomas.

\newgame
\mainline{1.d4 d5, 2.Nf3 Nf6, 3.c4 e6, 4.Nc3 Nbd7, 5.Bg5 c6, 6.e3 Qa5}

\chessboard[smallboard,
marginleft=false,
marginrightwidth=2em,
moverstyle=triangle]
\begin{wraptable}{r}{0.5\textwidth}
	\vspace{-13em}

(One of the objects of Black's method of defence is to attack White's Knight doubly by \bmove{Ne4}, followed by \bmove{dxc4}. But \wmove{7.Ne2} is probably a strong way of meeting this threat.) 

\end{wraptable}

There are, besides, two good reasons for this method of defence; first, that it is not as much played as some of the other defences and consequently not so well known, and second that it leaves Black with two Bishops against Bishop and Knight, which, in a general way, constitutes an advantage.

\mainline{7.Bxf6 Nxf6, 8.a3 Ne4, 9.Qb3 Be7} This is not the logical place for the Bishop which should have been posted at d6. In the opening, time is of great importance, and therefore the player should be extremely careful in his development and make sure that he posts his pieces in the right places.

\mainline{10.Bd3 Nxc3, 11.bxc3 dxc4, 12.Bxc4 Bf6} (I did not want White's Knight to come to e5, from where I could not dislodge it by f6 without weakening my e Pawn.) The same result could be accomplished by playing \bmove{Bd6}. Incidentally it bears out my previous statement that the Bishop should have been originally played to d6.

\mainline{13.O-O} The alternative was \wmove{e4}, followed by \wmove{e5}, and then \wmove{O-O}. White would thereby assume the initiative but would weaken his Pawn position considerably, and might be compelled to stake all on a violent attack against the King. This is a turning point in the game, and it is in such positions that the temperament and style of the player decide the course of the game.

\mainline{13... O-O, 14.e4 e5}

\chessboard[smallboard,
marginleft=false,
marginrightwidth=2em,
moverstyle=triangle]
\begin{wraptable}{r}{0.5\textwidth}
	\vspace{-13em}

\mainline{15.d5} (White might play \wmove{15.Rfd1}, keeping the option of breaking up the centre later on. I wanted him to advance this Pawn as there is now a fine post for my Bishop at c5.) 

\end{wraptable}

By this move White shows that he does not understand the true value of his position. His only advantage consisted in the undeveloped condition of Black's White square Bishop. He should therefore have made a plan to prevent the Bishop from coming out, or if that were not possible, then he should try to force Black to weaken his Pawn position in order to come out with the Bishop. There were three moves to consider: first, \wmove{a4}, in order to maintain the White Bishop in the dominating position that it now occupies. This would have been met by \bmove{Qc7}; second, either of the Rooks to d8 in order to threaten \wmove{16.dxe5 Bxe5, 17.Kxe5 Qxe5, 18 Bxf7+}. This would have been met by \bmove{Bg4}; and third, \wmove{h3} to prevent \bmove{bg4} and by playing either Rook to d1, followed up as previously stated to force Black to play \bmove{b5}, which would weaken his Queen's side Pawns. Thus by playing \wmove{h3} White would have attained the desired object. The next move blocks the action of the White Bishop and facilitates Black's development. Hereafter White will act on the defensive, and the interest throughout the rest of the game will centre mainly on Black's play and the manner in which he carries out the attack.

\mainline{15... Qc7, 16.Bd3} (This seems wrong, as it makes the development of Black's Queen wing easier. At present he cannot play \bmove{b6}, because of the reply \wmove{dxe6} followed by \wmove{Bd5}.)

\mainline{16... b6, 17.c4 Bb7, 18.Rfc1} (With the idea of \wmove{Rab1} and \wmove{c5}. But it only compels Black to bring his Bishop to c5, which he would do in any case.)

\mainline{18... Be7, 19.Rc2 Bc5, 20.Qb2 f6} (It would have been better, probably, to play \wmove{20... Rfe1}, with the idea of \bmove{f5}  presently.) Black's play hereabout is weak; it lacks force, and there seems to be no well-defined plan of attack. It is true that these are the most difficult positions to handle in a game. In such cases a player must conceive a plan on a large scale, which promises chances of success, and with it all, it must be a plan that can be carried out with the means at his disposal. From the look of the position it seems that Black's best chance would be to mass his forces for an attack against White's centre, to be followed by a direct attack against the King. He should, therefore, play \bmove{Rae1}, threatening \bmove{f5}. If White is able to defeat this plan, or rather to prevent it, then, once he has fixed some of the White pieces on the King's side, he should quickly shift his attack to the Queen's side, and open a line for his Rooks, which, once they enter in action, should produce an advantage on account of the great power of the two Bishops.

\mainline{21.Rb1 Rad8, 22.a4 Ba6, 23.Rd1} (White has clearly lost time with his Rook's moves.)

\mainline{23... Rfe8, 24.Qb3} (To bring his Queen across after \wmove{h4} and \wmove{Be2}.)

\mainline{24... Rd6, 25.Nh4 g6, 26.Be2}

\chessboard[smallboard,
marginleft=false,
marginrightwidth=2em,
moverstyle=triangle]
\begin{wraptable}{r}{0.5\textwidth}
	\vspace{-13em}

\mainline{26... cxd5} (I thought this exchange necessary here, as White is threatening to play his Bishop via \wmove{bg4} to \wmove{be6}. If he retook with the Bishop's Pawn I intended to exchange Bishops and rely on the two Pawns to one on the Queen's wing. 

\end{wraptable}

I did not expect him to retake it with the King's Pawn, which seemed to expose him to a violent King's side attack.) Black's judgment in this instance I believe to be faulty. Had White retaken with the c Pawn, as he expected, he would have had the worst of the Pawn position, as White would have had a passed Pawn well supported on the Queen's side. His only advantage would lie in his having a very well posted Bishop against a badly posted Knight, and on the fact that in such positions as the above, the Bishop is invariably stronger than the Knight. He could and should have prevented all that, by playing \bmove{Bc8}, as, had White then replied with \wmove{Qg3}, he could then play \bmove{cxd5}, and White would not have been able to retake with the c P on account of \bmove{Bxf2} winning the exchange.

\mainline{27.exd5 e4, 28.g3 e3} I do not like this move. It would have been better to hold it in reserve and to have played \bmove{f5}, to be followed in due time by \bmove{g5} and \bmove{f4}, after having placed the Queen at \bmove{d7}, \bmove{f7}, or some other square as the occasion demanded. The next move blocks the action of the powerful Bishop at c5, and tends to make White's position safer than it should have been. The move in itself is a very strong attacking move, but it is isolated, and there is no effective continuation. Such advances as a rule should only be made when they can be followed by a concerted action of the pieces.

\mainline{29.f4 Bc8, 30.Nf3 Bf5, 31.Rb2 Re4, 32.Kg2 Qc8, 33.Ng1 g5} (If now \wmove{34.Bf3 gxf4, 35.Bxe4 Bxe4+}, with a winning attack.)

\mainline{34.fxg5 fxg5, 35.Rf1 g4} \bmove{Rh6} was the alternative. White's only move would have been \wmove{Kh1}. The position now is evidently won for Black, and it is only a question of finding the right course. The final attack is now carried on by Sir George Thomas in an irreproachable manner.

\mainline{36.Bd3 Rf6, 37.Ne2 Qf8}

\chessboard[smallboard,
marginleft=false,
marginrightwidth=2em,
moverstyle=triangle]
\begin{wraptable}{r}{0.5\textwidth}
	\vspace{-13em}

(Again preventing \wmove{Bxe4}, by the masked attack on White's Rook. White therefore protects his Rook.) If \wmove{38.Nf4 e2!, 39.Nxe2 Rxe2, 40.Rxe2}
\wmove{Be4+!!, 41.Bxe4! Rxf1} 

\end{wraptable}

and White is lost. If, however, against \wmove{38.Nf4}, Black plays \bmove{Qh6}, and White \wmove{39.Qc2}, I take pleasure in offering the position to my readers as a most beautiful and extraordinary win for Black, beginning with \wmove{39... Qh3!!!} I leave the variations for the student to work out.

\mainline{38.Rbb1 Qh6 39.Qc2} (Making a double attack on the Rook which still cannot be taken and preparing to defend the h Pawn.) If either the Rook or Bishop are taken White would be mated in a few moves.

\mainline{39...  Qh3+, 40.Kh1 Rxc4!!}

\chessboard[smallboard,
marginleft=false,
marginrightwidth=2em,
moverstyle=triangle]
\begin{wraptable}{r}{0.5\textwidth}
	\vspace{-13em}

(If \variation{40... Rh6, 41.Ng1 Qxg3, 42.Qg2}. Black therefore tries to get the Queen away from the defence.) A very beautiful move, and the best way to carry on the attack.

\end{wraptable}

\mainline{41.Qxc4} (The best defence was \variation{41.Rxf5}, but Black would emerge with Queen against Rook and Knight.)

\mainline{41... Bxd3} (Again, not \bmove{Rh6}; because of \wmove{d6+})

\mainline{42.Rxf6} (If \variation{42.Qxd3 Rh6} wins for Black.)

\mainline{42... Bxc4, 43.Nf4 e2!}

\chessboard[smallboard,
marginleft=false,
marginrightwidth=2em,
moverstyle=triangle]
\begin{wraptable}{r}{0.5\textwidth}
	\vspace{-13em}

(The Queen has no escape, but White has no time to take it.)

\mainline{44.Rg1 Qf1} White resigns. A very fine game

\end{wraptable}

\clearpage

\chapter{End-Game Strategy}

We must now revert once more to the endings. Their importance will have become evident to the student who has taken the trouble to study my game with Janowski (Example 53). After an uneventful opening a Ruy Lopez in one of its normal variations, my opponent suddenly made things interesting by offering the exchange; an offer which, of course, I accepted. Then followed a very hard, arduous struggle, in which I had to defend myself against a very dangerous attack made possible by the excellent manœuvring of my adversary. Finally, there came the time when I could give back the material and change off most of the pieces, and come to an ending in which I clearly had the advantage. But yet the ending itself was not as simple as it at first appeared, and finally perhaps through one weak move on my part it became a very difficult matter to find a win. Had I been a weak end-game player the game would probably have ended in a draw, and all my previous efforts would have been in vain. Unfortunately, that is very often the case among the large majority of players; they are weak in the endings; a failing from which masters of the first rank are at times not free. Incidentally, I might call attention to the fact that all the world's champions of the last sixty years have been exceedingly strong in the endings: Morphy, Steinitz, and Dr. Lasker had no superiors in this department of the game while they held their titles.

\section{The Sudden Attack from a Different Side}

I have previously stated, when speaking about general theory, that at times the way to win consists in attacking first on one side, then, granted greater mobility of the pieces, to transfer the attack quickly from one side to the other, breaking through before your opponent has been able to bring up sufficient forces to withstand the attack. This principle of the middle-game can sometimes be applied in the endings in somewhat similar manner.

\subsubsection*{Example 56}

\newgame
\fenboard{6k1/p4p1p/1p4p1/3r4/2r5/P1P4P/2R2PP1/2R1K3 b - - 0 1}
\chessboard[smallboard,
marginleft=false,
marginrightwidth=2em,
moverstyle=triangle]
\begin{wraptable}{r}{0.5\textwidth}
	\vspace{-13em}

In the above position I, with the Black pieces, played:
\mainline{1... Re4+, 2.Re2 Ra4, 3.Ra2 h5}

\end{wraptable}

The idea, as will be seen very soon, is to play \bmove{h6} in order to fix White's King's side Pawns with a view to the future. It is evident to Black that White wants to bring his King to \wmove{Kd3} to support his two weak isolated Pawns, and thus to free his Rooks. Black, therefore, makes a plan to shift the attack to the King's side at the proper time, in order to obtain some advantage from the greater mobility of his Rooks.

\mainline{4.Rd1 Rda5} In order to force the Rook to Rook's square, keeping both Rooks tied up.

\mainline{5.Rda1 h4, 6.Kd2 Kg7, 7.Kc2 Rg5} Black begins to transfer his attack to the King's side.

\mainline{8.Rg1}A serious mistake, which loses quickly. White should have played \variation{8.Kb3 Raa5, 9.g3} and Black would have obtained an opening at g6 for his King, which in the end might give him the victory.

\mainline{8... Rf4} Now the King cannot go to b3, because of \bmove{Rb5+}.

\mainline{9.Kd3 Rf3+, 10.Ke2} If \variation{10.gxf3 Rxg1}; followed by \bmove{Rh1} winning,

and Black won after a few moves.

\subsubsection*{Example 57}

Another good example, in which is shown the advantage of the greater mobility of the pieces in an ending, is the following from a game Capablanca-Kupchick played at the Havana Masters Tournament, 1913. The full score and notes of the game can be found in the book of the tournament.

\newgame
\fenboard{2r2rk1/p1pp1p1p/2p2p2/8/8/2PP1P2/PP3P1P/R3R1K1 w - - 0 1}
\chessboard[smallboard,
marginleft=false,
marginrightwidth=2em,
moverstyle=triangle]
\begin{wraptable}{r}{0.5\textwidth}
	\vspace{-13em}
	
White's only advantage in the above position is that they possesses the open file and has the move, which will secure them the initiative. There is also the slight advantage of having Pawns on the Queen's side united, while Black has an isolated a Pawn. The proper course,

\end{wraptable}

as in the previous ending, is to bring the Rooks forward, so that at least one of them may be able to shift from one side of the board to the other, and thus keep Black's Rooks from moving freely.

What this means in general theory has been stated already; it really means: \emph{keep harassing the enemy; force them to use his big pieces to defend Pawns. If he has a weak point, try to make it weaker, or create another weakness somewhere else and his position will collapse sooner or later. If he has a weakness, and he can get rid of it, make sure that you create another weakness somewhere else.}

From the position in question the game continued thus:

\mainline{1.Re4 Rfe8} with the object of repeating White's manœuvre, and also not to allow White the control of the open file.

\mainline{2.Rae1 Re6, 3.R1e3 Rce8, 4.Kf1 Kf8} Black wants to bring his King to the centre of the board in order to be nearer to whatever point White decides to attack. The move is justified at least on the general rule that in such endings the King should be in the middle of the board. He does nothing after all but follow White's footsteps. Besides, it is hard to point out anything better. If \variation{4... d5, 5.Rg4+ Kf8} Black will eventually follow up with \bmove{Ke7}, this would leave Black in a very disagreeable position. If \variation{4... f5, 5.Rd4! Rxe3?, 6.fxe3 Rxe3, 7.Kf2 Re7, 8.Rh4}, winning the Pawn on h7, which would practically leave White with a passed Pawn ahead on the Queen's side, as the three Pawns of Black on the King's side would be held by the two of White.

\mainline{5.Kf2 Ke7, 6.Ra4 Ra8} The student should note that through the same manœuvre Black is forced into a position similar to the one shown in the previous ending.

\mainline{7.Ra5!} This move has a manifold object. It practically fixes all of Black's Pawns except the d Pawn, which is the only one that can advance two squares. It specially prevents the advance of Black's f Pawns, and at the same time threatens the advance of White's f Pawns to f4 and f5. By this threat it practically forces Black to play \bmove{d4}, which is all White desires, for reasons that will soon become evident.

\mainline{7... d5, 8.c4! Kd6} Evidently forced, as the only other move to save a Pawn would have been \bmove{dxc4}, which would have left all Black's Pawns isolated and weak. If \variation{8... d4, 9.Re4 Kd6, 10.b4! Re5, 11.Ra6}, and Black's game is hopeless.

\mainline{9.c5+ Kd7, 10.d4 f5} Apparently very strong, since it forces the exchange of Rooks because of the threat \wmove{Ra6}; but in reality it leads to nothing. The best chance was to play \bmove{Rae8}.

\mainline{11.Rxe6 fxe6, 12.f4} Up to now White had played with finesse, but this last move is weak. \wmove{Ra6} was the proper way to continue, so as to force Black to give up his a or c Pawn.

\mainline{12... Kc8, 13.Kd2} Again a bad move. The proper continuation, \variation{13.Ra3 Rb8, 14.b3 Kb7, 15.b4 Ka8, 16.Rb3}, with excellent winning chances; in fact, I believe, a won game.

\chessboard[smallboard,
marginleft=false,
marginrightwidth=2em,
moverstyle=triangle]
\begin{wraptable}{r}{0.5\textwidth}
	\vspace{-13em}

\mainline{13... Kb7} Black misses the only chance. \bmove{Rb8} would have drawn.

\mainline{14.Ra3 Rg8, 15.Rh3 Rg7, 16.Ke2 Ka6,}

\end{wraptable}

\mainline{17.Rh6 Re7, 18.Kd3 Kb7} Black goes back with the King to support the e Pawn, and thus be able to utilise the Rook. It is, however, useless, and only White's weak play later on gives Black further chances of a draw.

\mainline{19.a4 Kc8, 20.Rh5} To prevent the Black Rook from controlling the open file.

\mainline{20... Kd7, 21.Rg5 Rf7, 22.Kc3 Kc8} Black must keep the King on that side because White threatens to march with the King to a6 via b4.

\mainline{23.Kb4 Rf6, 24.Ka5 Kb7, 25.h4 a6, 26.h5 Rh6} Black can do nothing but wait for White. The next move stops White from moving the Rook, but only for one move.

\mainline{27.b4 Rf6} The only other move was \bmove{Ka7}; when White could play \wmove{Rg7}, or even \wmove{b5}.

\chessboard[smallboard,
marginleft=false,
marginrightwidth=2em,
moverstyle=triangle]
\begin{wraptable}{r}{0.5\textwidth}
	\vspace{-13em}

\mainline{28.b5} A weak move, which gives Black a fighting chance. In this ending, as is often the case with most players, White plays the best moves whenever the situation is difficult and requires careful handling, 

\end{wraptable}
 
but once his position seems to be overwhelming he relaxes his efforts and the result is nothing to be proud of. The right move was \variation{28.Rg7}.

\mainline{28... axb5, 29.axb5 Rf8!, 30.Rg7 Ra8+, 31.Kb4 cxb5, 32.Kxb5 Ra2, 33.c6+ Kb8, 34.Rxh7 Rb2+, 35.Ka5 Ra2+, 36.Kb4 Rxf2} Black misses the last chance: \bmove{Rb7+}, forcing the King to \wmove{c3}, in order to avoid the perpetual, would probably draw. The reader must bear in mind that my opponent was then a very young and inexperienced player, and consequently deserves a great deal of credit for the fight he put up.

\mainline{37.Re7 Rxf4} \bmove{Rb2+}; followed by \bmove{Rg7}, offered better chances.

\mainline{38.h6! Rxd4+, 39.Kb5 Rd1, 40.h7 Rb1+, 41.Kc5 Rc1+, 42.Kd4 Rd1+, 43.Ke5 Re1+, 44.Kf6 Rh1, 45.Re8+ Ka7, 46.h8=Q Rxh8, 47.Rxh8 Kb6, 48.Kxe6 Kxc6, 49.Kxf2 Kc5, 50.Ke5} Resigns.

This ending shows how easy it is to make weak moves, and how often, even in master-play, mistakes are made and opportunities are lost. It shows that, so long as there is no great advantage of material, even with a good position, a player, no matter how strong, cannot afford to relax his attention even for one move. 

\section{The Danger of as Safe Position}

\subsubsection*{Example 58}

\newgame
\hidemoves{1. e4 e5 2. d4 exd4 3. c3 d5 4. exd5 Qxd5 5. cxd4 Nc6 6. Nf3
Bg4 7. Be2 Nf6 8. h3 Bb4+ 9. Nc3 Bxf3 10. Bxf3 Qc4 11. Bxc6+
bxc6 12. Qe2+ Qxe2+ 13. Kxe2 O-O 14. Be3 Rfe8 15. Rac1 c5
16. dxc5 Bxc5 17. Nb5 Bxe3 18. fxe3 Rab8 19. Nxc7 Rxb2+
20. Kf3 Re5 21. Nd5 Rf5+ 22. Kg3 Rg5+ 23. Kf4 h6 24. Rc8+ Kh7
25. Nxf6+ gxf6}

\chessboard[smallboard,
marginleft=false,
marginrightwidth=2em,
moverstyle=triangle]
\begin{wraptable}{r}{0.5\textwidth}
	\vspace{-13em}

A good proof of the previous statement is shown in the following ending between Marshall and Kupchick in one of their two games in the same Tournament (Havana, 1913).

\end{wraptable}

It is evident that Marshall (White) is under great difficulties in the above position. Not only is he bound to lose a Pawn, but his position is rather poor. The best he could hope for was a draw unless something altogether unexpected happened, as it did. No reason can be given for Black's loss of the game except that he felt so certain of having the best of it with a Pawn more and what he considered a safe position, that he became exceedingly careless and did not consider the danger that actually existed. Let us see how it happened.

\mainline{26.g4 Rxa2} The mistakes begin. This is the first. Black sees that he can take a Pawn without any danger, and does not stop to think whether there is anything better. \bmove{Rf2+} was the right move. If then \wmove{Kg3, Rxa2}. If instead White played \wmove{Ke4, Re5+}, then followed by \bmove{Rxa2}.

\mainline{27.Rd1 Ra4+} Mistake number two, and this time such a serious one as to almost lose the game. The proper move was to play \bmove{f5} in order to break up White's Pawns and at the same time make room for the Black King, which is actually in danger, as will soon be seen.


\mainline{28.Rd4 Raa5} Mistake number three and this time fatal. His best move was \bmove{Rga5}. After the next move there is no defence. Black's game is lost. This shows that even an apparently simple ending has to be played with care. From a practically won position Black finds himself with a lost game, and it has only taken three moves.

\mainline{29.Rdd8 Rg7} If \variation{29... f5, 30.Rh8+ Kg6, 31.Rcg8+ Kf6, 32.Rxh6+ Rg6, 33.g5+ Ke7, 34.Rhxg6 fxg6, 35.Rg7+ Ke8, 36.Rxg6}, and wins easily.

\mainline{30.h4 h5, 31. Rh8+} Resigns.

The reason is evident. 

If \bmove{Kg6, 32.gxh5+ Rxh5, 33.Rxh5 Kxh5, 34.Rh8+ Kg6, 35.h5#}

\section{Endings With One Rook and Pawns}

The reader has probably realised by this time that endings of two Rooks and Pawns are very difficult, and that the same holds true for endings of one Rook and Pawns. Endings of two Rooks and Pawns are not very common in actual play; but endings of one Rook and Pawns are about the most common sort of endings arising on the chess board. Yet though they do occur so often, few have mastered them thoroughly. They are often of a very difficult nature, and sometimes while apparently very simple they are in reality extremely intricate. Here is an example from a game between Marshall and Rosenthal in the Manhattan Chess Club Championship Tournament of 1909-1910.

\newgame
\fenboard{8/7R/8/2k2P2/1p6/3r4/5PKP/8 w - - 0 1}
\chessboard[smallboard,
marginleft=false,
marginrightwidth=2em,
moverstyle=triangle]
\begin{wraptable}{r}{0.5\textwidth}
	\vspace{-13em}

In this position Marshall had a simple win by \wmove{Rc7+}, but played \wmove{f6}, and thereby gave Black a chance to draw. Luckily for him Black did not see the drawing move, played poorly, and lost. Had Black been up to the situation he would have drawn by playing \bmove{Rd6}.

\end{wraptable}

\mainline{1.f6 Rd6!} Now White has two continuations either \emph{(a)} \wmove{f7} or \emph{(b)} \wmove{Rc7+}. We have therefore;

\storegame{state1}

\emph{(a)} \mainline{2.f7 Rd8!, 3.Rh5+ Kc4} and White will finally have to sacrifice the Rook for Black's Pawn. Or -

\restoregame{state1}

\emph{(b)} \mainline{2.Rc7+ Kd4!, 3.f7 Rg6+} a very important move, as against \bmove{Rf6, Re7} wins.

\mainline{4.Kf1 Rf6, 5.Rb7 Kb6}

and White will finally have to sacrifice the Rook for the Pawn, or draw by perpetual check.

If there were nothing more in the ending it would not be of any great value, but there are other very interesting features. Now suppose that after \wmove{1.f6 Rd6, 2.f7}, Black did not realise that \bmove{Rd8} was the only move to draw. We would then have the following position:

\restoregame{state1}
\hidemoves{2.f7}
\storegame{state2}

\chessboard[smallboard,
marginleft=false,
marginrightwidth=2em,
moverstyle=triangle]
\begin{wraptable}{r}{0.5\textwidth}
	\vspace{-13em}

Now there would be two other moves to try: either \emph{(a)} \bmove{Rg6+}, or \emph{(b)} \bmove{Rf6}. Let us examine them.

\emph{(a)} \mainline{2... Rg6+, 3.Kf3 Rf6+, 4.Ke3 Re6+}

\end{wraptable}

if \variation{4... b3, 5.Rh5} wins, because if the King goes back, then \wmove{Rh4+}, followed by \wmove{Rf4} wins.

\mainline{5.Kd3 Rf6} If \variation{5... Rd6, 6.Ke4} wins.

\mainline{6.Rh5+ Kb6, 7.Rh6}

\restoregame{state2}

\clearpage

\emph{(b)} \mainline{2... Rf6, 3.Rg7! Kc4} If \variation{3... b3, 4.Rg3}, and White will either capture the Pawn or go to f3, and come out with a winning ending.

\mainline{4.h4 b3, 5.Rg4+ Kc3, 6.Rg3+} and White will either capture the Pawn or play R - K B 3, according to the circumstances, and come out with a winning ending.

Now, going back to the position first shown in this chapter, suppose that after \wmove{1.f6 Rd6!, 2.Rc7+}, Black did not realise that \bmove{Kd4} was the only move to draw, and consequently played \bmove{Kc6} instead, we would then have the following position:

\restoregame{state1}

\hidemoves{2.Rc7+ Kb6}

\chessboard[smallboard,
marginleft=false,
marginrightwidth=2em,
moverstyle=triangle]
\begin{wraptable}{r}{0.5\textwidth}
	\vspace{-13em}
	
Now the best coninuation would be:

\mainline{3.f7 Rg6+, 4.Kf1 Rf6, 5.Re7! Kc5} White threatened to check with the rook at e6.

\end{wraptable}

\mainline{6.Ke2 b3} Best. If \variation{6... Kc4}; both \wmove{h4} and \wmove{Ke3} will win; the last-named move particularly would win with ease.

\mainline{7.Re3 b2, 8.Rb3 Rxf7, 9.Rxb2 Rh7, 10.Rd2 Rxhx2, 11.Ke3}

\chessboard[smallboard,
marginleft=false,
marginrightwidth=2em,
moverstyle=triangle]
\begin{wraptable}{r}{0.5\textwidth}
	\vspace{-13em}

This position we have arrived at is won by White, because there are two files between the opposing King and the Pawn from which the King is cut off by the Rook, and besides, the Pawn can advance to the fourth rank before the opponent's Rook can begin to check on the file. 
	
\end{wraptable}

This last condition is very important, because if, instead of the position on the diagram, the Black Rook were at \bmove{Rh8}, and Black had the move, he could draw by preventing the advance of the Pawn, either through constant checks or by playing \bmove{Rf8} at the proper time.

Now that we have explained the reasons why this position is won, we leave it to the student to work out the correct solution.

The fact that out of one apparently simple ending we have been able to work out several most unusual and difficult endings should be sufficient to impress upon the student's mind the necessity of becoming well acquainted with all kinds of endings, and especially with endings of Rook and Pawns.

\clearpage

\section{A Difficult Ending: Two Rooks and Pawns}

Following our idea that the best way to learn endings as well as openings is to study the games of the masters, we give two more endings of two Rooks and Pawns. These endings, as already stated, are not very common, and the author is fortunate in having himself played more of these endings than is generally the case. By carefully comparing and studying the endings already given (Examples 56 and 57) with the following, the student no doubt can obtain an idea of the proper method to be followed in such cases. The way of procedure is somewhat similar in all of them.

\subsubsection*{Example 60}

\newgame
\hidemoves{1. d4 d5 2. Nf3 Nf6 3. Bf4 e6 4. e3 c5 5. c3 Nc6 6. Bd3 Bd6 7. Bxd6 Qxd6 8. Nbd2 e5 9. dxe5 Nxe5 10. Nxe5 Qxe5 11. Bb5+ Bd7 12. Qa4 Qc7 13. O-O-O O-O 14. Bxd7 Nxd7 15. Nf3 Qc6 16. Qxc6 bxc6 17. Nd2 Ne5 18. Kc2 c4 19. Rhf1 f5 20. Nf3 Nxf3 21. gxf3}
\chessboard[smallboard,
marginleft=false,
marginrightwidth=2em,
moverstyle=triangle]
\begin{wraptable}{r}{0.5\textwidth}
	\vspace{-13em}

From a game, Capablanca - Kreymborg, in the New York State Championship Tournament of 1910. 

It is Black's move, and no doubt thinking that drawing such a position (that was all Black played for) would be easy, he contented himself with a waiting policy. Such conduct must always be criticised. 

\end{wraptable}

It often leads to disaster. \emph{The best way to defend such positions is to assume the initiative and keep the opponent on the defensive.}

\mainline{21... Rae8} The first move is already wrong. There is nothing to gain by this move. Black should play \bmove{a5}; to be followed by \bmove{a6}; unless White plays \wmove{b3}. That would fix the Queen's side. After that he could decide what demonstration he could make with his Rooks to keep the opponent's Rooks at bay.

\mainline{22. Rd4} his move not only prevents \bmove{f4} which Black intended, but threatens \bmove{b3}, followed, after \bmove{cxb3+}, by the attack with one or both Rooks against Black's a Pawn.

\mainline{22... Rf6}probably with the idea of a demonstration on the King's side by \bmove{Rg6} and \bmove{Rg2}.

\mainline{23.b3 cxb3+, 24.axb3 Kf7, 25.Kd3} \wmove{Ra1} should have been played now, in order to force Black to defend with \bmove{Re7}. White, however, does not want to disclose his plan at once, and thus awaken Black to the danger of his position, hence this move, which seems to aim at the disruption of Black's Queen's side Pawns.

\mainline{25... Re7, 26.Ra1 Ke6} This is a mistake. Black is unaware of the danger of his position. He should have played \bmove{g5}; threatening \bmove{Rh6}, and, by making this demonstration against White's h Pawn, stop the attack against his Queen's side Pawns, which will now develop.

\mainline{27.Ra6 Rc7} He could not play \bmove{Kd6}, because \wmove{c4} would win at least a Pawn. This in itself condemns his last move \bmove{Ke6}, which has done nothing but make his situation practically hopeless.

\mainline{28.Rda4 g5} Now forced, but it is a little too late. He could not play \bmove{R6f7}, because \wmove{f4} would have left his game completely paralysed. Black now finally awakens to the danger, and tries to save the day by the counter-demonstration on the King's side, which he should have started before. Of course, White cannot play \wmove{Rxa7}, because of \bmove{Rxa7}, followed by \bmove{Rh6}, recovering the Pawn with advantage.

\mainline{29.h4! g4} Black is now in a very disagreeable position. If he played \variation{29... gxh4, 30.Rxh4} would leave him in a very awkward situation, as he could not go back with the King, nor could he do much with either Rook. He practically would have to play \bmove{h6}, when White would answer \wmove{31.b4}, threatening to win a Pawn by \wmove{b5}, or, if that were not enough, he might play \wmove{Kd4}, to be followed finally by the entry of the King at c5 or e5.

\mainline{30.Ke2} 

\chessboard[smallboard,
marginleft=false,
marginrightwidth=2em,
moverstyle=triangle]
\begin{wraptable}{r}{0.5\textwidth}
	\vspace{-13em}

\mainline{30... gxf3+} Again he cannot play \bmove{h5}, because \wmove{f4} would leave him paralysed. The advance of his h Pawn would make White's h Pawn safe, and consequently his f6 Rook would have to retire to f2 to defend the a Pawn. 

\end{wraptable}

That would make it impossible for his King to go to d2, because of the a Pawn, nor could he advance a single one of his Pawns. On the other hand, White would play \wmove{b4}, threatening to win a Pawn by \wmove{b5}, or he might first play \wmove{Kd4}, and then at the proper time \wmove{b5}, if there was nothing better. Black meanwhile could really do nothing but mark time with one of his Rooks. Compare this bottling-up system with the ending in Example 57, and it will be seen that it is very similar.

\mainline{31.Kxf3 Rff7, 32.Ke2} Probably wrong. \wmove{b4} at once was the right move.\footnote{Computer analysis disagrees with this assement and suggests \wmove{Kf4} is the best move.} The next move gives Black good chances of drawing.

\mainline{32... Kd6, 33.b4 Rb7} This could never have happened had White played \wmove{12.b4}, as he could have followed it up by \wmove{b5} after Black's \bmove{Kd6}.

\mainline{34.h5} Not good. \wmove{f4} offered the best chances of winning by force. If then \variation{34.f4 Rg7, 35.h5 Rg2+, 36.Kd3 Rh2, 37.Rxa7 Rxa7, 38.Rxa7 Rxh5, 39.Ra6}, with winning chances.


\chessboard[smallboard,
marginleft=false,
marginrightwidth=2em,
moverstyle=triangle]
\begin{wraptable}{r}{0.5\textwidth}
	\vspace{-13em}

\mainline{34... h6} Black misses his last chance. \variation{34... f4, 35.exf4 Rbe7+!, 36.Kf1 Rxf4, 37.Rxa7 Re3!}

\end{wraptable}

\mainline{35.f4 Rg7, 36.Kd3 Rge7, 37.Ra1 Rg7, 38.Kd4 Rg2, 39.R6a2 Rbg7} \bmove{R2g2} would have offered greater resistance, but the position is lost in any case. (I leave the student to work this out.)

\mainline{40.Kd3! Rxa2 41.Rxa2 Re7} Nothing would avail. If \variation{41... Rg1, 42.Ra6 Rd1+, 43.Kc2 Rh1, 44.b5 Rxh5, 45.Rxc6+ Kd7, 46.Ra6}, and White will win easily.

\mainline{42.Rg2 Re6, 43.Rg7 Re7, 44.Rg8 c5} Black is desperate. He sees he can no longer defend his Pawns.

\mainline{45.Rg6+ Re6, 46.bxc5+ Kd7, 47.Rg7+ Kc6, 48.Rxa7 Kxc5, 49.Rf7} Resigns

\subsubsection*{Example 61}

\newgame
\hidemoves{1.e4 e5 2.Nf3 Nc6 3.Nc3 Nf6 4.Bb5 a6 5.Bxc6 dxc6 6.O-O Bg4 7.h3 Bh5 8.Qe2 Bd6 9.d3 Qe7 10.Nd1 O-O-O 11.Ne3 Bg6 12.Nh4 Rhg8 13.Nef5 Qe6 14.f4 Bxf5 15.Nxf5 exf4 16.Bxf4 Bc5+ 17.Be3 Bf8 18.Qf2 Rd7 19.Bc5 Bxc5 20.Qxc5 Kb8 21.Rf2 Ne8 22.Raf1 f6 23.b3 Nd6 24.Rf4 Nxf5 25.Qxf5 Qxf5 26.Rxf5 Re8}
\chessboard[smallboard,
marginleft=false,
marginrightwidth=2em,
moverstyle=triangle]
\begin{wraptable}{r}{0.5\textwidth}
	\vspace{-13em}

Black's game has the disadvantage of his double c Pawns, which, to make matters worse, he cannot advance, because as soon as Black plays \bmove{b6}, White replies \wmove{b4}. It is on this fact that White builds his plans. He will stop Black's Queen's side Pawns from advancing, and will then bring his own King to e3.

\end{wraptable}

Then in due time he will play \wmove{d4}, and finally \wmove{e5}, or \wmove{g5}, thus forcing an exchange of Pawns and obtaining in that way a clear passed Pawn on the King's file. It will be seen that this plan was carried out during the course of the game, and that White obtained his winning advantage in that way. The play was based throughout on the chance of obtaining a passed Pawn on the King's file, with which White expected to win.

\mainline{27.g4} already preparing to play \wmove{g5} when the time comes.

\mainline{27... b6} Black wants to play \bmove{c5}, but White, of course, prevents it.

\mainline{28.b4! Kb7} This King should come to the King's side, where the danger lurks.

\mainline{29.Kf2 b5} With the object of playing \bmove{Kb6} and \bmove{a5}, followed by \wmove{bxa2} or \bmove{axb4}, and thus have an open file for his Rook and be able to make a counter-demonstration on the Queen's side in order to stop White's advance on the right. White, however, also prevents this.


\mainline{30.a4! Rd4} Of course if \wmove{axb5}; Black will have all his Pawns on the Queen's side disrupted and isolated, and White can easily regain the lost Pawn by playing either Rook on the a file.

\mainline{31.Rb1 Re5} He still wants to play \bmove{c5}, but as it is easy to foresee that White will again prevent it, the next move is really a serious loss of time. Black should bring his King over to the other side immediately.

\mainline{32.Ke3 Rd7, 33.a5} The first part of White's strategic plan is now accomplished. Black's Pawns on the Queen's side are \emph{fixed} for all practical purposes.

\mainline{33... Re6} If \variation{33... Rxf5, 34.gxf5} would have given White a very powerful centre. Yet it might have been the best chance for Black.

\mainline{34.Rbf1 Rde7, 35.g5 fxg5, 36.Rxg5}

\chessboard[smallboard,
marginleft=false,
marginrightwidth=2em,
moverstyle=triangle]
\begin{wraptable}{r}{0.5\textwidth}
	\vspace{-13em}

The second part of White's strategical plan is now accomplished. It remains to find out if the advantage obtained is sufficient to win. White not only has a passed Pawn, but his King is in a commanding position in the centre of the board ready to support the advance of White's Pawns, or, if necessary, to go to c5,

\end{wraptable}

or to move to the right wing in case of danger. Besides, White holds the open file with one of his Rooks. Altogether White's position is superior and his chances of winning are excellent.

\mainline{36... Rh6, 37.Rg3 Rhe6} to prevent \wmove{d4}. Also Black fears to keep his Rook in front of his two King's side Pawns which he may want to utilise later.

\mainline{38.h4 g6, 39.Rg5 h6} White threatens \wmove{h5}, which would finally force Black to take, and then White would double his Rooks against the isolated Pawn and win it, or tie up Black's Rooks completely. The next move, however, only helps White; therefore Black had nothing better than to hold tight and wait. \bmove{Re5} would not help much, as White would simply answer \variation{39... Re5, 40.Rf8 Re8, 41.Rxe5}, and whichever Rook Black took, White would have an easy game. (The student should carefully study these variations.)

\mainline{40.Rg4 Rg7, 41.d4 Kb8, 42.Rf8+ Kb7} \wmove{Kd2} would not help much, but since he made the previous move he should now be consistent and play it.

\mainline{43.e5 g5, 44.Ke4 Ree7, 45.hxg5 hxg5, 46.Rf5 Kc8, 47.Rgxg5 Rh7, 48.Rh5 Kd7, 49.Rxh7 Rxh7, 50.Rf8 Rh4+, 51.Kd3 Rh3+, 52.Kd2 c5, 53.bxc5 Ra3, 54.d5} Resigns.
        
The winning tactics in all these endings have merely consisted in keeping the opponent's Rooks tied to the defence of one or more Pawns, leaving my own Rooks free for action. This is a general principle which can be equally applied to any part of the game. It means in general terms—

\emph{Keep freedom of manœuvre while hampering your opponent.}

There is one more thing of great importance, and that is that the winning side has always had a general strategical plan capable of being carried out with the means at his disposal, while often the losing side had no plan at all, but simply moved according to the needs of the moment.

\clearpage

\section{Rook, Bishop and Pawns v. Rook, Knight and Pawns}

We shall now examine an ending of Rook, Bishop and Pawns against Rook, Knight and Pawns, where it will be seen that the Rook at times is used in the same way as in the endings already given.

\subsubsection*{Example 62}
From the first game of the Lasker-Marshall Championship Match in 1907.

\newgame
\hidemoves{1.e4 e5 2.Nf3 Nc6 3.Bb5 Nf6 4.d4 exd4 5.O-O Be7 6.e5 Ne4
7.Nxd4 O-O 8.Nf5 d5 9.Bxc6 bxc6 10.Nxe7+ Qxe7 11.Re1 Qh4
12.Be3 f6 13.f3 fxe5 14.fxe4 d4 15.g3 Qf6 16.Bxd4 exd4 17.Rf1
Qxf1+ 18.Qxf1 Rxf1+ 19.Kxf1}

\chessboard[smallboard,
marginleft=false,
marginrightwidth=2em,
moverstyle=triangle]
\begin{wraptable}{r}{0.5\textwidth}
	\vspace{-13em}

In this position it is Black's move. To a beginner the position may look like a draw, but the advanced player will realise immediately that there are great possibilities for Black to win, not only because he has the initiative, but because of White's undeveloped Queen's side and the fact that a

\end{wraptable}

Bishop in such a position is better than a Knight. It will take some time for White to bring his Rook and Knight into the fray, and Black can utilise it to obtain an advantage. There are two courses open to him. The most evident, and the one that most players would take, is to advance the Pawn to \bmove{c5} and \bmove{c6} immediately in conjunction with the Bishop check at a6 and any other move that might be necessary with the Black Rook. The other, and more subtle, course was taken by Black. It consists in utilising his Rook in the same way as shown in the previous endings, forcing White to defend something all the time, restricting the action of White's Knight and White's Rook, while at the same time keeping freedom of action for his own Rook and Bishop.

\mainline{19... Rb8} This forces \wmove{B3}, which blocks that square for the White Knight.

\mainline{20.b3 Rb5} bringing the Rook to attack the King's side Pawns so as to force the King to that side to defend them, and thus indirectly making more secure the position of Black's Queen's side Pawns.

\mainline{21.c4 Rh5, 22.Kg1 c5} Note that the White Knight's sphere of action is very limited, and that after \wmove{Nd2} White's own Pawns are in his way.

\mainline{23.Nd2 Kf7, 24.Rf1+} This check accomplishes nothing. It merely drives Black's King where it wants to go. Consequently it is a very bad move. \wmove{a3}\footnote{Computer analysis suggests \wmove{Nf3} was the best move.} at once was the best move.

\mainline{24... Ke7, 25.a3 Rh6} Getting ready to shift the attack to the Queen's side, where he has the advantage in material and position.

\mainline{26.h4 Ra6,} Notice how similar are the manœuvres with this Rook to those seen in the previous endings.

\mainline{27.Ra1 Bg4} Paralysing the action of the Knight and \emph{fixing} the whole King's side.

\mainline{28.Kf2 Ke6} White cannot answer \wmove{Nf3}, because \bmove{Bxf3} followed by \bmove{Ke5} will win a Pawn, on account of the check at \bmove{Rf6+} which cannot be stopped.

\mainline{29.a4 Ke5, 30.Kg2 Rf6, 31.Re1 d3, 32.Rf1 Kd4} Now the King attacks White's Pawns and all will soon be over.

\mainline{33.Rxf6 gxf6, 34.Kf2 c6} Merely to exhaust White's move, which will finally force him to move either the King or the Knight.

\mainline{35.a5 a6, 36.Nb1 Kxe4, 37.Ke1 Be2, 38.Nd2+ Ke3, 39.Nb1 f5, 40.Nd2 h5, 41.Nb1 Kf3, 42.Nc3 Kxg3, 43.Na4 f4, 44.Nxc5 f3, 45.Ne4+ Kf4} The quickest way to win. White should resign.

\mainline{46.Nd6 c5, 47.b4 cxb4, 48.c5 b3, 49.Nc4 Kg3, 50.Ne3 b2} Resigns.

A very good example on Black's part of how to conduct such an ending.

\begin{center}
\chessboard[largeboard,
moverstyle=triangle]
\end{center}

\chapter{Further Openings and Middle-Games}

\section{Some Salient Points About Pawns}

Before going back to the discussion of openings and middle-game positions, it might be well to bear in mind a few facts concerning Pawn positions which will no doubt help to understand certain moves, and sometimes even the object of certain variations in the openings, and of some manœuvres in the middle-games.

\newgame
\fenboard{8/2p2p1p/p3p1p1/1p1pP3/1P1P4/1P6/P4PPP/8 w - - 0 1}
\chessboard[smallboard,
marginleft=false,
marginrightwidth=2em,
moverstyle=triangle]
\begin{wraptable}{r}{0.5\textwidth}
	\vspace{-13em}

In the position of the diagram we have an exceedingly bad Pawn formation on Black's side. Black's c Pawn is altogether backward, and White could by means of the open file concentrate their forces against that weak point. 

\end{wraptable}

There is also the square at White's c5, which is controlled by White, and from where a White piece once established could not be dislodged. In order to get rid of it, Black would have to exchange it, which is not always an easy matter, and often when possible not at all convenient. The same holds true with regard to Black's e, f and g Pawns, which create what is called a "hole" at f6. Such Pawn formations invariably lead to disaster, and consequently must be avoided.

\subsubsection*{Example 64}

\newgame
\fenboard{8/ppp2ppp/4p3/3pP3/3P4/8/PPP2PPP/8 w - - 0 1}
\chessboard[smallboard,
marginleft=false,
marginrightwidth=2em,
moverstyle=triangle]
\begin{wraptable}{r}{0.5\textwidth}
	\vspace{-13em}

In this position we might say that the White centre Pawns have the attacking position, while the Black centre Pawns have the defensive position. 

\end{wraptable}

Such a formation of Pawn occurs in the French Defence. In such positions White most often attempts, by means of \wmove{f4} and wmove{f5}, to obtain a crushing attack against Black's King, which is generally Castled on the King's side. To prevent that, and also to assume the initiative or obtain material advantage, Black makes a counter-demonstration by \bmove{c5}, followed by \wmove{dxc5} (when White defends the Pawn by \wmove{c3}), and the concentrating of Black's pieces against the White Pawn at d4. This in substance might be said to be a determined attack against White's centre in order to paralyse the direct attack of White against Black's King. It must be remembered that at the beginning of the book it was stated that control of the centre was an essential condition to a successful attack against the King.

In an abstract way we may say that two or more Pawns are strongest when they are in the same rank next to one another. Thus the centre Pawns are strongest in themselves, so to speak, when placed at e4 and d4 respectively, hence the question of advancing either the one or the other to the fifth rank is one that must be most carefully considered. The advance of either Pawn often determines the course the game will follow.

Another thing to be considered is the matter of one or more passed Pawns when they are isolated either singly or in pairs. We might say that a passed Pawn is either very weak or very strong, and that its weakness or strength, whichever happens to be in the case to be considered, increases as it advances, and is at the same time in direct relation to the number of pieces on the board. In this last respect it might be generally said that a passed Pawn increases in strength as the number of pieces on the board diminishes.

Having all this clear in mind we will now revert to the openings and middle-game. We will analyse games carefully from beginning to end according to general principles. I shall, whenever possible, use my own games, not because they will better illustrate the point, but because, knowing them thoroughly, I shall be able to explain them more authoritatively than the games of others.

\section{Some Possible Developments from a Ruy Lopez}

That some of the variations in the openings and the manœuvres in the middle-game are often based on some of the elementary principles just expounded can be easily seen in the following case:

\subsubsection*{Example 65}

\newgame

\mainline{1.e4 e5, 2.Nf3 Nc6, 3.Bb5 a6, 4.Ba4 Nf6, 5.O-O Nxe4, 6.d4 b5, 7.Bb3 d5, 8.dxe5 Be6, 9.c3 Be7, 10.Re1 O-O, 11.Nbd2 Nc5, 12.Bc2 Bg4, 13.Nb3 Ne6} So far a very well-known variation of the Ruy Lopez. In fact, they are the moves of the Janowski-Lasker game in Paris, 1912.

\mainline{14.Qd3 g6}

\chessboard[smallboard,
marginleft=false,
marginrightwidth=2em,
moverstyle=triangle]
\begin{wraptable}{r}{0.5\textwidth}
	\vspace{-13em}

Let us suppose the game went on, and that in some way White, by playing one of the Knights to d4 at the proper time, forced the exchange of both Knights, and then afterwards both the Bishops were exchanged, and we arrived at some such position as shown in the following diagram. 

\end{wraptable}

(I obtained such a position in a very similar way once at Lodz in Poland. I was playing the White pieces against a consulting team headed by Salwe.)

\newgame
\fenboard{r3r1k1/2pq1p1p/p5p1/1p1pP3/3P4/3Q4/PP3PPP/2R1R1K1 w q - 0 1}
\chessboard[smallboard,
marginleft=false,
marginrightwidth=2em,
moverstyle=triangle]
\begin{wraptable}{r}{0.5\textwidth}
	\vspace{-13em}

Now we would have here the case of the backward c Pawn for Black, which will in no way be able to advance to d5. Such a position may be said to be theoretically lost, and in practice a first-class master will invariably win it from Black. (If I may be excused the reference, I will say that I won the game above referred to.)

\end{wraptable}

After a few moves the position may be easily thus: 

\hidemoves{1.Rc5 c6 2.Qa3 Re6 3.Rec1 Rc8 4.Qb4 Qb7 5.Qa3}

\chessboard[smallboard,
marginleft=false,
marginrightwidth=2em,
moverstyle=triangle]
\begin{wraptable}{r}{0.5\textwidth}
	\vspace{-13em}

The Black pieces can be said to be fixed. If White plays \wmove{Qc3}, Black must answer \bmove{Qd7}, otherwise they will lose a Pawn, and if White returns with the Queen to a3 Black will have again to return to b7 with the Queen or lose a Pawn.

\end{wraptable}

Thus Black can only move according to White's lead, and under such conditions White can easily advance with the Pawns to \wmove{f4} and \wmove{g4}, until Black will be forced to stop \wmove{f5} by playing \bmove{f5}, and we might finally have some such position as this:


\newgame
\fenboard{2r3k1/1q5p/p1p1r1p1/1pRpPp2/3P1PP1/Q6P/PP6/2R4K w - - 0 1}
\chessboard[smallboard,
marginleft=false,
marginrightwidth=2em,
moverstyle=triangle]
\begin{wraptable}{r}{0.5\textwidth}
	\vspace{-13em}

In this situation the game might go on as follows:

\mainline{1.gxf5 gxf5} White threatened to win a Pawn by \wmove{Qxa6}, and Black could not play \bmove{Rf8}, because \wmove{Rxc6} would also win a Pawn at least.

\end{wraptable}

\mainline{2.Qf3 Qd7, 3.R5c2 Rg6, 4.Rg2 Kh8, 5.Rcg1 Rcg8, 6.Qh5 Rxg2, 7.Rxg2 Rxg2, 8.Kxg2 Qg7+, 9.Kh2 Qg6, 10.Qxg6 hxg6, 11.b4} and White wins.

Now suppose that in the position in the preceding diagram it were Black's move, and he played \bmove{Rf8}. White would then simply defend his g Pawn by some move like \wmove{Qf3}, threatening \wmove{Rxc6}, and then he would bring his King up to g3, and when the time came, break through, as in the previous case. White might even be able to obtain the following position:

\newgame
\fenboard{3r3k/2q4p/p1p1r1p1/1pRpPp2/1P1P1PP1/2R3KP/P7/2Q5 b - - 1 1}

Black would now be forced to play \bmove{Rf8}, and White could then play \wmove{Qc2}, and follow it up with \wmove{Kf3}, and thus force Black to play \bmove{fxg3}, which would give White a greater advantage.

A careful examination of all these positions will reveal that, besides the advantage of freedom of manœuvre on White's part, the power of the Pawn at e5 is enormous, and that it is the commanding position of this Pawn, and the fact that it is free to advance, once all the pieces are exchanged, that constitute the pivot of all White's manœuvres.

I have purposely given positions without the moves which lead to them so that the student may become accustomed to build up in their own mind possible positions that may arise (out of any given situation). Thus he will learn to make strategical plans and be on their way to the master class. The student can derive enormous benefit by further practice of this kind.

\section{The Influence of a "Hole"}

The influence of a so-called "hole" in a game has already been illustrated in my game against Blanco (Example 51), where has been shown the influence exercised by the different pieces posted in the hole created at e5.

\subsubsection*{Example 67}

In order to further illustrate this point, I now give a game played in the Havana International Masters Tournament of 1913. (Queen's Gambit Declined.) White: D. Janowski. Black: A. Kupchick.

\newgame

\mainline{1.d4 d5, 2.c4 e6, 3.Nc3 Nf6, 4.Bg5 Be7, 5.e3 Nbd7, 6.Bd3 dxc4 7.Bxc4 Nb6} Of course the idea is to post a Knight at d5, but as it is the other Knight which will be posted there this manœuvre does not seem logical. The Knight at b6 does nothing except to prevent the development of his own b Pawn. The normal course \bmove{O-O}, followed by \bmove{c5}, is more reasonable. For a beautiful illustration of how to play White in that variation, see the Janowski-Rubinstein game of the St. Petersburg Tournament of 1914.

\mainline{8.Bd3} \wmove{Bb3} has some points in its favour in this position, the most important being the possibility of advancing the King's Pawn immediately after \variation{8.Bb3 Nfd5, 9.Bxe7 Qxe7}.

\mainline{8... Nfd5, 9.Bxe7 Qxe7, 10.Nf3} Had White's Bishop been at b3 he could now play \wmove{e4} as indicated in the previous note, a move which he cannot make in the present position, because of \bmove{Nf4} threatening, not only the g Pawn, but also \bmove{Nxd3+}. As White's King's Bishop should never be exchanged in this opening without a very good reason White therefore cannot play \wmove{e4}.

\mainline{10... O-O, 11.O-O Bd7, 12.Rc1}

\chessboard[smallboard,
marginleft=false,
marginrightwidth=2em,
moverstyle=triangle]
\begin{wraptable}{r}{0.5\textwidth}
	\vspace{-13em}

White is perfectly developed, and now threatens to win a Pawn as follows: \wmove{13.Nxd5 Nxd5, 14.e4 Nf4, 15Rxc7}.

\mainline{12... c6}

\end{wraptable}

The fact that Black is practically forced to make this move in order to avoid the loss of a Pawn is sufficient reason in itself to condemn the whole system of development on Black's part. In effect, he plays \bmove{Bd7} and now he has to shut off the action of his own Bishop, which thereby becomes little more than a Pawn for a while. In fact, it is hard to see how this Bishop will ever be able to attack anything. Besides, it can be easily seen that White will soon post his two Knights at e5 and c5 respectively, and that Black will not be able to dislodge them without seriously weakening his game, if he can do it at all. From all these reasons it can be gathered that it would probably have been better for Black to play \bmove{Nxc3} and thus get rid of one of the two White Knights before assuming such a defensive position. In such cases, the less the number of pieces on the board, the better chances there are to escape.

\mainline{13.Ne4 f5} This practically amounts to committing suicide, since it creates a hole at e5 for White's Knight, from where it will be practically impossible to dislodge him. If Black intended to make such a move he should have done it before, when at least there would have been an object in preventing the White Knight from reaching f4.

\mainline{14.Nc5 Be8, 15.Ne5} The position of White's Knights, especially the one at e5, might be said to be ideal, and a single glance shows how they dominate the position. The question henceforth will be how is White going to derive the full benefit from such an advantageous situation, This we shall soon see. 

\chessboard[smallboard,
marginleft=false,
marginrightwidth=2em,
moverstyle=triangle]
\begin{wraptable}{r}{0.5\textwidth}
	\vspace{-13em}

\mainline{15... Rb8} There is no object in this move, unless it is to be followed by \bmove{Nd7}. As that is not the case, he might have gone with the Rook to c1, as he does later.

\end{wraptable}

\mainline{16.Re1 Rf6, 17. Qf3 Rh6, 18.Qg3 Rc8} White threatened to win the exchange by playing either \wmove{Nf7} or \wmove{Ng4}.

\mainline{19.f3 Rc7, 20.a3 Kh8, 21.h3} Perhaps all these precautions are unnecessary, but White feels that he has more than enough time to prepare his attack, and wants to be secure in every way before he begins.

\mainline{21... g5, 22.e4 f4, 23.Qf2 Ne3} He had better have played \bmove{Nf6}; and tried later on to get rid of White's Knights by means of \bmove{Nd7}\footnote{Computer analysis disagrees with this assessment and suggest the move played \bmove{Ne3}}.

\chessboard[smallboard,
marginleft=false,
marginrightwidth=2em,
moverstyle=triangle]
\begin{wraptable}{r}{0.5\textwidth}
	\vspace{-13em}

\mainline{24.Rxe3} with this sacrifice of the Rook for a Knight and Pawn White obtains an overwhelming position.

\mainline{24... fxe3, 25.Qxe3 Nc8}

\end{wraptable}

\bmove{Nd7} was better in order to get rid of one of the two White Knights. There were, however, any number of good replies to it, among them the following: \variation{25... Nd7, 26.Ncxd7 Bxd7, 27.Qxg5 Qxg5, 28.Nf7+ Kg7, 29.Nxg5}, and with two Pawns for the exchange, and the position so much in his favour, White should have no trouble in winning. 

\mainline{26.Ng4 Rg6, 27.e5 Rg7, 28.Bc4 Bf7} All these moves are practically forced, and as it is easily seen they tie up Black's position more and more. White's manœuvres from move 24 onwards are highly instructive.

\mainline{29.Nf6 Nb6} This wandering Knight has done nothing throughout the game.

\mainline{30.Nce4 h6, 31.h4 Nd5, 32.Qd2 Rg6, 33.hxg5 Qf8} If \variation{33... hxg5, 34.Kf2}, and Black would be helpless.

\mainline{34.f4 Ne7, 35.g4 hxg5, 36.fxg5} Resigns.

There is nothing to be done. If \bmove{Bg8, Qh2+ Kg7, Bxe6}.

The student should notice that, apart from other things, White throughout the game has had control of the Black squares, principally those at e5 and c5.

\begin{center}
\chessboard[largeboard,
moverstyle=triangle]
\end{center}

From now on to the end of the book I shall give a collection of my games both lost and won, chosen so as to serve as illustrations of the general principles laid down in the foregoing pages.

\part{Illustrative Games}

\chapter{Game 1: Queen's Gambit Declined}

\newgame
\mainline{1.d4 d5, 2.c4 e6, 3.Nc3 Nf6, 4.Bg5 Be7, 5.e3 Ne4}

I had played this defence twice before in the match with good results, and although I lost this game I still played it until the very last game, when I changed my tactics. The reason was my total lack of knowledge of the different variations in this opening, coupled with the fact that I knew that Dr. E. Lasker had been successful with it against Marshall himself in 1907. I thought that since Dr. Lasker had played it so often, it should be good. The object is to exchange a couple of pieces and at the same time to bring about a position full of possibilities and with promising chances of success once the end-game stage is reached. On general principles it should be wrong, because the same Knight is moved three times in the opening, although it involves the exchange of two pieces. In reality the difficulty in this variation, as well as in nearly all the variations of the Queen's gambit, lies in the slow development of Black's Queen Bishop. However, whether this variation can or cannot be safely played is a question still to be decided, and it is outside the scope of this book. I may add that at present my preference is for a different system of development, but it is not unlikely that I should some time come back to this variation.

\mainline{6.Bxe7 Qxe7 7.Bd3} \variation{7.cxd5} is preferable for reasons that we shall soon see.

\mainline{7... Nxc3, 8.bxc3 Nd7} Now \bmove{dxc4} would be a better way to develop the game. The idea is that after \variation{8... dxc4, 9.Bxc4 b6}, followed by \bmove{Bb7}, would give Black's Bishop a powerful range. For this variation see the eleventh game of the match.

\mainline{9.Nf3 O-O} No longer would \variation{9... dxc4, 10.Bxc4 b6, 11.Bb5} be good, because \wmove{Bb5} would prevent \bmove{Bb7} on account of \wmove{Ne5}.

\mainline{10.cxd5 exd5, 11.Qb3 Nf6, 12.a4 c5} Played with the intention of obtaining the majority of Pawns on the Queen's side. Yet it is doubtful whether this move is good, since it leaves Black's Queen's-side Pawns disrupted in a way. The safer course would have been to play \bmove{c5}.

\mainline{13.Qa3 b6}

\chessboard[smallboard,
marginleft=false,
marginrightwidth=2em,
moverstyle=triangle]
\begin{wraptable}{r}{0.5\textwidth}
	\vspace{-13em}

This exposes Black to further attack by \wmove{a5} without any compensation for it. If I had to play this position nowadays I would simply play \variation{13... Rb8, 14.Qxc5 Qxc5, 15.dxc5}, and I believe that Black would regain the Pawn. If, instead, White played \wmove{14.dxc5} then \bmove{Bg4} would give Black an excellent game.

\end{wraptable}

\mainline{14.a5 Bb7, 15.O-O Qc7, 16.Rfb1 Nd7}

\chessboard[smallboard,
marginleft=false,
marginrightwidth=2em,
moverstyle=triangle]
\begin{wraptable}{r}{0.5\textwidth}
	\vspace{-13em}

Black's position was bad and perhaps lost in any case, but the next move makes matters worse. As a matter of fact I never saw White's reply \wmove{Bf5}. It never even passed through my mind that this was threatened. Black's best move would have been \variation{16... Rfb8}. If that loses, then any other move would lose as well.

\end{wraptable}

\mainline{17.Bf5 Rfc8} From bad to worse. \bmove{Kf6} offered the only hope.

\mainline{18.Bxd7 Qxd7, 19.a6 Bc6, 20.dxc5 bxc5, 21.Qxc5 Rab8} The game was lost. One move was as good as another.

\mainline{22.Rxb8 Rxb8, 23.Ne5 Qf5, 24.f4 Rb6, 25.Qxb6!} Resigns.
  
Of course, if \variation{25.Nxc6 Rb1+} would have drawn. the next move is pretty and finishes quickly. A well-played game on Marshall's part.

\begin{center}
\chessboard[normalboard,
moverstyle=triangle]
\end{center}

\chapter{Game 2: Queen's Gambit Declined}

A.K. Rubinstein v J.R. Capablanca (San Sebastian, 1911)

\newgame

\mainline{1.d4 d5, 2.Nf3 c5, 3.c4 e6, 4.cxd5 exd5, 5.Nc3 Nc6, 6.g3 Be6} \bmove{Nf6} is the normal move in this variation. White's development was first introduced by Schlechter and elaborated later on by Rubinstein. It aims at the isolation of Black's d Pawn, against which the White pieces are gradually concentrated. In making the next move I was trying to avoid the beaten track. Being a developing move there should be no objection to it in the way of general principles, except that the Knights ought to come out before the Bishops.

\mainline{7.Bg2 Be7, 8.O-O Rc8} In pursuance of the idea of changing the normal course of this variation, but with very poor success. The move in theory ought to be unsound, since Black's g1 Knight is yet undeveloped. I had not yet learned of the attack founded on \bmove{Ng4} and the exchange of the Bishop at e3. Either \bmove{Nf6} or \bmove{h6}; to prevent either Bishop or Knight to e5, was right.


\chessboard[smallboard,
marginleft=false,
marginrightwidth=2em,
moverstyle=triangle]
\begin{wraptable}{r}{0.5\textwidth}
	\vspace{-13em}

\mainline{9.dxc5 Bxc5, 10.Ng5 Nf6, 11.Nxe6 fxe6, 12.Bh3 Qe7, 13.Bg5 O-O}

\end{wraptable}

This is a mistake. The right move was \bmove{Rd1} in order to get the Rook away from the line of the Bishop at h3 and at the same time to support the d Pawn. Incidentally it shows that White failed to take proper advantage of Black's weak opening moves. Against the next move White makes a very fine combination which I had seen, but which I thought could be defeated.

\mainline{14.Bxf6 Qxf6} I considered \bmove{gxf6}, which it seemed would give me a playable game, but I thought White's combination unsound and therefore let him play it, to my lasting regret.

\begin{center}
\chessboard[normalboard,
moverstyle=triangle]
\end{center}

\mainline{15.Nxd5! Qh6}

\chessboard[smallboard,
marginleft=false,
marginrightwidth=2em,
moverstyle=triangle]
\begin{wraptable}{r}{0.5\textwidth}
	\vspace{-13em}

\mainline{16.Kg2!} This is the move which I had not considered. I thought that Rubinstein would have to play \wmove{Bg2}, when I had in mind the following winning combination: 

\end{wraptable}

\variation{16.Bg2 Ne5!, 17.Nf4 Ng4, 18.h3 Nxf2, 19.Rxf2 Bxf2+, 20.Kxf2 g5}, and Black should win.

It is curious that this combination has been overlooked. It has been taken for granted that I did not see the 17th move \wmove{Qc1}. There are further variations within this line like \wmove{17.Rc1 Qxc1!!, 18.Qxc1 Bxf2+} and \wmove{18.Nh3 Bxf2+}

\mainline{16... Rcd8} After White's last move there was nothing for me to do but submit to the inevitable.

\mainline{17.Qc1! exd5, 18.Qxc5 Qd2, 19.Qb5 Nd4, 20.Qd3 Qxd3, 21.exd3 Rfe8, 22.Bg4} This gives Black a chance. He should have played \variation{22.Rae1 Nf3, 23.Rxe8+ Rxe8, 24.Rc1 Re3, 25.Kf1 Nd4, 26.Rc8+ Kf7, 27.Rc7+ Re7, 28.Rc5} wins. With a further variation within this of \bmove{25... Rxd3, 26.Be6+ Kf8, 27.Bxd5}.

\mainline{22... Rd6, 23.Rfe1 Rxe1, 24.Rxe1 Rb6, 25.Re5 Rxb2, 26.Rxd5 Nc6, 27.Be6+ Kf8, 28.Rf5+ Ke8, 29.Bf7+ Kd7, 30.Bc4}

\chessboard[smallboard,
marginleft=false,
marginrightwidth=2em,
moverstyle=triangle]
\begin{wraptable}{r}{0.5\textwidth}
	\vspace{-13em}

\mainline{30... a6} A bad move, which gives away any legitimate chance Black had to draw. It loses a very important move. In fact, as the course of the game will show, it loses several moves. The proper way was to play \variation{30... Kd6, 31.Rb5 Rxb5, 32.Bxb5 Nd4}; 

\end{wraptable}

followed by \bmove{b5}; and White would have an exceedingly difficult game to draw on account of the dominating position of the Knight at d5 in conjunction with the extra Pawn on the Queen's side and the awkward position of White's King. (See how this is so.)

\mainline{31.Rf7+ Kd6, 32.Rxg7 b5, 33.Bg8 a5, 34.Rxh7 a4, 35.h4 b4, 36.Rh6+ Kc5, 37.Rh5+ Kb6, 38.Bd5} With these last three moves White again gives Black a chance. Even before the last move \wmove{Bc4} would have won with comparative ease, but the next move is a downright blunder, of which, fortunately for him, Black does not avail himself.

\chessboard[smallboard,
marginleft=false,
marginrightwidth=2em,
moverstyle=triangle]
\begin{wraptable}{r}{0.5\textwidth}
	\vspace{-13em}

\mainline{38... b3} \bmove{Rxa2} would make it practically impossible for White to win, if he can win at all. White's best continuation then would have been: \variation{38... Rxa2, 39.Bc4 Rc2, 40.Rb5+ Kc7, 41.Bd5 a3, 42.h5 a2, 43.Bxa2 Rxa2, 44.h6 Ra6!} offers excellent chances for a draw.

\end{wraptable}

\mainline{39.axb3 a3, 40.Bxc6 Rxb3} If \variation{40... a2, 41.Rb5+ Ka6, 42.Rb8}.

\mainline{41.Bd5 a2, 42.Rh6+} Resigns.

As an end game, this is rather a sad exhibition for two masters. The redeeming feature of the game is Rubinstein's fine combination in the middle game, beginning with \wmove{14.Bxf6}.


\begin{center}
\chessboard[largeboard,
moverstyle=triangle]
\end{center}

\chapter{Game 3: Irregular Defence}
Havana, 1913 D. Janowski v J. Capablanca
\newgame

\mainline{1.d4 Nf6, 2.Nf3 d6, 3.Bg5 Nbd7, 4.e3 e5, 5.Nc3 c6, 6.Bd3 Be7 7.Qe2 Qa5, 8.O-O Nf8, 9.Rfd1 Bg4}

At last Black is on his way to obtain full development. The idea of this irregular opening is mainly to throw White on his own resources. At the time the game was played, the system of defence was not as well known as the regular forms of the Queen's Pawn openings. Whether it is sound or not remains yet to be proved. Its good features are that it keeps the centre intact without creating any particular weakness, and that it gives plenty of opportunity for deep and concealed manœuvring. The drawback is the long time it takes Black to develop his game. It is natural to suppose that White will employ that time to prepare a well-conceived attack, or that he will use the advantage of his development actually to prevent Black's complete development, or failing that, to obtain some definite material advantage.

\mainline{10.h3 Bh5, 11.dxe5 dxe5, 12.Ne4}

\chessboard[smallboard,
marginleft=false,
marginrightwidth=2em,
moverstyle=triangle]
\begin{wraptable}{r}{0.5\textwidth}
	\vspace{-12em}

\mainline{12... Nxe4} A very serious mistake. I considered castling, which was the right move, but desisted because I was afraid that by playing \variation{12... O-O-O, 13.Bxf6 gxf6, 14.Ng3 Bg6, 15.Nf5}, White would obtain a winning position for the end game. 

\end{wraptable}

Whether right or wrong this shows how closely related are all parts of the game, and consequently how one will influence the other.

\mainline{13.Bxe7 Kxe7, 14.Bxe4 Bg6} Not good. The natural and proper move would have been \bmove{Ne6}, in order to bring all the Black pieces into play. \bmove{Bxf3} at once was also good, as it would have relieved the pressure against Black's King's Pawn, and at the same time have simplified the game.

Here it is seen how failure to comply with the elementary logical reasons, that govern any given position, often brings the player into trouble. I was no doubt influenced in my choice of moves by the fear of \wmove{Bf5}, which was a very threatening move.

\mainline{15.Qc4 Ne6, 16.b4 Qc7, 17.Bxg6 hxg6, 18.Qe4 Kf6}

\chessboard[smallboard,
marginleft=false,
marginrightwidth=2em,
moverstyle=triangle]
\begin{wraptable}{r}{0.5\textwidth}
	\vspace{-13em}

\mainline{19. Rd3} \wmove{h4}, to be followed by \wmove{g4}, might have been a more vigorous way to carry on the attack. Black's weak point is unquestionably the Pawn at K 4, 

\end{wraptable}

which he is compelled to defend with the King. the next move aims at doubling the Rooks, with the ultimate object of placing one of them at d6, supported by a Pawn at c5, Black could only stop this by playing \bmove{c5} which would create a "hole" at d4; or by playing \bmove{b6}, which would tie the Black Queen to the defence of the c Pawn as well as the e Pawn, which she already defends. Black, however, can meet all this by offering the exchange of Rooks, which destroys White's plans. For this reason \wmove{h4} appears the proper way to carry on the attack.

\mainline{19... Rad8, 20.Rad1 g5} This move is preparatory to \bmove{g6}, which would make Black's position secure. Unfortunately for Black, he did not carry out his original plan.

\mainline{21.c4 Rxd3} \bmove{g6} would have left Black with a perfectly safe game.

\mainline{22.Rxd3 Rd8} A very serious mistake, which loses a Pawn. \bmove{g6} was the right move, and would have left Black with a very good game. In fact, if it should come to a simple ending, the position of the Black King would be an advantage.

\mainline{23.Rxd8 Nxd8}

\chessboard[smallboard,
marginleft=false,
marginrightwidth=2em,
moverstyle=triangle]
\begin{wraptable}{r}{0.5\textwidth}
	\vspace{-13em}

\mainline{24. h4} This wins a Pawn, as will soon be seen. Black cannot reply \bmove{Ne6}; because \wmove{25.hxg5+ Nxg5, 26.Qh4} wins the Knight. 

\end{wraptable}

\mainline{24... gxh4, 25.Qxh4+ Ke6, 26.Qg4+ Kf6, 27.Qg5+ Ke6, 28.Qxg7 Qd6 29.c5 Qd5, 30.e4! Qd1+, 31.Kh2 f6, 32.Qg4+! Ke7, 33.Nxe5 Qxg4, 34.Nxg4 Ne6, 35.e5 fxe5, 36.Nxe5 Nd4} The game went on for a few more moves, and, there being no way to counteract the advance of White's two passed Pawns, Black resigned.

\mainline{37.g4 Ke6, 38.f4 a5, 39.bxa5 Kd5, 40.g5 Kxc5, 41.g6 Nf5, 42.Kh3 Kd5, 43.Kg4 Ng7, 44.Kg5 c5, 45.Nd7 c4, 46.Nb6+ Kd4, 47.Nxc4 Kxc4, 48.f5 Kd5, 49.f6 Ne6+, 50.Kh6} Resigns.

\chapter{Game 4: French Defence}

St. Petersburg, 1913 J. Capablanca v E. Znosko-Borovski
\newgame

\mainline{1.d4 e6, 2.e4 d5, 3.Nc3 Nf6, 4.Bg5 Bb4} This constitutes the \emph{McCutcheon Variation.} It aims at taking the initiative away from White. Instead of defending, Black makes a counter demonstration on the Queen's side. It leads to highly interesting games.

\mainline{5.exd5} At the time this game was played the variation \variation{5.e5} was in vogue, but I considered then, as I do now, the next move to be the stronger.

\mainline{5... Qxd5} This is considered superior to \bmove{exd5}. It has for its object, as I said before, to take the initiative away from White by disrupting White's Queen's side. White, however, has more than ample compensation through his breaking up Black's King's side. It might be laid down as a principle of the opening that the \emph{breaking up of the King's side is of more importance than a similar occurrence on the Queen's side.}

\mainline{6.Bxf6 gxf6, 7.Nf3 Bxc3+, 8.bxc3 b6} The plan of Black in this variation is to post his Bishop on the long diagonal so as to be able later on, in conjunction with the action of his Rooks along the open g file, to make a violent attack against White's King. It is, of course, expected that White will Castle on the King's side because of the broken-up condition of his Queen's side Pawns.

\mainline{9.Be2 Bb7, 10.Qd2 Nd7, 11.c4 Qf5, 12.O-O-O} An original idea, I believe, played for the first time in a similar position in a game against Mr. Walter Penn Shipley, of Philadelphia. My idea is that as there is no Black Bishop and because Black's pieces have been developed with a view to an attack on the King's side, it will be impossible for Black to take advantage of the apparently unprotected position of White's King. Two possibilities must be considered. Firstly: If Black Castles on the Queen's side, as in this game, it is evident that there is no danger of an attack. Secondly: If Black Castles on the King's side, White begins the attack first, taking advantage of the awkward position of Black's Queen. In addition to the attacking probabilities of the next move, White in one move brings his King into safety and brings one of his Rooks into play. Thus he gains several moves, "tempi" as they are called, which will serve him to develop whatever plan he may wish to evolve.

\mainline{12... O-O-O, 13.Qe3 Rhg8, 14.g3 Qa5} Unquestionably a mistake, overlooking White's fine reply, but a careful examination will show that White already has the better position.

\mainline{15.Rd3 Kb8, 16.Rhd1 Qf5}

\chessboard[smallboard,
marginleft=false,
marginrightwidth=2em,
moverstyle=triangle]
\begin{wraptable}{r}{0.5\textwidth}
	\vspace{-13em}

\mainline{17.Nh4} This move has been criticised because it puts the Knight out of the way for a few moves. But by forcing \bmove{Qg5}; White gains a very important move with \wmove{f4}, which not only consolidates his position, but also drives the Queen away, 

\end{wraptable}

putting it out of the game for the moment. Certainly the Queen is far more valuable than the Knight, to say nothing of the time gained and the freedom of action obtained thereby for White's more important pieces.

\mainline{17... Qg5, 18.f4 Qg7, 19.Bf3} In such positions it is generally very advantageous to get rid of the Black Bishop controlling a6 and c6, which form "holes" for White's pieces. The Bishop in such positions is of very great defensive value, hence the advantage of getting rid of it.

\mainline{19... Rge8, 20.Bxb7 Kxb7, 21.c5 c6} White threatened \wmove{c6+}.

\mainline{22.Nf3 Qf8} To prevent the Knight from moving to Q 6 via Q 2 and K 4 or Q B 4. It is self-evident that White has a great advantage of position.

\chessboard[smallboard,
marginleft=false,
marginrightwidth=2em,
moverstyle=triangle]
\begin{wraptable}{r}{0.5\textwidth}
	\vspace{-13em}

\mainline{23.Nd2?} I had considered \wmove{Rb3}, which was the right move, but gave it up because it seemed too slow, and that in such a position there had to be some quicker way of winning.

\end{wraptable}

\mainline{23... bxc5, 24.Nc4} \bmove{Ne5} or \bmove{Nb6} would have brought about an ending advantageous to White.

\mainline{24... Nb6, 25.Na5+ Ka8, 26.dxc5 Nd5, 27.Qd4 Rc8} If \variation{27... Rb8, 28.Nxc6 Rbc8, 29.Nxa7} would win.

\chessboard[smallboard,
marginleft=false,
marginrightwidth=2em,
moverstyle=triangle]
\begin{wraptable}{r}{0.5\textwidth}
	\vspace{-13em}

\mainline{28.c4} \bmove{Nc4} was the right move. I was, however, still looking for the "grand combination," and thought that the Pawn I would later on have at  d6 would win the game. 

\end{wraptable}

Black deserves great credit for the way in which he conducted this exceedingly difficult defence. He could easily have gone wrong any number of times, but from move 22 onwards he always played the best move.

\mainline{28... e5, 29.Qg1 e4, 30.cxd5 exd3, 31.d6 Re2, 32.d7 Rc2+, 33.Kb1 Rb8+, 34.Nb3 Qe7}

\chessboard[smallboard,
marginleft=false,
marginrightwidth=2em,
moverstyle=triangle]
\begin{wraptable}{r}{0.5\textwidth}
	\vspace{-13em}

\mainline{35.Rxd3} The position is most interesting. I believe I lost here my last chance to win the game, and if that is true it would vindicate my judgment when, on move 28, I played \wmove{c4}. The student can find out what would happen if White plays \wmove{Qc4!} at once.

\end{wraptable}
      
I have gone over the following variation: \variation{35.Qd4 Rxh2, 36.Qxd3! Rd8, 37.Qa6 Kb8, 38.Qxc6} and White will at least have a draw.    

\mainline{35... Re2, 36.Qd4 Rd8, 37.Qa4 Qe4, 38.Qa6 Kb8} There is nothing to be done against this simple move, since White cannot play \wmove{Nd4}, because \bmove{Qh1#} mates.

\mainline{39.Kc1 Rxd7, 40.Nd4 Re1+, 41.Kd2 Rxd4} Resigns. \footnote{There is a Mate in 6 on the board. \variation{41...Rxd4, 42.a3 Re3, 43.Kc1 Qh1+, 44.Kb2 Re2+, 45.Rd2 Rexd2+, 46.Kc3 Qf3+, 47.Qd3 Qxd3#}.}

A very interesting battle.


\begin{center}
\chessboard[largeboard,
moverstyle=triangle]
\end{center}

\chapter{Game 5: Ruy Lopez}
St. Petersburg, 1914 E. Lasker v J. Capablanca

\newgame
\mainline{1.e4 e5, 2.Nf3 Nc6, 3.Bb5 a6, 4.Bxc6} The object of this move is to bring about speedily a middle-game without Queens, in which White has four Pawns to three on the King's side, while Black's superiority of Pawns on the other side is somewhat balanced by the fact that one of Black's Pawns is doubled. On the other hand, Black has the advantage of remaining with two Bishops while White has only one.

\mainline{4... dxc6, 5.d4 exd4, 6.Qxd4 Qxd4, 7.Nxd4 Bd6} Black's idea is to Castle on the King's side. His reason is that the King ought to remain on the weaker side to oppose later the advance of White's Pawns. Theoretically there is very much to be said in favour of this reasoning, but whether in practice that would be the best system would be rather difficult to prove. The student should notice that if now all the pieces were exchanged White would practically be a Pawn ahead, and would therefore have a won ending.

\mainline{8.Nc3 Ne7} A perfectly sound form of development. In any other form adopted the Black Knight could not be developed either as quickly or as well. e7 is the natural position for the Black Knight in this variation, in order not to obstruct Black's Pawns, and also, in some eventualities, in order to go to \bmove{Ng6}. There is also the possibility of its going to \bmove{Nd4} via \bmove{Nc6} after \bmove{c5}.

\mainline{9.O-O O-O, 10.f4} This move I considered weak at the time, and I do still. It leaves the K P weak, unless it advances to \wmove{e5}, and it also makes it possible for Black to pin the Knight by \bmove{Bc5}.

\mainline{10... Re8} Best. It threatens \wmove{11.Be3 Nd5, 12.exd5 Rxe3}. It also prevents \wmove{Be3} because of \bmove{Nd5} or \bmove{Nf5}

\mainline{11.Nb3 f6} Preparatory to \bmove{b3}, followed by \bmove{c5} and \bmove{Bb2} in conjunction with \bmove{Ng6}, which would put White in great difficulties to meet the combined attack against the two centre Pawns.

\mainline{12.f5}

\chessboard[smallboard,
marginleft=false,
marginrightwidth=2em,
moverstyle=triangle]
\begin{wraptable}{r}{0.5\textwidth}
	\vspace{-13em}

It has been wrongly claimed that this wins the game, but I would like nothing better than to have such a position again. It required several mistakes on my part finally to obtain a lost position.

\end{wraptable}

\mainline{12... b6, 13.Bf4}

\clearpage

\chessboard[smallboard,
marginleft=false,
marginrightwidth=2em,
moverstyle=triangle]
\begin{wraptable}{r}{0.5\textwidth}
	\vspace{-13em}

\mainline{13... Bb7} Played against my better judgment. The right move of course was \bmove{Bxf4}. Dr. Lasker gives the following variation: \variation{13... Bxf4, 14.Rxf4 c5, 15.Rd1 Bb7, 16.Rf2 Rad8, 17.Rxd8 Rxd8, 18.Rd2 Rxd2, 19.Nxd2}

\end{wraptable}

and he claims that White has the best of it. But, as Niemzovitch pointed out immediately after the game, \wmove{16... Rad8} given in Dr. Lasker's variation, is not the best. If \wmove{16... Rac8!} then White will have great difficulty in drawing the game, since there is no good way to stop Black from playing \bmove{Nc6}, followed by \bmove{Ne5}, threatening \bmove{Ng4}. And should White attempt to meet this manœuvre by withdrawing the Knight at b3; then the Black Knight can go to d4, and the White Pawn at e4 will be the object of the attack. Taking Dr. Lasker's variation, however, whatever advantage there might be disappears at once if Black plays \wmove{19... Nc6}, threatening \wmove{Nb5} and also \wmove{Nd5}, neither of which can be stopped. If White answers \wmove{20.Nd5 Nd4} for Black will at least draw. In fact, after \wmove{19... Nc6} Black threatens so many things that it is difficult to see how White can prevent the loss of one or more Pawns.

\mainline{14.Bxd6 cxd6, 15.Nd4} It is a curious but true fact that I did not see this move when I played \wmove{13... Bb7}, otherwise I would have played the right move \wmove{13... Bxf4}.

\mainline{15... Rad8} The game is yet far from lost, as against the entry of the Knight, Black can later on play \bmove{c5}, followed by \bmove{e5}.

\mainline{16.Ne6 Rd7, 17.Rad1}

\chessboard[smallboard,
marginleft=false,
marginrightwidth=2em,
moverstyle=triangle]
\begin{wraptable}{r}{0.5\textwidth}
	\vspace{-13em}

I now was on the point of playing \bmove{c5} to be followed by \bmove{e5}, which I thought would give me a draw, but suddenly I became ambitious and thought that I could play the next move, \wmove{17... Nc8}, and later on sacrifice the exchange for the Knight at e6, winning a Pawn for it, and leaving White's e Pawn still weaker. 

\end{wraptable}

I intended to carry this plan either before or after playing \bmove{b5} as the circumstances demanded. Now let us analyse: \wmove{17... c5, 18.Nd5 Bxd5, 19.exd5 b5}; and a careful analysis will show that Black has nothing to fear. Black's plan in this case would be to work his Knight around to e5, via c8, b6, and c4 or d7. Again, \wmove{17... c5, 18.Rf2 d5, 19.exd5 Bxd5, 20.Nxd5} (best, since if \wmove{20.Rfd2 Bxe6} give Black the advantage), \bmove{Rxd5, 21.Rxd5 Nxd5}; and there is no good reason why Black should lose.

\mainline{17... Nc8, 18.Rf2 b5, 19.Rfd2 Rde7, 20.b4 Kf7, 21.a3 Ba8} Once more changing my plan and this time without any good reason. Had I now played \variation{21... Rxe6, 22.fxe6+, Rxe6}; as I intended to do when I went back with the Knight to c8. I doubt very much if White would have been able to win the game. At least it would have been extremely difficult.

\mainline{22.Kf2 Ra7, 23.g4 h6, 24.Rd3 a5, 25.h4 axb4, 26.axb4 Rae7} This, of course, has no object now. Black, with a bad game, flounders around for a move. It would have been better to play \variation{26... }R - R 6 to keep the open file, and at the same time to threaten to come out with the Knight at \bmove{Nb6} and \bmove{Nc4}.

\mainline{27.Kf3 Rg8, 28.Kf4 g6} Again bad. White's last two moves were weak, since the White King does nothing here. He should have played his \wmove{Rg3} on the 27th move. Black now should have played \bmove{g5+}. After missing this chance White has it all his own way, and finishes the game most accurately, and Black becomes more helpless with each move. The game needs no further comment, excepting that my play throughout was of an altogether irresolute character. When a plan is made, it must be carried out if at all possible. Regarding the play of White, I consider his 10th and 12th moves were very weak; he played well after that up to the 27th move, which was bad, as well as his 28th move. The rest of his play was good, probably perfect.

\mainline{29.Rg3 g5+, 30.Kf3 Nb6, 31.hxg5 hxg5, 32.Rh3 Rd7, 33.Kg3 Ke8, 34.Rdh1 Bb7, 35.e5 dxe5, 36.Ne4 Nd5, 37.N6c5 Bc8, 38.Nxd7 Bxd7, 39.Rh7 Rf8, 40.Ra1 Kd8, 41.Ra8+ Bc8, 42.Nc5} Resigns.

\begin{center}
\chessboard[normalboard,
moverstyle=triangle]
\end{center}

\chapter{Game 6: French Defence}
Rice Memorial Tournament, 1916 O. Chajes v J. Capablanca

\newgame
\mainline{1.e4 e6, 2.d4 d5, 3.Nc3 Nf6, 4.Bg5 Bb4}
Of all the variations of the French Defence I like this best, because it gives Black more chances to obtain the initiative.

\mainline{5.e5} Rhough I consider \wmove{exd5} the best move, there is much to be said in favour of this move, but not of the variation as a whole, which White adopted in this game.

\mainline{5... h6, 6.Bd2 Bxc3, 7.bxc3 Ne4, 8.Qg4 Kf8} The alternative, \variation{8... g6, 9.h4}; leaves Black's King's side very weak. White by playing \wmove{h4} would force Black to play \bmove{h5}; and later, on White's Bishop by going to d3, would threaten the weakened g Pawn. By the next move Black gives up Castling, but gains time for an attack against White's centre and Queen's side.


\mainline{9.Bc1 c5} Threatening \bmove{Qa5} and stopping thereby White's threat of \wmove{Ba3}. It demonstrates that White's last move was a complete loss of time and merely weakened his position.

\mainline{10.Bd3 Qa5, 11.Ne2 cxd4, 12.O-O dxc3, 13.Bxe4 dxe4, 14.Qxe4 Nc6}

\chessboard[smallboard,
marginleft=false,
marginrightwidth=2em,
moverstyle=triangle]
\begin{wraptable}{r}{0.5\textwidth}
	\vspace{-13em}

Black has come out of the opening with a Pawn to the good. His development, however, has suffered somewhat, and there are Bishops of opposite colour, so that it cannot be said as yet, that Black has a won game; but he has certainly the best of the position,

\end{wraptable}

because, besides being a Pawn to the good, he threatens White's e Pawn, which must of course be defended, and this in turn will give him the opportunity to post his Knight at d5 via e2. When the Black Knight is posted at d5, the Bishop will be developed to c6 via d7, as soon as the opportunity presents itself, and it will be Black that will then have the initiative, and can consequently decide the course of the game.

\mainline{15.Rd1} To prevent \bmove{Ne7}; which would be answered by \wmove{Nxc3}, or still better by \wmove{Ba3}. The move, however, is strategically wrong, since by bringing his pieces to the Queen's side, White loses any chance he might have of making a determined attack on the King's side before Black is thoroughly prepared for it.

\mainline{15... g6, 16.f4 Kg7, 17.Be3} Better would have been \wmove{h4}, in order to play \wmove{Ba3}. The White Bishop would be much better posted on the open diagonal than here, where it acts purely on the defensive.

\mainline{17... Ne7, 18.Bf2 Nd5} This Knight completely paralyses the attack, as it dominates the whole situation, and there is no way to dislodge it. Behind it Black can quietly develop his pieces. The game can now be said to be won for Black strategically.

\mainline{19.Rd3 Bd7, 20.Nd4 Rac8, 21.Rg3 Kh7, 22.h4 Rhg8, 23.h5 Qb4} In order to pin the Knight and be ready to come back to either e7 or f1. Also to prevent \wmove{Rb1}. In reality nearly all these precautions are unnecessary, since White's attack amounts to nothing. Probably Black should have left aside all these considerations, and played \bmove{Qa4} now, in order to follow it up with \bmove{f5}, as he did later, but under less favourable circumstances.

\mainline{24.Rh3}
\chessboard[smallboard,
marginleft=false,
marginrightwidth=2em,
moverstyle=triangle]
\begin{wraptable}{r}{0.5\textwidth}
	\vspace{-13em}

\mainline{24... f5} Not the best, as White will soon prove. \bmove{Qf1}  would have avoided everything, but Black wants to assume the initiative at once and plunges into complications. However, as will soon be seen, the move is not a losing one by any means.

\end{wraptable}

\mainline{25.exf6 Nxf6, 26.hxg6+ Rxg6}

\clearpage

\chessboard[smallboard,
marginleft=false,
marginrightwidth=2em,
moverstyle=triangle]
\begin{wraptable}{r}{0.5\textwidth}
	\vspace{-13em}

\mainline{27.Rxh6+} This wins the Queen.

\mainline{27... Kxh6, 28.Nf5+ exf5, 29.Qxb4}

\end{wraptable}

\chessboard[smallboard,
marginleft=false,
marginrightwidth=2em,
moverstyle=triangle]
\begin{wraptable}{r}{0.5\textwidth}
	\vspace{-13em}

The position looks most interesting. I thought it would be possible to get up such an attack against the White King as to make it impossible for him to hold out much longer, but I was wrong, unless it could have been done by playing \bmove{Bc3} first,
 
\end{wraptable}

forcing \wmove{g3} and then playing \bmove{Kh5}. I followed a similar plan, but lost a very important move by playing \bmove{Rg8}; which gave White time to play \wmove{Rd1}. I am convinced, however, that \bmove{Bc6} at once was the right move. White would be forced to play \wmove{g3}, and Black would reply with either \bmove{Kh5}; as already indicated, which looks the best (the plan, of course, is to play \bmove{Rh8}; and follow it up with \bmove{Kg4}; threatening mate, or some other move according to circumstances. In some cases, of course, it will be better first to play \bmove{Kg4}, or \bmove{Ne4}, which will at least give him a draw. There are so many possibilities in this position that it would be impossible to give them all. It will be worth the reader's time to go carefully through the lines of play indicated above.

\mainline{29... Rcg8} As stated \bmove{Bc3} was the best move.

\mainline{30.g3 Bc6, 31.Rd1 Kh5} The plan, of course, as explained above, is to go to \bmove{Ne5} in due time and threaten mate at \bmove{Rh1}, but it is now too late, the White Rook having come in time to prevent the manœuvre. Instead of the next move, therefore, Black should have played \bmove{Ne5}; which would have given him a draw at the very least. After the next moves the tables are turned. It is now White who has the upper hand, and Black who has to fight for a draw.

\mainline{32.Rd6 Be4} \bmove{Ne5} was still the right move, and probably the last chance Black had to draw against White's best play.

\mainline{33.Qxc3 Nd5, 34.Rxg6 Kxg6} \variation{34... Nxc3, 35.Rxg8 Nxa2} was no better.

\mainline{35.Qe5 Kf7, 36.c4 Re8, 37.Qb2 Nf6, 38.Bd4 Rh8, 39.Qb5 Rh1+, 40.Kf2 a6, 41.Qb6 Rh2+, 42.Ke1 Nd7, 43.Qd6 Bc6, 44.g4 fxg4, 45.f5 Rh1+, 46.Kd2 Ke8, 47.f6 Rh7, 48.Qe6+ Kf8, 49.Be3 Rf7, 50.Bh6+ Kg8} Most players will be wondering, as the spectators did, why I did not resign. The reason is that while I knew the game to be lost, I was hoping for the following variation, which Chajes came very near playing: \variation{50... Kg8, 51.Qxg4+ Kh7, 52.Qh5 Rxf6, 53.Bg5+ Kg7, 54.Bxf6+ Kxf6}; and while White has a won game it is by no means easy. If the reader does not believe it, let him take the White pieces against a master and see what happens. My opponent, who decided to take no chances, played \wmove{51.Bg7}, and finally won as shown below.

\mainline{51.Bg7 g3, 52.Ke2 g2, 53.Kf2 Nf8, 54.Qg4 Nd7, 55.Kg1 a5, 56.a4 Bxa4, 57.Qh3 Rxf6, 58.Bxf6 Nxf6, 59.Qxg2+ Kf8, 60.Qxb7 Be8}and after a very few more moves Black resigned.

A very fine game on Chajes' part from move 25 on, for while Black, having the best of the position, missed several chances, White, on the other hand, missed none.

The game finishes in the following manner.

\mainline{61.Qb6 Ke7, 62.Qxa5 Nd7, 63.Kf2 Bf7, 64.Ke3 Kd6, 65.Kd4 Kc6, 66.Qf5} Resigns.

\begin{center}
\chessboard[largeboard,
moverstyle=triangle]
\end{center}

\chapter{Game 7: Ruy Lopez}
San Sebastian, 1911 J. Capablanca v A. Burn

\newgame
\mainline{1.e4 e5, 2.Nf3 Nc6, 3.Bb5 a6, 4.Ba4 Nf6, 5.d3} This is a very solid development, to which I was much addicted at the time, because of my ignorance of the multiple variations of the openings.

\mainline{5... d6 6.c3 Be7} In this variation there is the alternative of developing this Bishop via \wmove{g7}, after \wmove{g6}.

\mainline{7.Nbd2 O-O, 8.Nf1 b5, 9.Bc2 d5, 10.Qe2 dxe4, 11.dxe4 Bc5} Evidently to make room for the Queen at e7, but I do not think the move advisable at this stage. \wmove{Be6} is a more natural and effective move. It develops a piece and threatens \wmove{Bc4}, which would have to be stopped.

\mainline{12.Bg5 Be6} Now it is not so effective, because White's dark square Bishop is out, and the Knight, in going to e3 to defend the square c4, does not block the dark square Bishop.

\mainline{13.Ne3 Re8, 14.O-O Qe7} This is bad. Black's game was already not good. He probably had no choice but to take the Knight with the Bishop before making this move.

\chessboard[smallboard,
marginleft=false,
marginrightwidth=2em,
moverstyle=triangle]
\begin{wraptable}{r}{0.5\textwidth}
	\vspace{-13em}

\mainline{15.Nd5 Bxd5 16.exd5 Nb8} in order to bring the knight d7, to support the other Knight and also his King's Pawn. White, however, does not allow time for this,

\end{wraptable}

and by taking advantage of his superior position is able to win a Pawn.

\mainline{17.a4 b4} Since he had no way to prevent the loss of a Pawn, he should have given it up where it is, and played \variation{17... Nbd7}, in order to make his position more solid. The next move not only loses a Pawn, but leaves Black's game very much weakened.

\mainline{18.cxb4 Bxb4, 19.Bxf6 Qxf6, 20.Qe4 Bd6, 21.Qxh7+ Kf8}

\chessboard[smallboard,
marginleft=false,
marginrightwidth=2em,
moverstyle=triangle]
\begin{wraptable}{r}{0.5\textwidth}
	\vspace{-13em}

With a Pawn more and all his pieces ready for action, while Black is still backward in development, it only remains for White to drive home his advantage before Black can come out with his pieces, in which case, by using the open h file,

\end{wraptable}

Black might be able to start a strong attack against White's King. White is able by his next move to eliminate all danger.

\mainline{22.Nh4 Qh6} This is practically forced. Black could not play \bmove{g6} because of \wmove{Bxg6}, and White meanwhile threatened \wmove{Qh8+} followed by \wmove{Nf5+} and \wmove{Qxe5}.

\mainline{23.Qxh6 gxh6, 24.Nf5 h5, 25.Bd1 Nd7, 26.Bxh5 Nf6, 27.Be2 Nxd5, 28.Rfd1 Nf4, 29.Bc4 Red8, 30.h4 a5} Black must lose time assuring the safety of this Pawn.

\mainline{31.g3 Ne6, 32.Bxe6 fxe6, 33.Ne3 Rdb8, 34.Nc4 Ke7} Black fights a hopeless battle. He is two Pawns down for all practical purposes, and the Pawns he has are isolated and have to be defended by pieces.

\mainline{35.Rac1 Ra7} White threatened \wmove{Nxd6}, followed by \wmove{Rxc7+}.

\mainline{36.Re1 Kf6 37.Re4 Rb4 38.g4 Ra6} If \variation{38... Rxa4, 39.Nxd6} of course would win a piece.

\mainline{39.Rc3 Bc5
40.Rf3+ Kg7 41.b3 Bd4 42.Kg2 Ra8 43.g5 Ra6 44.h5 Rxc4 45.bxc4
Rc6 46.g6} Resigns.

\begin{center}
\chessboard[largeboard,
moverstyle=triangle]
\end{center}

\chapter{Game 8: Centre Game}

Berlin 1913, J.Mieses v J.Capablanca

\newgame
\mainline{1.e4 e5, 2.d4 exd4, 3.Qxd4 Nc6, 4.Qe3 Nf6, 5.Nc3 Bb4, 6.Bd2 O-O, 7.O-O-O Re8} In this position, instead of the next move, \bmove(d6) is often played in order to develop the Queenside Bishop. My idea was to exert sufficient pressure against the e Pawn to win it, and thus gain a material advantage, which would, at least, compensate whatever slight advantage of position White might have. The plan, I think, is quite feasible, my subsequent difficulties being due to faulty execution of the plan.

\mainline{8.Qg3 Nxe4 9.Nxe4 Rxe4 10.Bf4}

\begin{center}
\chessboard[smallboard,
moverstyle=triangle]
\end{center}

\mainline{10... Qf6} White's threat to regain the Pawn was merely with the idea of gaining time to develop his pieces. Black could have played \bmove{d6}; opening the way for the light squared Bishop, when would have followed, \variation{10... d6, 11.Bd3 Re8, 12.Nf3} and White would soon start a powerful direct attack against Black's King. With the next move Black aims at taking the initiative away from White in accordance with the principles laid down in this book.

\mainline{11.Nh3} If \variation{11.Bxc7 d6}; and White's Bishop would be completely shut off, and could only be extricated, if at all, with serious loss of position. the next move aims at quick development to keep the initiative.

\mainline{11... d6} This now is not only a developing move, but it also threatens to win a piece by \bmove{Bxh3}.

\mainline{12.Bd3 Nd4} This complicates the game unnecessarily. \variation{12... Re8}; was simple, and perfectly safe.

\mainline{13.Be3}

\chessboard[smallboard,
marginleft=false,
marginrightwidth=2em,
moverstyle=triangle]
\begin{wraptable}{r}{0.5\textwidth}
	\vspace{-13em}

\mainline{13... Bg4} This is a serious mistake. The position was most interesting, and though in appearance dangerous for Black, not so in reality. The right move would have been \bmove{Rg4}, when we would have 

\end{wraptable}

\variation{13... Rg4, 14.Bxd4 Rxd4, 15.c3 Bxc3, 16.bxc3 Rg4, 17.Qe3! Qxc3+, 18.Bc2 Qxe3+, 19.fxe3 Rxg2}, and Black has the best of the game with four Pawns for a Knight, besides the fact that all the White Pawns are isolated.

\mainline{14.Ng5! Rxe3} There was nothing better.

\mainline{15.Qxg4! Ne2+}

\chessboard[smallboard,
marginleft=false,
marginrightwidth=2em,
moverstyle=triangle]
\begin{wraptable}{r}{0.5\textwidth}
	\vspace{-13em}

\mainline{16.Bxe2 Rxe2, 17.Ne4 Rxe4, 18.Qxe4 Qg5+, 19.f4 Qb5, 20.c3 Bc5, 21.Rhe1 Qc6, 22.Rd5}

\end{wraptable}

\wmove{Qxc6} would have given White a decided advantage, enough to win with proper play. Mieses, however, feared the difficulties of an ending where, while having the exchange, he would be a Pawn minus. He preferred to keep the Queens on the board and keep up the attack. At first sight, and even after careful thought, there seems to be no objection to his plan; but in truth such is not the case. From this point the game will gradually improve in Black's favour until, with the exchange ahead, White is lost.

\mainline{22... Qd7, 23.f5 c6, 24.Rd2 d5}

\chessboard[smallboard,
marginleft=false,
marginrightwidth=2em,
moverstyle=triangle]
\begin{wraptable}{r}{0.5\textwidth}
	\vspace{-13em}

My plan for the moment is very simple. It will consist in bringing my Bishop around to f6. Then I shall try to paralyse White's attack against my King by playing \bmove{h6}, and also prevent White from ever playing \wmove{g5}. 

\end{wraptable}

Once my King is safe from attack I shall begin to advance my Queen's side Pawns, where there are four to three; and that advantage, coupled with the enormous attacking power of my Bishop at f6, will at least assure me an even chance of success.

\mainline{25.Qf3 Be7, 26.Rde2 Bf6, 27.Qh5 h6, 28.g4 Kh7} To prevent \wmove{h4}, which I would answer with \bmove{g6}, winning the Queen. It can now be considered that my King is safe from attack. White will have to withdraw his Queen via h3, and Black can use the time to begin his advance on the Queen's side.

\mainline{29.Kb1 Rd8, 30.Rd1 c5} Notice that, on assuming the defensive, White has placed his Rooks correctly from the point of view of strategy. They are both on white squares free from the possible attack of the Black Bishop.

\mainline{31.Qh3 Qa4} This gains time by attacking the Rook and holding the White Queen at h3 for the moment, on account of the g Pawn. Besides, the Queen must be in the middle of the fray now that the attack has to be brought home. White has actually more value in material, and therefore Black must utilise everything at his command in order to succeed.

\mainline{32.Red2 Qe4+, 33.Ka1 b5} threatening \bmove{b4}; which would open the line of action of the Bishop and also secure a passed Pawn.


\mainline{34.Qg2 Qa4} indirectly defending the d Pawn, which White cannot take on account of \wmove{35.Qxd5 Qxd1+}.

\mainline{35.Kb1 b4} The attack increases in force as it is gradually brought home directly against the King. The position now is most interesting and extremely difficult. It is doubtful if there is any valid defence against Black's best play. The variations are numerous and difficult.

\chessboard[smallboard,
marginleft=false,
marginrightwidth=2em,
moverstyle=triangle]
\begin{wraptable}{r}{0.5\textwidth}
	\vspace{-13em}

\mainline{36.cxb4 Qxb4} Black has now a passed Pawn, and his Bishop exerts great pressure. White cannot very well play now \wmove{37.Rxd5} because of \bmove{Rxd5, 38.Rxd5 Bxb2}; 

\end{wraptable}

and White could not take the Bishop because \bmove{Qe4+} would win the Rook, leaving Black a clear passed Pawn ahead.

\mainline{37.a3 Qa4, 38.Rxd5 Rb8, 39.R1d2 c4, 40.Qg3 Rb3, 41.Qd6}

\chessboard[smallboard,
marginleft=false,
marginrightwidth=2em,
moverstyle=triangle]
\begin{wraptable}{r}{0.5\textwidth}
	\vspace{-13em}

\mainline{41... c3} \variation{41... Bxb2} would also win, which shows that White's game is altogether gone. In these cases, however, it is not the prettiest move that should be played, but the most effective one, the move that will make your opponent resign soonest.

\end{wraptable}

\mainline{42.Rc2 cxb2, 43.Rd3 Qe4, 44.Rd1 Rc3} Resigns
           
Of course White must play \wmove{Qd2} then \bmove{Rxa3}

\clearpage

\chapter{Game 9: Queen's Gambit Declined}

Berlin, 1913. J.Capablanca v R.Teichmann

\newgame
\mainline{1.d4 d5, 2.Nf3 Nf6, 3.c4 e6, 4.Bg5 Be7, 5.Nc3 Nbd7, 6.e3 O-O, 7.Rc1 b6, 8.cxd5 exd5, 9.Bb5} An invention of my own, I believe. I played it on the spur of the moment simply to change the normal course of the game. Generally the Bishop goes to \wmove{Bd3}, or to \wmove{Ba6}, after \wmove{Qa4}. the next move is in the nature of an ordinary developing move, and as it violates no principle it cannot be bad.

\mainline{9... Bb7, 10.O-O a6, 11.Ba4 Rc8, 12.Qe2 c5, 13.dxc5 Nxc5} If \variation{13... bxc5, 14. Rfd1}, and White would play to win one of Black's centre Pawns. The drawback to the next move is that it leaves Black's d Pawn isolated, and consequently weak and subject to attack.

\mainline{14.Rfd1 Nxa4} The alternative would have been \variation{14... b5, 15.Bc2 b4, 16.Na4 Nce4}.

\mainline{15.Nxa4 b5, 16.Rxc8 Qxc8, 17.Nc3 Qc4} Black aims at the exchange of Queens in order to remain with two Bishops for the ending, but in this position such a course is a mistake, because the Bishop at b7 is inactive and cannot come into the game by any means, unless Black gives up the isolated Queen's Pawn which the Bishop must defend.

\mainline{18.Nd4} Not, of course, \wmove{Rd4}, because of \variation{18.Rd4 Qxe2, 19.Nxe2}; and there would be no good way to prevent \bmove{Rc2}.

\mainline{18... Qxe2 19.Ncxe2!} Notice the co-ordination of the Knights' moves. They are manœuvred chain-like, so to speak, in order to maintain one of them, either at Q 4 or ready to go there. Now White threatens to take the open file, and therefore forces Black's next move.

\mainline{19... Rc8}

\chessboard[smallboard,
marginleft=false,
marginrightwidth=2em,
moverstyle=triangle]
\begin{wraptable}{r}{0.5\textwidth}
	\vspace{-13em}

the student should examine this position carefully. There seems to be no particular danger, yet, as White will demonstrate, Black may be said to be lost. If the game is not altogether lost, the defence is at least of the most difficult kind; indeed, I must confess that I can see no adequate defence against White's next move.

\end{wraptable}

\mainline{20.Nf5! Kf8} If \variation{20... Bd8, 21.Nd6 Rc7, 22.Nxb7 Rxb7, 23.Bxf6 Bxf6, 24.Rxd5 Rc5, 25.Rd2}, and White is a Pawn ahead. If \variation{20... Bd4, 21.Bxf6 gxf6}, doubling the f Pawns and isolating all of Black's King's side Pawns.

\mainline{21.Nxe7 Kxe7, 22.Nd4 g6} This is practically forced, as White threatened \wmove{Nf5+}. Notice that the Black Knight is pinned in such a way that no relief can be afforded except by giving up the h Pawn or abandoning the open file with the Rook, which would be disastrous, as White would immediately sieze it.

\mainline{23.f3!}

\chessboard[smallboard,
marginleft=false,
marginrightwidth=2em,
moverstyle=triangle]
\begin{wraptable}{r}{0.5\textwidth}
	\vspace{-13em}

\mainline{23...h6} Black could do nothing else except mark time with his Rook along the open file, since as soon as he moved away White would take it. White, on the other hand, threatens to march up with his King to e5 

\end{wraptable}

via f2, g3, f4, after having, of course, prepared the way. Hence, Black's best chance was to give up a Pawn, as in the text, in order to free his Knight.

\mainline{24.Bxh6 Nd7, 25.h4 Nc5, 26.Bf4 Ne6} Black exchanges Knights to remain with Bishops of opposite colours, which gives him the best chance to draw.
   
\mainline{27.Nxe6 Kxe6} \variation{27... fxe6} would be worse, as White would then be able to post his Bishop at e5.

\mainline{28.Rd2 Rh8}

\chessboard[smallboard,
marginleft=false,
marginrightwidth=2em,
moverstyle=triangle]
\begin{wraptable}{r}{0.5\textwidth}
	\vspace{-13em}

Black wants to force \wmove{Bf4}. \wmove{g3} would be bad, on account of \bmove{d4}; which would get the Black Bishop into the game, even though White could answer \wmove{e4}. 

\end{wraptable}

The next move is, however, weak, as will soon be seen. His best chance was to play \bmove{b4}; and follow it up with \bmove{a4} and \bmove{Ba3}. White meanwhile could play \wmove{g4} and \wmove{g5}, obtaining a passed Pawn, which, with proper play, should win.
   
\mainline{29.Rc2! Rc8, 30.Rxc8 Bxc8} There are now Bishops of opposite colour, but nevertheless White has an easily-won game.

\mainline{31.Kf2}

\chessboard[smallboard,
marginleft=false,
marginrightwidth=2em,
moverstyle=triangle]
\begin{wraptable}{r}{0.5\textwidth}
	\vspace{-13em}

\mainline{31... d4} Practically forced. Otherwise the White King would march up to d4 and then to c5 and win Black's Queen's side Pawns. If Black attempted to stop this by putting his King at c6 then the White King would enter through e5 into Black's King's side and win just as easily.

\end{wraptable}

\mainline{32.exd4 Kd5, 33.Ke3 Be6, 34.Kd3 Kc6, 35.a3 Bc4+, 36.Ke3 Be6, 37.Bh6} It is better not to hurry \bmove{g5} because of \wmove{f4}; for although White could win in any case, it would take longer. Now the White King threatens to help by going in through g4 after posting the Bishop at g7, where it not only protects the d Pawn, but indirectly also the c Pawn.

\mainline{37... Kd5, 38.Bg7} Resigns.

The student ought to have realised by this time the enormous importance of playing well every kind of ending. In this game again, practically from the opening, White aimed at nothing but the isolation of Black's d Pawn. Once he obtained that, he tried for and obtained, fortunately, another advantage of position elsewhere which translated itself into the material advantage of a Pawn. Then by accurate playing in the ending he gradually forced home his advantage. This ending has the merit of having been played against one of the finest players in the world.

\chapter{Game 10: Petroff's Defence}

St. Petersburg, 1914. J.Capablanca v F.Marshall

\newgame
\mainline{1.e4 e5, 2.Nf3 Nf6, 3.Nxe5 d6, 4.Nf3 Nxe4, 5.Qe2 Qe7, 6.d3 Nf6, 7.Bg5} Played by Morphy, and a very fine move. The point is that should Black exchange Queens he will be a move behind in development and consequently will get a cramped game if White plays accurately.

\mainline{7... Be6} Marshall thought at the time that this was the best move and consequently played it in preference to \variation{7... Qxe2}.

\mainline{8.Nc3 h6, 9.Bxf6 Qxf6, 10.d4 Be7, 11.Qb5+ Nd7, 12.Bd3!}

\chessboard[smallboard,
marginleft=false,
marginrightwidth=2em,
moverstyle=triangle]
\begin{wraptable}{r}{0.5\textwidth}
	\vspace{-13em}

It is now time to examine the result of the opening. On White's side we find the minor pieces well posted and the Queen out in a somewhat odd place, it is true, but safe from attack and actually attacking a Pawn. White is also ready to Castle. 

\end{wraptable}

White's position is evidently free from danger and his pieces can easily manœuvre.

On Black's side the first thing we notice is that he has retained both his Bishops, unquestionably an advantage; but on the other hand we find his pieces bunched together too much, and the Queen in danger of being attacked without having any good square to go to. The Bishop at e7 has no freedom and it blocks the Queen, which, in its turn, blocks the Bishop. Besides, Black cannot Castle on the King's side because \bmove{0-0, 13.Qxb7 Rab8, 14.Qe4} threatening mate, wins a Pawn. Nor can he Castle on the Queen's side because \wmove{Qa5} would put Black's game in imminent danger, since he cannot play \wmove{a6} because of \bmove{Bxa6}; nor can he play \bmove{Kb8} because of \wmove{Nb5}. Consequently we must conclude that the opening is all in White's favour.

\mainline{12... g5} to make room for his Queen, threatening also \bmove{g4}.

\mainline{13.h3 O-O} giving up a Pawn in an attempt to free his game and take the initiative. It was difficult for him to find a move, as White threatened \wmove{Ne4}, and should Black go with the Queen to g7, then \wmove{d5 Bf5, Nxd6+}, followed by \wmove{Bxf5}.

\mainline{14.Qxb7 Rab8 15.Qe4 Qg7 16.b3 c5} In order to break up White's centre and bring his Knight to c5 and thus lay the foundation for a violent attack against White's King. The plan, however, fails, as it always must in such cases, because Black's development is backward, and consequently his pieces are not properly placed.

\mainline{17.O-O cxd4 18.Nd5!} A simple move, which destroys Black's plan utterly. Black will now have no concerted action of his pieces, and, as his Pawns are all weak, he will sooner or later lose them.

\chessboard[smallboard,
marginleft=false,
marginrightwidth=2em,
moverstyle=triangle]
\begin{wraptable}{r}{0.5\textwidth}
	\vspace{-13em}

\mainline{18... Bd8, 19.Bc4 Nc5, 20.Qxd4 Qxd4} The fact that he has to exchange Queens when he is a Pawn behind shows that Black's game is lost.

\end{wraptable}

\mainline{21.Nxd4 Bxd5, 22.Bxd5 Bf6, 23.Rad1 Bxd4} The Knight was too threatening. But now the ending brought about is one in which the Bishop is stronger than the Knight; which makes Black's plight a desperate one. The game has no further interest, and it is only because of its value as a study of this variation of the Petroff that I have given it. Black was able to fight it out until the sixtieth move on account of some poor play on White's part. The rest of the moves are given merely as a matter of form.

\mainline{24.Rxd4 Kg7, 25.Bc4 Rb6, 26.Re1 Kf6, 27.f4 Ne6, 28.fxg5+ hxg5, 29.Rf1+ Ke7, 30.Rg4 Rg8, 31.Rf5 Rc6, 32.h4 Rgc8, 33.hxg5 Rc5, 34.Bxe6 fxe6, 35.Rxc5 Rxc5, 36.g6 Kf8, 37.Rc4 Ra5, 38.a4 Kg7, 39.Rc6 Rd5, 40.Rc7+ Kxg6, 41.Rxa7 Rd1+, 42.Kh2 d5, 43.a5 Rc1, 44.Rc7 Ra1, 45.b4 Ra4, 46.c3 d4, 47.Rc6 dxc3, 48.Rxc3 Rxb4, 49.Ra3 Rb7, 50.a6 Ra7, 51.Ra5 Kf6, 52.g4 Ke7, 53.Kg3 Kd6, 54.Kf4 Kc7, 55.Ke5 Kd7, 56.g5 Ke7, 57.g6 Kf8, 58.Kxe6 Ke8, 59.g7 Rxg7, 60.a7 Rg6+, 61.Kf5} Resigns

\begin{center}
\chessboard[normalboard,
moverstyle=triangle]
\end{center}

\chapter{Game 11: Ruy Lopez}

St. Petersburg, 1914. J.Capablanca v D.Janowski

\newgame
\mainline{1.e4 e5, 2.Nf3 Nc6, 3.Bb5 a6, 4.Bxc6 dxc6, 5.Nc3} I played this move after having discussed it with Alechine on several occasions. Alechine considered it, at the time, superior to \variation{5.d4}, which is generally played. He played it himself later on in the Tournament, in one of his games against Dr. E. Lasker, and obtained the superior game, which he only lost through a blunder.

\mainline{5... Bc5} \variation{5... f6} is probably the best move in this position. I do not like the next move.

\mainline{6.d3 Bg4, 7.Be3 Bxe3} This opens the f file for White, and also reinforces his centre, but Black naturally did not want to make a second move with this Bishop.

\mainline{8.fxe3 Qe7 9.O-O O-O-O} Bold play, typical of Janowski. 

\mainline{10.Qe1 Nh6}

\chessboard[smallboard,
marginleft=false,
marginrightwidth=2em,
moverstyle=triangle]
\begin{wraptable}{r}{0.5\textwidth}
	\vspace{-13em}

The problem for White now is to advance his b Pawn to Kt 5 as fast as he can. If he plays \wmove{b4} at once, Black simply takes it. If he plays first \wmove{a3} and then \wmove{b4}, he will still have to protect his b Pawn before he can go on and play \wmove{a4} and \wmove{b5}. 

\end{wraptable}

As a matter of fact White played a rather unusual move, but one which, under the circumstances, was the best, since after it he could at once play \wmove{b4} and then \wmove{a4} and \wmove{b5}.

\mainline{11.Rb1 f6, 12.b4 Nf7, 13.a4 Bxf3} He simplifies, hoping to lighten White's attack, which will have to be conducted practically with only the heavy pieces on the board. He may have also done it in order to play \bmove{Ng5} and \bmove{Ne6}. 

\mainline{14.Rxf3} Taking with the Pawn would have opened a possibility for a counter attack.

\mainline{14... b6} He is forced to this in order to avoid the breaking up of his Queen's side Pawns. The only alternative would have been \variation{14... b5}; which on the face of it looks bad.

\mainline{15.b5 cxb5, 16.axb5 a5, 17.Nd5 Qc5, 18.c4}

\chessboard[smallboard,
marginleft=false,
marginrightwidth=2em,
moverstyle=triangle]
\begin{wraptable}{r}{0.5\textwidth}
	\vspace{-13em}

The White Knight is now a tower of strength. Behind it White will be able to prepare an attack, which will begin with \wmove{d4}, to drive away the Black Queen and thus leave himself free to play \bmove{c5}. There is only one thing to take care of and that is to prevent Black from sacrificing the Rook for the Knight and a Pawn.

\end{wraptable}

\mainline{18... Ng5 19.Rf2 Ne6 20.Qc3 Rd7} Had White on his 19th move played \wmove{Rff1} instead of \wmove{Rf2}, Black could have played now instead of the next move, \bmove{Rxd5, 21.exd5 Qxd5+}; followed by \bmove{Nc5} with a winning game.

\mainline{21.Rd1 Kb7} It would have been better for Black to play \bmove{Ke8}. The next move loses very rapidly.

\mainline{22.d4 Qd6, 23.Rc2 exd4, 24.exd4 Nf4, 25.c5 Nxd5, 26.exd5 Qxd5, 27.c6+ Kb8, 28.cxd7 Qxd7, 29.d5 Re8, 30.d6 cxd6, 31.Qc6} Resigns.


\chapter{Game 12: French Defence}
New York, 1918. J.Capablanca v O.Chajes

\newgame
\mainline{1.e4 e6, 2.d4 d5, 3.Nc3 Nf6, 4.Bd3}

Not the most favoured move, but a perfectly natural developing one, and consequently it cannot be bad.

\mainline{4... dxe4} \bmove{c5} is generally played in this case instead of the next move.

\mainline{5.Nxe4 Nbd7, 6.Nxf6+ Nxf6, 7.Nf3 Be7}

\chessboard[smallboard,
marginleft=false,
marginrightwidth=2em,
moverstyle=triangle]
\begin{wraptable}{r}{0.5\textwidth}
	\vspace{-13em}

\mainline{8.Qe2} This is played to prevent \wmove{b3}, followed by \wmove{Bb2}2, which is the general form of development for Black in this variation. If Black now plays \variation{8.b3 Bb4+, 9.Bd2 Ne4} and White obtains a considerable advantage in position.

\end{wraptable}

\mainline{8... O-O, 9.Bg5 h6} Of course Black could not play \bmove{b6} because of \wmove{Bxf6}, followed by \wmove{Qe4}.

\mainline{10.Bxf6 Bxf6 11.Qe4 g6} This weakens Black's King's side. \bmove{Re8} was the right move.

\mainline{12.h4}

\chessboard[smallboard,
marginleft=false,
marginrightwidth=2em,
moverstyle=triangle]
\begin{wraptable}{r}{0.5\textwidth}
	\vspace{-13em}

\mainline{12... e5} This is merely giving up a Pawn in order to come out quickly with his light squared Bishop. But as he does not obtain any compensation for his Pawn, the move is bad. He should have played \bmove{Qd5} and tried to fight the game out that way. 

\end{wraptable}

It might have continued thus: \variation{12... Qd5, 13.Qf4 Bg7, 14.Qxc7 Bxd4, 15.Nxd4 Qxd4.16.O-O-O} with considerable advantage of position for White. The next move might be considered a mild form of suicide.

\mainline{13.dxe5 Bf5, 14.Qf4 Bxd3, 15.O-O-O Bg7, 16.Rxd3 Qe7, 17.Qc4} In order to keep the Black Queen from coming into the game.

\mainline{17... Rad8, 18.Rhd1} A better plan would have been to play \wmove{Re1}, threatening \wmove{e6}.

\mainline{18... Rxd3, 19.Rxd3 Re8, 20.c3 c6} Of course if \variation{20... Bxe5, 21.Nxe5 Qxe5, 22.Re3}. Black with a Pawn minus fights very hard.

\mainline{21.Re3} The Pawn had now to be defended after Black's last move, because after \bmove{Bxe5, 21.Nxe5 Qxe5, 22.Re3}3, Black could now play \bmove{Qb8} defending the Rook.

\mainline{21... c5, 22.Kc2 b6, 23.a4} White's plan now is to \emph{fix} the Queen's side in order to be able to manœuvre freely on the other side, where he has the advantage of material.

\mainline{23... Qd7, 24.Rd3 Qc8, 25.Qe4 Qe6, 26.Rd5 Kf8, 27.c4 Kg8}

\chessboard[smallboard,
marginleft=false,
marginrightwidth=2em,
moverstyle=triangle]
\begin{wraptable}{r}{0.5\textwidth}
	\vspace{-13em}

Black sees that he now stands in his best defensive position, and therefore waits for White to show how he intends to break through. He notices, of course, that the White Knight is in the way of the f Pawn, which cannot advance to f4 to defend, or support rather, the Pawn at e5. 

\end{wraptable}

\mainline{28.b3 Kf8, 29.Kd3 Kg8, 30.Rd6 Qc8, 31.Rd5 Qe6, 32.g4 Kf8, 33.Qf4 Kg8, 34.Qe4 Kf8}

\chessboard[smallboard,
marginleft=false,
marginrightwidth=2em,
moverstyle=triangle]
\begin{wraptable}{r}{0.5\textwidth}
	\vspace{-13em}

Black persists in waiting for developments. He sees that if \wmove{35.h5 gxh5, 36.gxh5}, the Queen goes to \bmove{Qh3}, and White will have to face serious difficulties. 

\end{wraptable}

In this situation White decides that the only course is to bring his King to g3, so as to defend the squares h3 and g4, where the Black Queen might otherwise become a source of annoyance.

\mainline{35.Ke2 Kg8, 36.Kf1 Kf8, 37.Kg2 Kg8, 38.Kg3 Kf8} Now that he has completed his march with the King, White is ready to advance.

\chessboard[smallboard,
marginleft=false,
marginrightwidth=2em,
moverstyle=triangle]
\begin{wraptable}{r}{0.5\textwidth}
	\vspace{-13em}

\mainline{39.h5 gxh5} \variation{39... g5} would be answered by \wmove{Qf5}, with a winning game.

\end{wraptable}

\mainline{40.gxh5 Qe7} Against \variation{40... Kg8}; White would play \wmove{Qg4}, practically forcing the exchange of Queens, after which White would have little trouble in winning the ending, since Black's Bishop could not do much damage in the resulting position.

\mainline{41.Qf5 Kg8} Black overlooks the force of \wmove{42.Rd7}. His best defence was \variation{41... Rd8}; against which White could either advance the King or play \wmove{Nh4}, threatening \wmove{Ng6+}.

\mainline{42.Rd7 Bxe5+} This loses a piece, but Black's position was altogether hopeless.

\mainline{43.Kg4 Qf6, 44.Nxe5 Qg7+, 45.Kf4} Resigns.

The interest of this game centres mainly on the opening and on the march of the White King during the final stage of the game. It is an instance of the King becoming a fighting piece, even while the Queens are still on the board.

\begin{center}
\chessboard[largeboard,
moverstyle=triangle]
\end{center}

\chapter{Game 13: Ruy Lopez}

New York, 1918. J.Morrison v J.Capablanca

\newgame
\mainline{1.e4 e5, 2.Nf3 Nc6, 3.Bb5 d6, 4.Nc3 Bd7, 5.d4 exd4, 6.Nxd4 g6} In this form of defence of the Ruy Lopez the development of the dark squared bishop via g7 is, I think, of great importance. The Bishop at g7 exerts great pressure along the long diagonal. At the same time the position of the Bishop and Pawns in front of the King, once it is Castled, is one of great defensive strength. Therefore, in this form of development, the Bishop, we might say, exerts its maximum strength (Compare this note with the one in the Capablanca-Burn game at San Sebastian (Game 7).

\mainline{7.Nf3 Bg7, 8.Bg5 Nf6} Of course not \variation{8... Nge7}; because of \wmove{Nd5}. The alternative would have been \variation{8... f6}; to be followed by \bmove{Ne7}; but in this position it is preferable to have the Knight at f6.

\mainline{9.Qd2 h6, 10.Bh4} An error of judgment. White wants to keep the Knight pinned, but it was more important to prevent Black from Castling immediately. \variation{10.Bf4} would have done this.

\mainline{10... O-O, 11.O-O-O} Bold play, but again faulty judgment, unless he intended to play to win or lose, throwing safety to the winds. The Black Bishop at g7 becomes a very powerful attacking piece. The strategical disposition of the Black pieces is now far superior to White's, therefore it will be Black who will take the offensive.

\mainline{11... Re8, 12. Rhe1}

\chessboard[smallboard,
marginleft=false,
marginrightwidth=2em,
moverstyle=triangle]
\begin{wraptable}{r}{0.5\textwidth}
	\vspace{-13em}

White wanted to keep his Rook on the open file behind the Queen, and consequently brings over his other Rook to the centre to defend his e Pawn, which Black threatened to win by \bmove{g5}, followed by \bmove{Nxe4}.

\end{wraptable}

\mainline{12... g5!} Now that the h Rook is in the centre, Black can safely advance, since, in order to attack on the King's side, White would have to shift his Rooks, which he cannot do so long as Black keeps up the pressure in the centre.

\mainline{13.Bg3 Nh5} Uncovering the Bishop, which now acts along the long diagonal, and at the same time preventing \wmove{14.e5}, which would be answered by \bmove{Nxg3, 15.hxg3 Nxe5}; etc., winning a Pawn.

\mainline{14.Nd5 a6} Black drives the Bishop away so as to \emph{unpin} his pieces and be able to manœuvre freely.

\mainline{15.Bd3 Be6} Preparing the onslaught. Black's pieces begin to bear against the King's position.

\mainline{16.c3}

\chessboard[smallboard,
marginleft=false,
marginrightwidth=2em,
moverstyle=triangle]
\begin{wraptable}{r}{0.5\textwidth}
	\vspace{-13em}

With the last move White not only blocks the action of Black's dark Bishop, but he also aims at placing his Bishop at b1 and his Queen at c2, and then advancing his e Pawn, to check at h7 with the Queen.

\end{wraptable}

\mainline{16... f5!} Initiating an attack to which there is no reply, and which has for its ultimate object either the winning of the White dark Bishop or cutting it off from the game. (Compare this game with the Winter-Capablanca game at Hastings.)

\mainline{17.h4 f4} The Bishop is now out of action. White naturally counter attacks violently against the seemingly exposed position of the Black King, and, with very good judgment, even offers the Bishop.

\chessboard[smallboard,
marginleft=false,
marginrightwidth=2em,
moverstyle=triangle]
\begin{wraptable}{r}{0.5\textwidth}
	\vspace{-13em}

\mainline{18.hxg5! hxg5!} Taking the Bishop would be dangerous, if not actually bad, while the text move accomplishes Black's object, which is to put the Bishop out of action.

\end{wraptable}

\mainline{19.Rh1 Bf7, 20.Kb1} This move unquestionably loses time. Since he would have to retire his Bishop to h2 sooner or later, he might have done it immediately. It is doubtful, however, if at this stage of the game it would be possible for White to save the game.

\mainline{20... Ne5, 21.Nxe5 Rxe5} It was difficult to decide which way to retake. I took with the Rook in order to have it prepared for a possible attack against the King.

\mainline{22.Bh2 Nf6} Now that the Bishop has been driven back, Black wants to get rid of White's strongly posted Knight at d5, which blocks the attack of the Bishop at f7. It may be said that the Knight at d5 is the key to White's defence.

\chessboard[smallboard,
marginleft=false,
marginrightwidth=2em,
moverstyle=triangle]
\begin{wraptable}{r}{0.5\textwidth}
	\vspace{-13em}

\mainline{23.g3} White strives not only to have play for his Bishop, but also he wants to break up Black's Pawns in order to counter-attack. The alternative would have been \variation{23.Nxf6+ Qxf6}; and Black would be threatening \bmove{Ra4}, and also \bmove{Qe6}. 

\end{wraptable}

The student should notice that Black's drawback in all this is the fact that he is playing minus the services of his a file Rook. It is this fact that makes it possible for White to hold out longer.

\mainline{23... Nxe4, 24.Bxe4 Rxe4, 25.gxf4 c6}

\chessboard[smallboard,
marginleft=false,
marginrightwidth=2em,
moverstyle=triangle]
\begin{wraptable}{r}{0.5\textwidth}
	\vspace{-13em}

\mainline{26.Ne3} \variation{26.Nb4} was the alternative, but in any event White could not resist the attack. I leave it to the reader to work this out for himself, as the variations are so numerous that they would take up too much space.

\end{wraptable}

\mainline{26... Qa5, 27.c4 Qxd2, 28.Rxd2 gxf4, 29.Ng4 Bg6} This forces the King to the corner, where he will be in a mating net.

\mainline{30.Ka1 Rae8} Now at last the Rook on the eighth rank enters into the game and soon the battle is over.

\mainline{31.a3} If \variation{31.Rxd6 Re1+, 32.Rd1 R8e2}.

\mainline{31... Re1+, 32.Rxe1 Rxe1+, 33.Ka2 Bf7, 34.Kb3 d5} the quickest way to finish the game.

\mainline{35.Bxf4 dxc4+, 36.Kb4 c3, 37.bxc3 Re4+, 38.c4 Rxc4+, 39.Ka5 Rxf4, 40.Rd8+ Kh7, 41. d7 Be6} Resigns.

A very lively game.

\begin{center}
\chessboard[normalboard,
moverstyle=triangle]
\end{center}

\chapter{Game 14: Queen's Gambit Declined}
New York, 1918. F.Marshall v J.Capablanca

\newgame
\mainline{1.d4 d5, 2.Nf3 Nf6, 3.c4 e6, 4.Nc3 Nbd7, 5.Bg5 Be7, 6.e3 O-O, 7.Rc1 c6} This is one of the oldest systems of defence against the Queen's Gambit. I had played it before in this Tournament against Kostic, and no doubt Marshall expected it. At times I change my defences, or rather systems of defence; on the other hand, during a Tournament, if one of them has given me good results, I generally play it all the time.

\mainline{8.Qc2 dxc4, 9.Bxc4 Nd5, 10.Bxe7 Qxe7, 11.O-O Nxc3, 12.Qxc3 b6} This is the key to this system of defence. Having simplified the game considerably by a series of exchanges, Black will now develop his light square Bishop along the long diagonal without having created any apparent weakness. The proper development of the light square Bishop is Black's greatest problem in the Queen's Gambit.

\mainline{13.e4 Bb7, 14.Rfe1 Rfd8}

\chessboard[smallboard,
marginleft=false,
marginrightwidth=2em,
moverstyle=triangle]
\begin{wraptable}{r}{0.5\textwidth}
	\vspace{-13em}

The developing stage can now be said to be complete on both sides. The opening is over and the middle-game begins. White, as is generally the case, has obtained the centre. Black, on the other hand, is entrenched in his first three ranks, and if given time will post his a Rook at c1 and his Knight at f6, 

\end{wraptable}

and finally play \bmove{c5}, in order to break up White's centre and give full action to the Black Bishop posted at b7. In this game White attempts to anticipate that plan by initiating an advance on the centre, which, when carefully analysed, is truly an attack against Black's e Pawn.

\mainline{15.d5 Nc5!} Against Kostic in a previous game I had played \bmove{Nf8}. It was carelessness on my part, but Marshall believed differently, otherwise he would not have played this variation, since, had he analysed this move, he would, I think, have realised that Black would obtain an excellent game. Black now threatens not only \bmove{cxd5}; but also \bmove{Nxe4}; followed by \bmove{cxd5}. The position is very interesting and full of possibilities.

\chessboard[smallboard,
marginleft=false,
marginrightwidth=2em,
moverstyle=triangle]
\begin{wraptable}{r}{0.5\textwidth}
	\vspace{-13em}

\mainline{16.dxe6 Nxe6, 17.Bxe6 Qxe6}

\end{wraptable}

played under the impression that White had to lose time in defending his a Pawn, when I could play \bmove{c5}, obtaining a very superior game. But, as will be seen, my opponent had quite a little surprise for me.

\mainline{18.Nd4!} 

\chessboard[smallboard,
marginleft=false,
marginrightwidth=2em,
moverstyle=triangle]
\begin{wraptable}{r}{0.5\textwidth}
	\vspace{-13em}

\mainline{18... Qe5} Of course, if \variation{18... Qxa2, 19.Ra1} would win the Queen. The next move is probably the only satisfactory move in the position.  

\end{wraptable}

The obvious move would have been \bmove{Qd7} to defend the c Pawn, and then would have come \variation{18... Qd7, 19.Nf5 f6, 20.Qg3 Kh8, 21.Rcd1 Qf7, 22.a4}, with a tremendous advantage in position. The text move, on the other hand, assures Black an even game at the very least, as will soon be seen.

\mainline{19.Nxc6 Qxc3, 20.Rxc3 Rd2, 21.Rb1} A very serious error of judgment. White is under the impression that he has the better game, because he is a Pawn ahead, but that is not so. The powerful position of the Black Rook at d2 fully compensates Black for the Pawn minus. Besides, the Bishop is better with Rooks than the Knight, and, as already stated, with Pawns on both sides of the board the Bishop is superior because of its long range. Incidentally, this ending will demonstrate the great power of the Bishop. White's best chance was to take a draw at once, thus. \variation{21.Ne7+ Kf8, 22.Rc7 Re8, 23.Rxb7 Rxe7, 24.Rb8+ Re8, 25.Rxe8+ Kxe8}, and with proper play White will draw.

It is curious that, although a Pawn ahead, White is the one who is always in danger. It is only now, after seeing this analysis, that the value of Black's 18th move \bmove{Qe5} can be fully appreciated.

\mainline{21... Re8} With this powerful move Black begins, against White's centre, an assault which will soon be shifted against the King itself. White is afraid to play \wmove{22.f3} because of \bmove{f5}.

\mainline{22.e5 g5} To prevent \wmove{f4}. The White Knight is practically pinned, because he does not dare move on account of \bmove{Rxe5}.

\chessboard[smallboard,
marginleft=false,
marginrightwidth=2em,
moverstyle=triangle]
\begin{wraptable}{r}{0.5\textwidth}
	\vspace{-13em}

\mainline{23.h4} This is a sequel to the previous move. White expects to disrupt Black's Pawns, and thus make them weak.

\end{wraptable}

\mainline{23... gxh4} Though doubled and isolated this Pawn exercises enormous pressure. Black now threatens \bmove{Re6}; to be followed by \bmove{Rg6} and \bmove{h3} and \bmove{h2} at the proper time.

\mainline{24.Re1} White cannot stand the slow death any longer. He sees danger everywhere, and wants to avert it by giving up his Queen's side Pawns, expecting to regain his fortunes later on by taking the initiative on the King's side.


\mainline{24... Re6!} Much better than taking Pawns. This forces White to defend the Knight with the Rook at e1, because of the threat \bmove{Rxc6}.

\mainline{25.Rec1 Kg7} Preparatory to \bmove{Rxc6}. The game is going to be decided on the King's side, and it is the isolated double Pawn that will supply the finishing touch.

\mainline{26.b4 b5} To prevent \wmove{b5}, defending the Knight and liberating the Rooks.


\mainline{27.a3 Rg6 28.Kf1 Ra2}

\chessboard[smallboard,
marginleft=false,
marginrightwidth=2em,
moverstyle=triangle]
\begin{wraptable}{r}{0.5\textwidth}
	\vspace{-13em}

Notice the remarkable position of the pieces. White cannot move anything without incurring some loss. His best chance would have been to play \wmove{29.e6}, but that would only have prolonged the game, which is lost in any case.

\end{wraptable}

\mainline{29.Kg1 h3, 30.g3 a6}Again forcing White to move and to lose something thereby, as all his pieces are tied up.

\chessboard[smallboard,
marginleft=false,
marginrightwidth=2em,
moverstyle=triangle]
\begin{wraptable}{r}{0.5\textwidth}
	\vspace{-13em}

\mainline{31.e6 Rxe6} Not even now can White move the Knight because of \wmove{32.Nd4 h2+, 33.Kxh2 Rh6+, 34.Kg1 Rh1#}.

\end{wraptable}

\mainline{32.g4 Rh6, 33.f3} If \variation{33.g5 h2+, 34.Kh1 Rxc6, 35.Rxc6 Rxf2}, winning easily.

\mainline{33... Rd6, 34.Ne7 Rdd2, 35.Nf5+ Kf6, 36.Nh4 Kg5, 37.Nf5 Rg2+, 38.Kf1 h2, 39.f4+ Kxf4} Resigns.

An ending worth very careful study.

\end{document}